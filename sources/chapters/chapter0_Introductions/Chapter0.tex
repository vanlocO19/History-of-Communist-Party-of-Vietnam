\chapter{Chương nhập môn: Đối tượng, chức năng, nhiệm vụ, nội dung và phương pháp nghiên cứu, học tập Lịch sử Đảng Cộng sản Việt Nam}
Đảng Cộng sản Việt Nam do Chủ tịch Hồ Chí Minh sáng lập ngày 03/2/1930. Từ thời điểm lịch sử đó, lịch sử của Đảng hòa quyện cùng lịch sử của dân tộc. Đảng đã lãnh đạo và đưa sự nghiệp cách mạng của giai cấp công nhân và dân tộc Việt Nam đi từ thắng lợi này đến thắng lợi khác, "có được cơ đồ và vị thế như ngày nay" \footfullcite[tr. 20]{HNTW4K12}. "\textit{Đảng Cộng sản Việt Nam} là đội tiền phong của giai cấp công nhân, đồng thười là đội tiền phong của nhân dân lao động và của dân tộc Việt Nam, đại biểu trung thành lợi ích của giai cấp công nhân, nhân dân lao động và của dân tộc. Đảng lấy chủ nghĩa Mác $-$ Lênin và tư tưởng Hồ Chí Minh làm nền tảng tư tưởng, kim chỉ nam cho hành động, lấy tập trung dân chủ làm nguyên tắc tổ chức cơ bản" \footfullcite[tr. 88]{DH11}.

Lịch sử Đảng Cộng sản Việt Nam là một chuyên ngành, một bộ phận của khoa học lịch sử. Chuyên ngành lịch sử Đảng Cộng sản Việt Nam đã được nghiên cứu từ rất sớm. Năm 1933, tác giả Hồng Thế Công (tức Hà Huy Tập) đã công bố tác phẩm \textit{Sơ thảo lịch sử phong trào cộng sản Đông Dương}. Ở các thời kỳ lịch sử của Đảng, Hồ Chí Minh và các nhà lãnh đạo đã trình bày lịch sử và có những tổng kết quan trọng. Đại hội đại biểu toàn quốc lần thứ III của Đảng (1960) đã nêu rõ nhiệm vụ nghiên cứu, tổng kết lịch sử Đảng, nhất là tổng kết kinh nghiệm, bài học lãnh đạo của Đảng, con đường và quy luật phát triển của cách mạng Việt Nam.

Năm 1962, cơ quan chuyên trách nghiên cứu, biên soạn lịch sử Đảng là Ban Nghiên cứu Lịch sử Đảng Trung ương được thành lập (nay là Viện Lịch sử Đảng). Từ những năm 60 của thế kỷ XX, bộ môn lịch sử Đảng đã được giảng dạy, học tập chính thức trong các trường đại học, cao đẳng, trung cấp chuyên nghiệp. Theo sự chỉ đạo của Bộ Chính trị khóa VII, ngày 13/7/1992, Chủ tịch Hội đồng Bộ trưởng (nay là Thủ tướng Chính phủ) đã ban hành Quyết định số 255CT thành lập Hội đồng chỉ đạo biên soạn giáo trình quốc gia các bộ môn khoa học Mác - Lênin, tư tưởng Hồ Chí Minh, trong đó có bộ môn Lịch sử Đảng Cộng sản Việt Nam.

Giáo trình bộ môn Lịch sử Đảng Cộng sản Việt Nam dùng trong các trường đại học được biên soạn lần này là sự kế thừa và phát triển các giáo trình đã biên soạn trước đây, phù hợp với yêu cầu đổi mới căn bản, toàn diện giáo dục và đào tạo theo quan điểm của Đảng.

\section{Đối tượng nghiên cứu của môn học Lịch sử Đảng Cộng sản Việt Nam}
\textit{Đối tượng nghiên cứu của khoa học Lịch sử Đảng là sự ra đời, phát triển và hoạt động lãnh đạo của Đảng qua các thời kỳ lịch sử.}

\begin{enumerate}
\item Trước hết là các sự kiện lịch sử Đảng. Cần phân biệt rõ sự kiện lịch sử Đảng gắn trực tiếp với sự lãnh đạo của Đảng. Phân biệt sự kiện lịch sử Đảng với sự kiện lịch sử dân tộc và lịch sử quân sự trong cùng thời kỳ, thời điểm lịch sử. Môn học lịch sử Đảng Cộng sản Việt Nam nghiên cứu sâu sắc, có hệ thống \textit{các sự kiện lịch sử Đảng}, hiểu rõ nội dung, tính chất, bản chất, của các sự kiện đó gắn liền với sự lãnh đạo của Đảng. Các sự kiện thể hiện quá trình Đảng ra đời, phát triển và lãnh đạo sự nghiệp giải phóng dân tộc, kháng chiến cứu quốc và xây dựng, phát triển đất nước theo con đường xã hội chủ nghĩa, trên các lĩnh vực chính trị, quân sự, kihn tế, xã hội, văn hóa, quốc phòng, an ninh, đối ngoại, ...

Sự kiện lịch sử Đảng là hoạt động lãnh đạo, đấu tranh phong phú và oanh liệt của Đảng làm sáng rõ bản chất cách mạng của Đảng với tư cách là một đảng chính trị, "là đội tiên phong của giai cấp công nhân, đồng thời là đội tiên phong của nhân dân lao động và của dân tộc Việt Nam, đại biểu trung thành lợi ích của giai cấp công nhân, nhân dân lao động và của dân tộc". Hệ thống các sự kiện lịch sử Đảng làm rõ thắng lợi, thành tựu của cách mạng, đồng thời cũng thấy rõ những khó khăn, thách thức, hiểu rõ những hi sinh, cống hiến lớn lao của toàn Đảng, toàn dân, toàn quân, sự hi sinh, phấn đấu của các tổ chức lãnh đạo của Đảng từ Trung ương tới cơ sở, của mỗi cán bộ, đảng viên, với những tấm gương tiêu biểu. Các sự kiện phải được tái hiện trên cơ sở tư liệu lịch sử chính xác, trung thực, khách quan.
\item Đảng lãnh đạo cách mạng giải phóng dân tộc, xây dựng và phát triển đất nước bằng \textit{Cương lĩnh, đường lối, chủ trương, chính sách lớn}. Lịch sử Đảng có đối tượng nghiên cứu là Cương lĩnh, đường lối của Đảng, phải nghiên cứu, làm sáng tỏ nội dung Cương lĩnh, đường lối của Đảng, cơ sở lý luận, thực tiễn và giá trị hiện thực của đường lối trong tiến trình phát triển của cách mạng Việt Nam. Cương lĩnh, đường lối đúng đắn là điều kiện trước hết quyết định thắng lợi của cách mạng. Phải không ngừng bổ sung, phát triển đường lối phù hợp với sự phát triển của lý luận và thực tiễn và yêu cầu của cuộc sống; chống nguy cơ sai lầm về đường lối, nếu sai lầm về đường lối sẽ dẫn tới đổ vỡ, thất bại.

Đảng đề ra Cương lĩnh chính trị đầu tiên (2/1930); Luận chương chính trị (10/1930); Chính cương của Đảng (2/1951); Cương lĩnh xây dựng đất nước trong thời kỳ quá độ lên chủ nghĩa xã hội (6/1991) và bổ sung, phát triển năm 2011. Quá trình lãnh đạo, Đảng đề ra đường lối nhằm cụ thể hóa Cương lĩnh trên những vấn đề nổi bật ở mỗi thời kỳ lịch sử. Đường lối cách mạng giải phóng dân tộc. Đường lối kháng chiến bảo vệ Tổ quốc. Đường lối cách mạng dân tộc dân chủ nhân dân. Đường lối cách mạng xã hội chủ nghĩa. Đường lối đổi mới. Đường lối quân sự. Đường lối đối ngoại v.v. Đảng quyết định những vấn đề chiến lược, sách lược và phương pháp cách mạng. Đảng là người tổ chức phong trào cách mạng của quần chúng nhân dân hiện thực hóa đường lối đưa đến thắng lợi.

\item Đảng lãnh đạo thông qua quá trình \textit{chỉ đạo, tổ chức} thực tiễn trong tiến trình cách mạng. Nghiên cứu, học tập lịch sử Đảng Cộng sản Việt Nam làm rõ \textit{thắng lợi, thành tựu, kinh nghiệm, bài học của cách mạng Việt Nam} do Đảng lãnh đạo trong sự nghiệp giải phóng dân tộc, kháng chiến giành độc lập, thống nhất, thành tựu của công cuộc đổi mới. Từ một quốc gia phong kiến, kinh tế nông nghiệp lạc hậu, một nước thuộc địa, bị đế quốc, thực dân cai trị, dân tộc Việt Nam đã giành lại độc lập bừng cuộc Cách mạng tháng Tám năm 1945 với bản \textit{Tuyên ngôn độc lập} lịch sử; tiến hành hai cuộc kháng chiến giải phóng, bảo vệ Tổ quốc, thống nhất đất nước; thực hiện công cuộc đổi mới đưa đất nước quá độ lên chủ nghĩa xã hội với những thành tựu to lớn, có ý nghĩa lịch sử. Đảng cũng thẳng thắn nêu rõ những khuyết điểm, hạn chế, khó khăn, thách thức, nguy cơ cần phải khắc phục, vượt qua.

Nghiên cứu, học tập lịch sử Đảng là giáo dục sâu sắc những kinh nghiệm, bài học trong lãnh đạo của Đảng. Tổng kết kinh nghiệm, bài học, tìm ra quy luật riêng của cách mạng Việt Nam là công việc thường xuyên của Đảng ở mỗi thời kỳ lịch sử. Đó là nội dung và yêu cầu của công tác lý luận, tư tưởng của Đảng, nâng cao trình độ lý luận, trí tuệ của Đảng. Lịch sử Đảng là quá trình nhận thức, vận dụng và phát triển sáng tạo chủ nghĩa Mác - Lênin, tư tưởng Hồ Chí Minh vào thực tiễn Việt Nam. Cần nhận thức rõ và chú trọng giáo dục những truyền thống nổi bật của Đảng; truyền thống đấu tranh kiên cường, bất khuất của Đảng; truyền thống đoàn kết, thống nhất trong Đảng; truyền thống gắn bó mật thiết với nhân dân, vì lợi ích quốc gia, dân tộc; truyền thống của chủ nghĩa quốc tế trong sáng.
\item Nghiên cứu Lịch sử Đảng là làm rõ hệ thống tổ chức Đảng, công tác xây dựng Đảng qua các thời kỳ lịch sử. Nghiên cứu, học tập lịch sử Đảng để nêu cao hiểu biết về \textit{công tác xây dựng Đảng trong các thời kỳ lịch sử} về chính trị, tư tưởng, tổ chức và đạo đức. Xây dựng Đảng về chính trị bảo đảm tính đúng đắn của đường lối, củng cố chính trị nội bộ và nâng cao bản lĩnh chính trị của Đảng. Xây dựng Đảng về tư tưởng "Đảng lấy chủ nghĩa Mác $-$ Lênin, tư tưởng Hồ Chí Minh làm nền tảng tư tưởng, kim chỉ nam cho hành động. Xây dựng Đảng về tổ chức, củng cố, phát triển hệ thống tổ chức và đội ngũ cán bộ, đảng viên của Đảng, tuân thủ các nguyên tắc tổ chức cơ bản". Xây dựng Đảng về đạo đức với những chuẩn mực về đạo đức trong Đảng và ngăn chặn, đẩy lùi sự suy thoái đạo đức, lối sống của một bộ phận cán bộ, đảng viên hiện nay.
\end{enumerate}
\section{Chức năng, nhiệm vụ của môn học Lịch sử Đảng Cộng sản Việt Nam}
Là một ngành của khoa học lịch sử, Lịch sử Đảng Cộng sản Việt Nam có chức năng, nhiệm vụ của khoa học lịch sử, đồng thời có những điểm cần nhấn mạnh.
\subsection{Chức năng của khoa học Lịch sử Đảng}
Trước hết đó là \textit{chức năng nhận thức}. Nghiên cứu và học tập lịch sử Đảng Cộng sản Việt Nam để nhận thức đầy đủ, có hệ thống những tri thức lịch sử lãnh đạo, đấu tranh và cầm quyền của Đảng, nhận thức rõ về Đảng với tư cách một Đảng chính trị $-$ tổ chức lãnh đạo của giai cấp công nhân, nhân dân lao động và dân tộc Việt Nam. Quy luật ra đời và phát triển của Đảng là sự kết hợp chủ nghĩa Mác $-$ Lênin với phong trào công nhân và phong trào yêu nước Việt Nam. Đảng được trang bị học thuyết lý luận, có Cương lĩnh, đường lối rõ ràng, có tổ chức, kỷ luật chặt chẽ, hoạt động có nguyên tắc. Từ năm 1930 đến nay, Đảng là tổ chức lãnh đạo duy nhất của cách mạng Việt Nam. TỪ Cách mạng tháng Tám năm 1945, Đảng trở thành Đảng cầm quyền, nghĩa là Đảng nắm chính quyền, lãnh đạo Nhà nước và xã hội. Sự lãnh đạo đúng đắn của Đảng là nhân tố hàng đầu quyết định thắng lợi của cách mạng. Đảng thường xuyên tự xây dựng và chỉnh đốn để hoàn thành sứ mệnh lịch sử trước đất nước và dân tộc.

Nghiên cứu, học tập lịch sử Đảng Cộng sản Việt Nam còn nhằm nâng cao nhận thức về thời đại mới của dân tộc $-$ thời đại Hồ Chí Minh, góp phần bồi đắp nhận thức lý luận từ thực tiễn Việt Nam. Nâng cao nhận thức về giác ngộ chính trị, góp phần làm rõ những vấn đề của khoa học chính trị (chính trị học) và khoa học lãnh đạo, quản lý. Nhận thức rõ những vấn đề lớn của đất nước, dân tộc trong mối quan hệ với những vấn đề của thời đại và thế giới. Tổng kết lịch sử Đảng để nhận thức quy luật của cách mạng giải phóng dân tộc, xây dựng và bảo vệ Tổ quốc, quy luật đi lên chủ nghĩa xã hội ở Việt Nam. Năng lực nhận thức và hành động theo quy luật là điều kiện bảo đảm sự lãnh đạo đúng đắn của Đảng. 

Nghiên cứu, biên soạn, giảng dạy, học tập lịch sử Đảng Cộng sản Việt Nam cần quán triệt \textit{chức năng giáo dục} của khoa học lịch sử. Giáo dục sâu sắc tinh thần yêu nước, ý thức, niềm tự hào, tự tôn, ý chí tự lực, tự cường dân tộc. Tinh thần đó hình thành trong lịch sử dụng nước, giữ nước của dân tộc và phát triển đến đỉnh cao ở thời kỳ Đảng lãnh đạo sự nghiệp cách mạng của dân tộc. Lịch sử Đảng Cộng sản Việt Nam giáo dục lý tưởng cách mạng với mục tiêu chiến lược là độc lập dân tộc và chủ nghĩa xã hội. Đó cũng là sự giáo dục tư tưởng chính trị, nâng cao nhận thức tư tưởng, lý luận, con đường phát triển của cách mạng và dân tộc Việt Nam. Lịch sử Đảng Cộng sản Việt Nam giáo dục chủ nghĩa anh hùng cách mạng, tinh thần chiến đấu bất khuất, đức hi sinh, tính tiên phong gương mẫu của các tổ chức đảng, những chiến sĩ cống sản tiêu biểu trong sự nghiệp đấu tranh giải phóng dân tộc và phát triển đất nước. Lịch sử Đảng Cộng sản Việt Nam có vai trò quan trọng trong giáo dục truyền thống của Đảng và dân tộc, góp phần giáo dục đạo đức cách mạng, nhân cách, lối sống cao đẹp như Hồ Chí Minh khẳng định: "Đảng ta là đạo đức, là văn minh".

Cùng với hai chức năng cơ bản của sử học là \textit{nhận thức} và \textit{giáo dục}, khoa học lịch sử Đảng còn có chức năng \textit{dự báo} và \textit{phê phán}. Từ nhận thức những gì đã diễn ra trong quá khứ để hiểu rõ hiện tại và dự báo tương lai của sự phát triển. Năm 1942, trong tác phẩm \textit{Lịch sử nước ta}, Hồ Chí Minh đã dự báo: "Năm 1945 Việt Nam độc lập". Sau này, Người còn nhiều lần dự báo chính xác trong 2 cuộc kháng chiến. Lãnh đạo đòi hỏi phải thấy trước. Hiện nay, Đảng nhấn mạnh năng lực tự dự báo. Để tăng cường sự lãnh đạo, nâng cao sức chiến đấu của Đảng, tự phê bình và phê bình là quy luật phát triển của Đảng. Phải kiên quyết phê phán những biểu hiện tiêu cực, lạc hậu, hư hỏng. Hiện nay, sự phê phán nhằm ngăn chặn, đẩy lùi sự suy thoái về tư tưởng chính trị, đạo đức, lối sống và những biểu hiện "tự diễn biến", "tự chuyển hóa" trong nội bộ.

\subsection{Nhiệm vụ của khoa học lịch sử Đảng}
Nhiệm vụ của khoa học lịch sử Đảng được đặt ra từ đối tượng nghiên cứu đồng thười cụ thể hóa chức năng của khoa học lịch sử Đảng.
\begin{itemize}
\item \textit{Nhiệm vụ trình bày có hệ thống Cương lĩnh, đường lối của Đảng}. Khoa học lịch sử Đảng có \textit{nhiệm vụ hằng đầu là khẳng định, chứng minh giá trị khoa học và hiện thực của những mục tiêu chiến lược và sách lược cách mạng} mà Đảng đề ra trong Cương lĩnh, đường lối từ khi Đảng ra đời và suốt quá trình lãnh đạo cách mạng. Mục tiêu và con đường đó là sự kết hợp, thống nhất giữa thực tiễn lịch sử với nền tảng lý luận nhằm thúc đẩy tiến trình cách mạng, nhận thức và cải biến đất nước, xã hội theo con đường đúng đắn. Sự lựa chọn mục tiêu độc lập dân tộc gắn liền với chủ nghĩa xã hội phù hợp với quy luật tiến hóa của lịch sử, đã và đang được hiện thực hóa.
\item \textit{Nhiệm vụ tái hiện tiến trình lịch sử lãnh đạo, đấu tranh của Đảng}. Từ hiện thực lịch sử và các nguồn tư liệu thành văn và không thành văn, \textit{khoa học lịch sử Đảng có nhiệm vụ rất quan trọng là làm rõ những sự kiện lịch sử, làm nổi bật các thời kỳ, giai đoạn và dấu mốc phát triển căn bản của tiến trình lịch sử}, nghĩa là tái hiện quá trình lịch sử lãnh đạo và đấu tranh của Đảng. Những kiến thức, tri thức lịch sử Đảng được làm sáng tỏ từ vai trò lãnh đạo, hoạt động thực tiễn của Đảng, vai trò, sức mạnh của nhân dân, của khối đại đoàn kết toàn dân tộc. Hoạt động của Đảng không biệt lập mà thống nhất và khơi dậy mạnh mẽ nguồn sức mạnh từ giai cấp công nhân, nhân dân lao động và toàn dân tộc.
\item \textit{Nhiệm vụ tổng kết lịch sử của Đảng}. Lịch sử Đảng Cộng sản Việt nam không dừng lại mô tả, tái hiện sự kiện và tiến trình lịch sử, mà còn \textit{có nhiệm vụ tổng kết từng chặng đường và suốt tiến trình lịch sử, làm rõ kinh nghiệm, bài học, quy luật và những vấn đề lý luận} của cách mạng Việt Nam. Kinh nghiệm lịch sử gắn liền với những sự kiện hoặc một giai đoạn lịch sử nhất định. Bài học lịch sử khái quát cao hơn gắn liền với một thời kỳ dài, một vấn đề của chiến lược cách mạng hoặc khái quát toàn bộ tiến trình lịch sử của Đảng. Quy luật và những vấn đề lý luận ở tầm tổng kết cao hơn. Hồ Chí Minh nêu rõ:

"Lý luận là đem \textit{thực tế} trong lịch sử, trong kinh nghiệm, trong các cuộc tranh đấu, xem xét, so sánh thật kỹ lưỡng, rõ ràng, làm thành kết luận. Rồi lại đem nó chứng minh với thực tế. Đó là lý luận chân chính" \footfullcite[tr. 273]{HCMtt5}.

"Lý luận do kinh nghiệm cách mạng ở các nước và trong nước ta, do kinh nghiệm từ trước và kinh nghiệm hiện nay gom góp phân tích và kết luận những kinh nghiệm đó thành ra lý luận" \footfullcite[tr. 312]{HCMtt5}.

Hồ Chí Minh nhiều lần đặt ra yêu cầu phải tổng kết, tìm ra quy luật riêng của cách mạng Việt Nam. Qua nhiều lần tổng kết, Đàng Cộng sản Việt Nam khẳng định:

"Con đường đi lên chủ nghĩa xã hội ở nước ta ngày càng được xác định rõ hơn" \footfullcite[tr. 356]{VKDtt55}.

"Con đường đi lên chủ nghĩa xã hội ở nước ta là phù hợp với thực tiễn của Việt Nam và xu thế phát triển của lịch sử" \footfullcite[tr. 66]{DH12}.
\item Một nhiệm vụ quan trọng của lịch sử Đảng là làm rõ vai trò, sức chiến đấu của hệ thống tổ chức đảng từ Trung ương ddeeens cơ sở trong lãnh đạo, tổ chức thực tiễn. Những truyền thống nổi bật của Đảng. Trí tuệ, tính tiên phong, gương mẫu, bản lĩnh của cán bộ, đảng viên. Tấm gương của Chủ tịch Hồ Chí Minh và các nhà lãnh đạo, những chiến sĩ cộng sản tiêu biểu trong các thời kỳ cách mạng. Những giá trị truyền thống, đức hi sinh và tấm gương tiêu biểu luôn luôn là động lực cho sự phát triển và bản chất cách mạng của Đảng. Nghị quyết Trung ương 4 khóa XII (tháng 10/2016) khẳng định: "Chúng ta có quyền tự hào về bản chất tốt đẹp, truyền thống anh hùng và lịch sử vẻ vang của Đảng ta $-$ Đảng của Chủ tịch Hồ Chí Minh vĩ đại, đại biểu của dân tộc Việt Nam anh hùng". Có một nhiệm vụ là hoàn thiện hệ thống tư liệu lịch sử Đảng.
\end{itemize}
\section{Phương pháp nghiên cứu, học tập môn học Lịch sử Đảng Cộng sản Việt Nam}