\chapter{Chương nhập môn: Đối tượng, chức năng, nhiệm vụ, nội dung và phương pháp nghiên cứu, học tập Lịch sử Đảng Cộng sản Việt Nam}
Đảng Cộng sản Việt Nam do Chủ tịch Hồ Chí Minh sáng lập ngày 03/2/1930. Từ thời điểm lịch sử đó, lịch sử của Đảng hòa quyện cùng lịch sử của dân tộc. Đảng đã lãnh đạo và đưa sự nghiệp cách mạng của giai cấp công nhân và dân tộc Việt Nam đi từ thắng lợi này đến thắng lợi khác, "có được cơ đồ và vị thế như ngày nay" \footfullcite[p. 20]{HNTW4K12}. "\textit{Đảng Cộng sản Việt Nam} là đội tiền phong của giai cấp công nhân, đồng thười là đội tiền phong của nhân dân lao động và của dân tộc Việt Nam, đại biểu trung thành lợi ích của giai cấp công nhân, nhân dân lao động và của dân tộc. Đảng lấy chủ nghĩa Mác $-$ Lênin và tư tưởng Hồ Chí Minh làm nền tảng tư tưởng, kim chỉ nam cho hành động, lấy tập trung dân chủ làm nguyên tắc tổ chức cơ bản" \footfullcite[p. 88]{DH11}.