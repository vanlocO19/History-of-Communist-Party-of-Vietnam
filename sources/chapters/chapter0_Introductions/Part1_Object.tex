\section{Đối tượng nghiên cứu của môn học Lịch sử Đảng Cộng sản Việt Nam}
\textit{Đối tượng nghiên cứu của khoa học Lịch sử Đảng là sự ra đời, phát triển và hoạt động lãnh đạo của Đảng qua các thời kỳ lịch sử.}

\begin{enumerate}
\item Trước hết là các sự kiện lịch sử Đảng. Cần phân biệt rõ sự kiện lịch sử Đảng gắn trực tiếp với sự lãnh đạo của Đảng. Phân biệt sự kiện lịch sử Đảng với sự kiện lịch sử dân tộc và lịch sử quân sự trong cùng thời kỳ, thời điểm lịch sử. Môn học lịch sử Đảng Cộng sản Việt Nam nghiên cứu sâu sắc, có hệ thống \textit{các sự kiện lịch sử Đảng}, hiểu rõ nội dung, tính chất, bản chất, của các sự kiện đó gắn liền với sự lãnh đạo của Đảng. Các sự kiện thể hiện quá trình Đảng ra đời, phát triển và lãnh đạo sự nghiệp giải phóng dân tộc, kháng chiến cứu quốc và xây dựng, phát triển đất nước theo con đường xã hội chủ nghĩa, trên các lĩnh vực chính trị, quân sự, kihn tế, xã hội, văn hóa, quốc phòng, an ninh, đối ngoại, ...

Sự kiện lịch sử Đảng là hoạt động lãnh đạo, đấu tranh phong phú và oanh liệt của Đảng làm sáng rõ bản chất cách mạng của Đảng với tư cách là một đảng chính trị, "là đội tiên phong của giai cấp công nhân, đồng thời là đội tiên phong của nhân dân lao động và của dân tộc Việt Nam, đại biểu trung thành lợi ích của giai cấp công nhân, nhân dân lao động và của dân tộc". Hệ thống các sự kiện lịch sử Đảng làm rõ thắng lợi, thành tựu của cách mạng, đồng thời cũng thấy rõ những khó khăn, thách thức, hiểu rõ những hi sinh, cống hiến lớn lao của toàn Đảng, toàn dân, toàn quân, sự hi sinh, phấn đấu của các tổ chức lãnh đạo của Đảng từ Trung ương tới cơ sở, của mỗi cán bộ, đảng viên, với những tấm gương tiêu biểu. Các sự kiện phải được tái hiện trên cơ sở tư liệu lịch sử chính xác, trung thực, khách quan.
\item Đảng lãnh đạo cách mạng giải phóng dân tộc, xây dựng và phát triển đất nước bằng \textit{Cương lĩnh, đường lối, chủ trương, chính sách lớn}. Lịch sử Đảng có đối tượng nghiên cứu là Cương lĩnh, đường lối của Đảng, phải nghiên cứu, làm sáng tỏ nội dung Cương lĩnh, đường lối của Đảng, cơ sở lý luận, thực tiễn và giá trị hiện thực của đường lối trong tiến trình phát triển của cách mạng Việt Nam. Cương lĩnh, đường lối đúng đắn là điều kiện trước hết quyết định thắng lợi của cách mạng. Phải không ngừng bổ sung, phát triển đường lối phù hợp với sự phát triển của lý luận và thực tiễn và yêu cầu của cuộc sống; chống nguy cơ sai lầm về đường lối, nếu sai lầm về đường lối sẽ dẫn tới đổ vỡ, thất bại.

Đảng đề ra Cương lĩnh chính trị đầu tiên (2/1930); Luận chương chính trị (10/1930); Chính cương của Đảng (2/1951); Cương lĩnh xây dựng đất nước trong thời kỳ quá độ lên chủ nghĩa xã hội (6/1991) và bổ sung, phát triển năm 2011. Quá trình lãnh đạo, Đảng đề ra đường lối nhằm cụ thể hóa Cương lĩnh trên những vấn đề nổi bật ở mỗi thời kỳ lịch sử. Đường lối cách mạng giải phóng dân tộc. Đường lối kháng chiến bảo vệ Tổ quốc. Đường lối cách mạng dân tộc dân chủ nhân dân. Đường lối cách mạng xã hội chủ nghĩa. Đường lối đổi mới. Đường lối quân sự. Đường lối đối ngoại v.v. Đảng quyết định những vấn đề chiến lược, sách lược và phương pháp cách mạng. Đảng là người tổ chức phong trào cách mạng của quần chúng nhân dân hiện thực hóa đường lối đưa đến thắng lợi.

\item Đảng lãnh đạo thông qua quá trình \textit{chỉ đạo, tổ chức} thực tiễn trong tiến trình cách mạng. Nghiên cứu, học tập lịch sử Đảng Cộng sản Việt Nam làm rõ \textit{thắng lợi, thành tựu, kinh nghiệm, bài học của cách mạng Việt Nam} do Đảng lãnh đạo trong sự nghiệp giải phóng dân tộc, kháng chiến giành độc lập, thống nhất, thành tựu của công cuộc đổi mới. Từ một quốc gia phong kiến, kinh tế nông nghiệp lạc hậu, một nước thuộc địa, bị đế quốc, thực dân cai trị, dân tộc Việt Nam đã giành lại độc lập bừng cuộc Cách mạng tháng Tám năm 1945 với bản \textit{Tuyên ngôn độc lập} lịch sử; tiến hành hai cuộc kháng chiến giải phóng, bảo vệ Tổ quốc, thống nhất đất nước; thực hiện công cuộc đổi mới đưa đất nước quá độ lên chủ nghĩa xã hội với những thành tựu to lớn, có ý nghĩa lịch sử. Đảng cũng thẳng thắn nêu rõ những khuyết điểm, hạn chế, khó khăn, thách thức, nguy cơ cần phải khắc phục, vượt qua.

Nghiên cứu, học tập lịch sử Đảng là giáo dục sâu sắc những kinh nghiệm, bài học trong lãnh đạo của Đảng. Tổng kết kinh nghiệm, bài học, tìm ra quy luật riêng của cách mạng Việt Nam là công việc thường xuyên của Đảng ở mỗi thời kỳ lịch sử. Đó là nội dung và yêu cầu của công tác lý luận, tư tưởng của Đảng, nâng cao trình độ lý luận, trí tuệ của Đảng. Lịch sử Đảng là quá trình nhận thức, vận dụng và phát triển sáng tạo chủ nghĩa Mác - Lênin, tư tưởng Hồ Chí Minh vào thực tiễn Việt Nam. Cần nhận thức rõ và chú trọng giáo dục những truyền thống nổi bật của Đảng; truyền thống đấu tranh kiên cường, bất khuất của Đảng; truyền thống đoàn kết, thống nhất trong Đảng; truyền thống gắn bó mật thiết với nhân dân, vì lợi ích quốc gia, dân tộc; truyền thống của chủ nghĩa quốc tế trong sáng.
\item Nghiên cứu Lịch sử Đảng là làm rõ hệ thống tổ chức Đảng, công tác xây dựng Đảng qua các thời kỳ lịch sử. Nghiên cứu, học tập lịch sử Đảng để nêu cao hiểu biết về \textit{công tác xây dựng Đảng trong các thời kỳ lịch sử} về chính trị, tư tưởng, tổ chức và đạo đức. Xây dựng Đảng về chính trị bảo đảm tính đúng đắn của đường lối, củng cố chính trị nội bộ và nâng cao bản lĩnh chính trị của Đảng. Xây dựng Đảng về tư tưởng "Đảng lấy chủ nghĩa Mác $-$ Lênin, tư tưởng Hồ Chí Minh làm nền tảng tư tưởng, kim chỉ nam cho hành động. Xây dựng Đảng về tổ chức, củng cố, phát triển hệ thống tổ chức và đội ngũ cán bộ, đảng viên của Đảng, tuân thủ các nguyên tắc tổ chức cơ bản". Xây dựng Đảng về đạo đức với những chuẩn mực về đạo đức trong Đảng và ngăn chặn, đẩy lùi sự suy thoái đạo đức, lối sống của một bộ phận cán bộ, đảng viên hiện nay.
\end{enumerate}