\section{Chức năng, nhiệm vụ của môn học Lịch sử Đảng Cộng sản Việt Nam}
Là một ngành của khoa học lịch sử, Lịch sử Đảng Cộng sản Việt Nam có chức năng, nhiệm vụ của khoa học lịch sử, đồng thời có những điểm cần nhấn mạnh.
\subsection{Chức năng của khoa học Lịch sử Đảng}
Trước hết đó là \textit{chức năng nhận thức}. Nghiên cứu và học tập lịch sử Đảng Cộng sản Việt Nam để nhận thức đầy đủ, có hệ thống những tri thức lịch sử lãnh đạo, đấu tranh và cầm quyền của Đảng, nhận thức rõ về Đảng với tư cách một Đảng chính trị $-$ tổ chức lãnh đạo của giai cấp công nhân, nhân dân lao động và dân tộc Việt Nam. Quy luật ra đời và phát triển của Đảng là sự kết hợp chủ nghĩa Mác $-$ Lênin với phong trào công nhân và phong trào yêu nước Việt Nam. Đảng được trang bị học thuyết lý luận, có Cương lĩnh, đường lối rõ ràng, có tổ chức, kỷ luật chặt chẽ, hoạt động có nguyên tắc. Từ năm 1930 đến nay, Đảng là tổ chức lãnh đạo duy nhất của cách mạng Việt Nam. TỪ Cách mạng tháng Tám năm 1945, Đảng trở thành Đảng cầm quyền, nghĩa là Đảng nắm chính quyền, lãnh đạo Nhà nước và xã hội. Sự lãnh đạo đúng đắn của Đảng là nhân tố hàng đầu quyết định thắng lợi của cách mạng. Đảng thường xuyên tự xây dựng và chỉnh đốn để hoàn thành sứ mệnh lịch sử trước đất nước và dân tộc.

Nghiên cứu, học tập lịch sử Đảng Cộng sản Việt Nam còn nhằm nâng cao nhận thức về thời đại mới của dân tộc $-$ thời đại Hồ Chí Minh, góp phần bồi đắp nhận thức lý luận từ thực tiễn Việt Nam. Nâng cao nhận thức về giác ngộ chính trị, góp phần làm rõ những vấn đề của khoa học chính trị (chính trị học) và khoa học lãnh đạo, quản lý. Nhận thức rõ những vấn đề lớn của đất nước, dân tộc trong mối quan hệ với những vấn đề của thời đại và thế giới. Tổng kết lịch sử Đảng để nhận thức quy luật của cách mạng giải phóng dân tộc, xây dựng và bảo vệ Tổ quốc, quy luật đi lên chủ nghĩa xã hội ở Việt Nam. Năng lực nhận thức và hành động theo quy luật là điều kiện bảo đảm sự lãnh đạo đúng đắn của Đảng. 

Nghiên cứu, biên soạn, giảng dạy, học tập lịch sử Đảng Cộng sản Việt Nam cần quán triệt \textit{chức năng giáo dục} của khoa học lịch sử. Giáo dục sâu sắc tinh thần yêu nước, ý thức, niềm tự hào, tự tôn, ý chí tự lực, tự cường dân tộc. Tinh thần đó hình thành trong lịch sử dụng nước, giữ nước của dân tộc và phát triển đến đỉnh cao ở thời kỳ Đảng lãnh đạo sự nghiệp cách mạng của dân tộc. Lịch sử Đảng Cộng sản Việt Nam giáo dục lý tưởng cách mạng với mục tiêu chiến lược là độc lập dân tộc và chủ nghĩa xã hội. Đó cũng là sự giáo dục tư tưởng chính trị, nâng cao nhận thức tư tưởng, lý luận, con đường phát triển của cách mạng và dân tộc Việt Nam. Lịch sử Đảng Cộng sản Việt Nam giáo dục chủ nghĩa anh hùng cách mạng, tinh thần chiến đấu bất khuất, đức hi sinh, tính tiên phong gương mẫu của các tổ chức đảng, những chiến sĩ cống sản tiêu biểu trong sự nghiệp đấu tranh giải phóng dân tộc và phát triển đất nước. Lịch sử Đảng Cộng sản Việt Nam có vai trò quan trọng trong giáo dục truyền thống của Đảng và dân tộc, góp phần giáo dục đạo đức cách mạng, nhân cách, lối sống cao đẹp như Hồ Chí Minh khẳng định: "Đảng ta là đạo đức, là văn minh".

Cùng với hai chức năng cơ bản của sử học là \textit{nhận thức} và \textit{giáo dục}, khoa học lịch sử Đảng còn có chức năng \textit{dự báo} và \textit{phê phán}. Từ nhận thức những gì đã diễn ra trong quá khứ để hiểu rõ hiện tại và dự báo tương lai của sự phát triển. Năm 1942, trong tác phẩm \textit{Lịch sử nước ta}, Hồ Chí Minh đã dự báo: "Năm 1945 Việt Nam độc lập". Sau này, Người còn nhiều lần dự báo chính xác trong 2 cuộc kháng chiến. Lãnh đạo đòi hỏi phải thấy trước. Hiện nay, Đảng nhấn mạnh năng lực tự dự báo. Để tăng cường sự lãnh đạo, nâng cao sức chiến đấu của Đảng, tự phê bình và phê bình là quy luật phát triển của Đảng. Phải kiên quyết phê phán những biểu hiện tiêu cực, lạc hậu, hư hỏng. Hiện nay, sự phê phán nhằm ngăn chặn, đẩy lùi sự suy thoái về tư tưởng chính trị, đạo đức, lối sống và những biểu hiện "tự diễn biến", "tự chuyển hóa" trong nội bộ.

\subsection{Nhiệm vụ của khoa học lịch sử Đảng}
Nhiệm vụ của khoa học lịch sử Đảng được đặt ra từ đối tượng nghiên cứu đồng thười cụ thể hóa chức năng của khoa học lịch sử Đảng.
\begin{itemize}
\item \textit{Nhiệm vụ trình bày có hệ thống Cương lĩnh, đường lối của Đảng}. Khoa học lịch sử Đảng có \textit{nhiệm vụ hằng đầu là khẳng định, chứng minh giá trị khoa học và hiện thực của những mục tiêu chiến lược và sách lược cách mạng} mà Đảng đề ra trong Cương lĩnh, đường lối từ khi Đảng ra đời và suốt quá trình lãnh đạo cách mạng. Mục tiêu và con đường đó là sự kết hợp, thống nhất giữa thực tiễn lịch sử với nền tảng lý luận nhằm thúc đẩy tiến trình cách mạng, nhận thức và cải biến đất nước, xã hội theo con đường đúng đắn. Sự lựa chọn mục tiêu độc lập dân tộc gắn liền với chủ nghĩa xã hội phù hợp với quy luật tiến hóa của lịch sử, đã và đang được hiện thực hóa.
\item \textit{Nhiệm vụ tái hiện tiến trình lịch sử lãnh đạo, đấu tranh của Đảng}. Từ hiện thực lịch sử và các nguồn tư liệu thành văn và không thành văn, \textit{khoa học lịch sử Đảng có nhiệm vụ rất quan trọng là làm rõ những sự kiện lịch sử, làm nổi bật các thời kỳ, giai đoạn và dấu mốc phát triển căn bản của tiến trình lịch sử}, nghĩa là tái hiện quá trình lịch sử lãnh đạo và đấu tranh của Đảng. Những kiến thức, tri thức lịch sử Đảng được làm sáng tỏ từ vai trò lãnh đạo, hoạt động thực tiễn của Đảng, vai trò, sức mạnh của nhân dân, của khối đại đoàn kết toàn dân tộc. Hoạt động của Đảng không biệt lập mà thống nhất và khơi dậy mạnh mẽ nguồn sức mạnh từ giai cấp công nhân, nhân dân lao động và toàn dân tộc.
\item \textit{Nhiệm vụ tổng kết lịch sử của Đảng}. Lịch sử Đảng Cộng sản Việt nam không dừng lại mô tả, tái hiện sự kiện và tiến trình lịch sử, mà còn \textit{có nhiệm vụ tổng kết từng chặng đường và suốt tiến trình lịch sử, làm rõ kinh nghiệm, bài học, quy luật và những vấn đề lý luận} của cách mạng Việt Nam. Kinh nghiệm lịch sử gắn liền với những sự kiện hoặc một giai đoạn lịch sử nhất định. Bài học lịch sử khái quát cao hơn gắn liền với một thời kỳ dài, một vấn đề của chiến lược cách mạng hoặc khái quát toàn bộ tiến trình lịch sử của Đảng. Quy luật và những vấn đề lý luận ở tầm tổng kết cao hơn. Hồ Chí Minh nêu rõ:

"Lý luận là đem \textit{thực tế} trong lịch sử, trong kinh nghiệm, trong các cuộc tranh đấu, xem xét, so sánh thật kỹ lưỡng, rõ ràng, làm thành kết luận. Rồi lại đem nó chứng minh với thực tế. Đó là lý luận chân chính" \footfullcite[tr. 273]{HCMtt5}.

"Lý luận do kinh nghiệm cách mạng ở các nước và trong nước ta, do kinh nghiệm từ trước và kinh nghiệm hiện nay gom góp phân tích và kết luận những kinh nghiệm đó thành ra lý luận" \footfullcite[tr. 312]{HCMtt5}.

Hồ Chí Minh nhiều lần đặt ra yêu cầu phải tổng kết, tìm ra quy luật riêng của cách mạng Việt Nam. Qua nhiều lần tổng kết, Đàng Cộng sản Việt Nam khẳng định:

"Con đường đi lên chủ nghĩa xã hội ở nước ta ngày càng được xác định rõ hơn" \footfullcite[tr. 356]{VKDtt55}.

"Con đường đi lên chủ nghĩa xã hội ở nước ta là phù hợp với thực tiễn của Việt Nam và xu thế phát triển của lịch sử" \footfullcite[tr. 66]{DH12}.
\item Một nhiệm vụ quan trọng của lịch sử Đảng là làm rõ vai trò, sức chiến đấu của hệ thống tổ chức đảng từ Trung ương ddeeens cơ sở trong lãnh đạo, tổ chức thực tiễn. Những truyền thống nổi bật của Đảng. Trí tuệ, tính tiên phong, gương mẫu, bản lĩnh của cán bộ, đảng viên. Tấm gương của Chủ tịch Hồ Chí Minh và các nhà lãnh đạo, những chiến sĩ cộng sản tiêu biểu trong các thời kỳ cách mạng. Những giá trị truyền thống, đức hi sinh và tấm gương tiêu biểu luôn luôn là động lực cho sự phát triển và bản chất cách mạng của Đảng. Nghị quyết Trung ương 4 khóa XII (tháng 10/2016) khẳng định: "Chúng ta có quyền tự hào về bản chất tốt đẹp, truyền thống anh hùng và lịch sử vẻ vang của Đảng ta $-$ Đảng của Chủ tịch Hồ Chí Minh vĩ đại, đại biểu của dân tộc Việt Nam anh hùng". Có một nhiệm vụ là hoàn thiện hệ thống tư liệu lịch sử Đảng.
\end{itemize}