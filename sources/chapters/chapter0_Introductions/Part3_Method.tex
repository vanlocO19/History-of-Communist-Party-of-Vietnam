\section{Phương pháp nghiên cứu, học tập môn học Lịch sử Đảng Cộng sản Việt Nam}
\subsection{Quán triệt phương pháp luận sử học}
Phương pháp nghiên cứu, học tập Lịch sử Đảng Cộng sản Việt Nam cần dựa trên phương pháp luận khoa học mác xít, đặc biệt là nắm vững chủ nghĩa duy vật biện chứng và chủ nghĩa duy vật lịch sử để xem xét và nhận thức lịch sử một cách khách quan, trung thực và đúng quy luật. Chú trọng nhận thức lịch sử theo quan điểm khách quan, toàn diện, phát triển và lịch sử cụ thể. Tư duy từ thực tiễn, từ hiện thực lịch sử, coi thực tiễn và kết quả của hoạt động thực tiễn là tiêu chuẩn cho chân lý. Chân lý là cụ thể, cách mạng là sáng tạo. Nhận thức rõ các sự kiện và tiến trình lịch sử trong các quan hệ: nguyên nhân và kết quả, hình thức và nội dung, hiện tượng và bản chất, cái chung và cái riêng, phổ biến và đặc thù.

Chủ nghĩa duy vật lịch sử là kết quả của tư duy biện chứng, khoa học để xem xét, nhận thức lịch sử. Khi nghiên cứu, học tập lịch sử Đảng Cộng sản Việt Nam, cần thiết phải nhận thức, vận dụng chủ nghĩa duy vật lịch sử để nhận thức tiến trình cách mạng do Đảng Cộng sản Việt Nam lãnh đạo. Lý luận về hình thái kinh tế $-$ xã hội; về giai cấp và đấu tranh giai cấp; về dân tộc và đấu tranh dân tộc; về vai trò của quần chúng nhân dân và cá nhân trong lịch sử; về cách mạng xã hội chủ nghĩa và tính tất yếu của cách mạng xã hội chủ nghĩa, sứ mệnh lịch sử của giai cấp vô sản và Đảng Cộng sản.

Cùng với chủ nghĩa Mác $-$ Lênin, tư tưởng Hồ Chí Minh là nền tảng tư tưởng và kim chỉ nam cho hành động của Đảng. Tư tưởng Hồ Chí Minh dẫn dắt sự nghiệp cách mạng của Đảng và dân tộc. Nghiên cứu, nắm vững tư tưởng Hồ Chí Minh có ý nghĩa quan trọng để hiểu rõ lịch sử Đảng. Tư tưởng Hồ Chí Minh và tư duy, phong cách khoa học của Người là cơ sở và định hướng về phương pháp nghiên cứu, học tập Lịch sử Đảng Cộng sản Việt Nam, không ngừng sáng tạo, chống chủ nghĩa giáo điều và chủ quan duy ý chí.

\subsection{Các phương pháp cụ thể}
Khoa học lịch sử và chuyên ngành khoa học Lịch sử Đảng Cộng sản Việt Nam đều sử dụng hai phương pháp cơ bản: phương pháp lịch sử và phương pháp logic, đồng thời chú trọng vận dụng các phương pháp nghiên cứu, học tập các môn khoa học xã hội khác.
\subsubsection{Phương pháp lịch sử}
"\textit{Phương pháp lịch sử} là các con đường, cách thức tìm hiểu và trình bày quá trình phát triển của các sự vật và hiện tượng nói chung, của lịch sử loài người nói riêng với đầy đủ tính cụ thể, sống động, quanh co của chúng".

"Phương pháp lịch sử là nhằm diễn lại tiến trình phát triển của lịch sử với muôn màu muôn vẻ, nhằm thể hiện cái lịch sử với tính cụ thể, hiện thực, tính sinh động của nó. Nó giúp chúng ta nắm vững được cái lịch sử để có cơ sở nắm cái logic được sâu sắc".

Phương pháp lịch sử đi sâu vào tính muôn vẻ của lịch sử để tìm ra cái đặc thù, cái cá biệt trong cái phổ biến. Các hiện tượng lịch sử thường hay tái diễn, nhưng không bao giờ hoàn toàn như cũ; phương pháp lịch sử chú ý tìm ra cái khác trước, cái không lặp để thấy những nét đặc thù của lịch sử. Phương pháp lịch sử để thấy bước quanh co, có khi thụt lùi tạm thời của quá trình lịch sử. Phương pháp lịch sử đòi hỏi nghiên cứu thấu đáo mọi chi tiết lịch sử để hiểu vai trò, tâm lý, tình cảm của quần chúng, hiểu điểm và diện, tổng thể đến cụ thể. Chú trọng về không gian, thời gian, tên đất, tên người để tái hiện lịch sử đúng như nó đã diễn ra. Phương pháp lịch sử không có nghĩa là học thuộc lòng sự kiện, diễn biến lịch sử mà phải hiểu tính chất, bản chất của sự kiện, hiện tượng, do đó không tách rời phương pháp logic.

\subsubsection{Phương pháp logic}
"\textit{Phương pháp logic} là phương pháp nghiên cứu các hiện tượng lịch sử trong hình thức tổng quát, nhằm mục đích vạch ra bản chất, quy luật, khuynh hướng chung trong sự vận động của chúng".

Phương pháp logic đi sâu tìm hiểu cái bản chất, cái lặp lại của các hiện tượng, các sự kiện, phân tích, so sánh, tổng hợp với tư duy khái quát để tìm ra bản chất các sự kiện, hiện tượng. Xác định rõ các bước phát triển tất yếu của quá trình lịch sử để tìm ra quy luật vận động khách quan của lịch sử, phương pháp logic chú trọng những sự kiện, nhân vật, giai đoạn mang tính điển hình. Cần thiết phải nắm vững logic học và rèn luyện tư duy logic, phương pháp logic có ý nghĩa quyết định đến sự nhận thức đúng đắn thế giới khách quan, hiện thực lịch sử, thấy rõ được hướng phát triển của lịch sử. Từ nắm vững quy luật khách quan mà vận dụng vào thực tiễn cách mạng, góp phần chủ động cải tạo, cải biến thế giới và lịch sử.

Chỉ có nắm vững phương pháp lịch sử và phương pháp logic mới có thể hiểu rõ bản chất, nhận thức đúng đắn, giảng dạy và học tập Lịch sử Đảng Cộng sản Việt Nam một cách có hiệu quả, với tư cách một môn khoa học. Phương pháp lịch sử và phương pháp logic có quan hệ mật thiết với nhau và đó là sự thống nhất của phương pháp biện chứng mác xít trong nghiên cứu và nhận thức lịch sử. Các phương pháp đó không tách rời mà luôn luôn gắn với nguyên tắc tính khoa học và tính đảng trong khoa học lịch sử và trong chuyên ngành Lịch sử Đảng Cộng sản Việt Nam.

\subsubsection{Các phương pháp khác}
Cùng với hai phương pháp cơ bản là phương pháp lịch sử, phương pháp logic, nghiên cứu, học tập lịch sử Đảng Cộng sản Việt Nam cần coi trọng \textit{phương pháp tổng kết thực tiễn lịch sử} gắn với nghiên cứu lý luận để làm rõ kinh nghiệm, bài học, quy luật phát triển và những vấn đề về nhận thức lý luận của cách mạng Việt Nam do Đảng lãnh đạo. Chú trọng \textit{phương pháp so sánh}, so sánh giữa các giai đoạn, thời kỳ lịch sử, so sánh các sự kiện, hiện tượng lịch sử, làm rõ các mối quan hệ, so sánh trong nước và thế giới, v. v.

Phương pháp học tập của sinh viên, hết sức coi trọng nghe giảng trên lớp để nắm vững những nội dung cơ bản từng bài giảng của giảng viên, và nội dung tổng thể của môn học. Thực hiện \textit{phương pháp làm việc nhóm}, tiến hành thảo luận, trao đổi các vấn đề do giảng viên đặt ra để hiểu rõ hơn nội dung chủ yếu của môn học. Sử dụng công nghệ thông tin trong giảng dạy và học tập. Tổ chức các cuộc làm việc tại bảo tàng lịch sử quốc gia, bảo tàng địa phương và các di tích lịch sử đặc biệt gắn với sự lãnh đạo của Đảng. Thực hiện kiểm tra, thi cử theo đúng quy chế đào tạo của Bộ Giáo dục và Đào tạo và của các trường đại học.

Nghiên cứu, giảng dạy, học tập lịch sử Đảng Cộng sản Việt Nam cần chú trọng phương pháp vận dụng lý luận vào thực tiễn. Điều đó đòi hỏi nắm vững lý luận cơ bản của chủ nghĩa Mác $-$ Lênin, bao gồm triết học, kinh tế chính trị học, chủ nghĩa xã hội khoa học, nắm vững tư tưởng Hồ Chí Minh, luôn luôn liên hệ lý luận với thực tiễn Việt Nam để nhận thức đúng đắn bản chất của mỗi hiện tượng, sự kiện của lịch sử lãnh đạo, đấu tranh của Đảng.

Tính khoa học là sự phản ánh kết quả nghiên cứu sự vật, hiện tượng, sự kiện lịch sử phải đạt đến chân lý khách quan. Tính khoa học đòi hỏi phản ánh lịch sử khách quan, trung thực với những đánh giá, kết hợp dựa trên luận cứ khoa học, tôn trọng hiện thực lịch sử. Tính khoa học yêu cầu phương pháp nghiên cứu sáng tạo, nghiêm túc và có trách nhiệm. Tính đảng cống sản trong nghiên cứu lịch sử và lịch sử Đảng là đứng trên lập trường chủ nghĩa Mác $-$ Lênin, tư tưởng Hồ Chí Minh để nhận thức lịch sử một cách khoa học, đúng đắn; là sự phản ánh đúng đắn quan điểm, đường lối của Đảng vì lợi ích của giai cấp vô sản, của nhân dân lao động và của dân tộc; là thể hiện tính chiến đấu, biểu dương cái đúng đắn, tốt đẹp, phê phán cái xấu, cái lạc hậu, hư hỏng và những nhận thức lệch lạc, sai trái, phản động của các thế lực thù địch; luôn luôn kế thừa và phát triển sáng tạo. Tính khoa học và tính đảng là thống nhất và đều hướng tới phục vụ nhiệm vụ chính trị của Đảng, của cách mạng vì lý tưởng, mục tiêu độc lập dân tộc và chủ nghĩa xã hội.

Đối với hệ đại học không chuyên về lý luận chính trị, với phân bổ 2 tín chỉ (30 tiết giảng lý thuyết), tập trung nghiên cứu các chương tương ứng với 3 thời kỳ nổi bật của lịch sử Đảng: Đảng Cộng sản Việt Nam ra đời và lãnh đạo đấu tranh giải phóng dân tộc (1930 $-$ 1945); Đảng lãnh đạo hai cuộc kháng chiến giành độc lập hoàn toàn, thống nhất đất nước và xây dựng chủ nghĩa xã hội trên miền Bắc (1945 $-$ 1975); Đảng lãnh đạo cả nước quá độ lên chủ nghĩa xã hội, bảo vệ Tổ quốc và thực hiện công cuộc đổi mới (1975 $-$ nay). Với hệ đại học chuyên lý luận chính trị (3 tín chỉ), giảng viên giargn sâu hơn các thời kỳ lịch sử đồng thời có chương về tổng thể các bài học về sự lãnh đạo của Đảng và có thể giới thiệu có hệ thống, sâu sắc các Cương lĩnh của Đảng. Sinh viên chú trọng hơn tự nghiên cứu.

Với hệ đại học không chuyền về lý luận chính trị, sinh viên cần nắm vững có hệ thống những vấn đề cơ bản của lịch sử Đảng Cộng sản Việt Nam. Hiểu rõ đặc điểm, mâu thuẫn chủ yếu của xã hội thuộc địa, phong kiến ở Việt Nam cuối thế kỷ XIX đầu thế kỷ XX. Sự phát triển tất yếu của đấu tranh giai cấp, đấu tranh dân tộc để giải phóng dân tộc, giải phóng xã hội và giải phóng con người. Các phong trào yêu nước chống thực dân Pháp xâm ucowj \textit{từ lập trường Cần vương đến lập trường tư sản, tiểu tư sản, qua khảo nghiệm lịch sử đều lần lượt thất bại}. Tình hình đất nước \textit{đen tối như không có đường ra}. Trong hoàn cảnh đó, lãnh tụ Nguyễn Ái Quốc $-$ Hồ Chí Minh đã tìm ra con đường đấu tranh đúng đắn để tự giải phóng dân tộc, xã hội, vì cuộc sống của nhân dân. Người đã truyền bá lý luận cách mạng là chủ nghĩa Mác $-$ Lênin vào phong trào công nhân và phong trào yêu nước Việt Nam và phát triển sáng tạo học thuyết lý luận đó vào thực tiễn Việt Nam; chuẩn bị những điều kiện về tư tưởng, lý luận, chính trị, tổ chức, cán bộ để thành lập Đảng Cộng sản Việt Nam. Đảng Cộng sản Việt Nam ra đời mùa Xuân năm 1930 với Cương lĩnh chính trị đúng đắn đã mở ra thời kỳ phát triển mới của cách mạng và dân tộc Việt Nam.

Từ năm 1930 đến năm 1945, Đảng và lãnh tụ Hồ Chí Minh không ngừng bổ sung, phát triển Cương lĩnh, đường lối, giương cao ngọn cờ độc lập dân tộc theo con đường xã hội chủ nghĩa, hoàn thiện đường lối giải phóng dân tộc, lãnh đạo các phong trào cách mạng rộng lớn (1930 $-$ 1931), (1936 $-$ 1939) và cao trào giải phóng dân tộc (1939 $-$ 1945) dẫn đến thắng lợi của cuộc Cách mạng tháng Tám năm 1945. Cần nắm vững tính chất, đặc điểm, ý nghĩa lịch sử của cuộc Cách mạng tháng Tám và bản \textit{Tuyên ngôn độc lập} (02/9/1945) $-$ một thời đại mới được mở ra trong lịch sử dân tộc và cách mạng Việt Nam.

Cần hiểu được hoàn cảnh lịch sử những khó khăn, thách thức của thời kỳ mới, Đảng phải có đường lối, chiến lược và sách lược thích hợp để vừa kháng chiến vừa kiến quốc, xây dựng chính quyền nhà nước và chế độ mới. Đề ra đường lối và lãnh đạo kháng chiến làm thất bại các kế hoạch chiến tranh của thực dân Pháp đưa đến chiến thắng lịch sử Điện Biên Phủ (07/5/1954) và các nước ký kết Hiệp nghị Geneve (21/7/1954). Đế quốc Mỹ thay thế thực dân Pháp, áp đặt chủ nghĩa thực dân mới ở miền Nam Việt Nam và tiến hành cuộc chiến tranh xâm lược Việt Nam từ năm 1954 đến năm 1975 với các chiến lược chiến tranh tàn bạo chống lại dân tộc Việt Nam và phong trào cách mạng giải phóng trên thế giới. Đảng đề ra đường lối, kiên trì lãnh đạo đấu tranh, vượt qua thách thức hiểm nghèo dẫn đến toàn thắng của Chiến dịch Hồ Chí Minh lịch sử giải phóng miền Nam, thống nhất đất nước (30/4/1975).

Cần nhận thức rõ quá trình Đảng lãnh đạo cách mạng xã hội chủ nghĩa ở miền Bắc, với đường lối do Đại hội III của Đảng đề ra (9/1960) và Đảng lãnh đạo đưa cả nước quá độ lên chủ nghĩa xã hội, xây dựng và bảo vệ Tổ quốc sau năm 1975. Hiểu được quá trình đổi mới tư duy lý luận, khảo nghiệm thực tiễn trong những năm 1975 $-$ 1986 để hình thành con đường đổi mới đất nước. Nắm vững đường lối đổi mới được hoạch định tại Đại hội VI (12/1986). Sự phát triển đường lối và tổ chức thực hiện hơn 30 năm qua đưa đất nước vững bước phát triển trên con đường xã hội chủ nghĩa. Lãnh đạo cách mạng dân tộc dân chủ nhân dân, các cuộc kháng chiến, cách mạng xã hội chủ nghĩa phải vượt qua nhiều nguy cơ, khó khăn, thách thức, trong đó có cả khuyết điểm, yếu kém ở mỗi thời kỳ. Đảng đã kiên cường cùng toàn dân vượt qua, quyết tâm sửa chữa khuyết điểm, thực hiện thành công sứ mệnh lịch sử lãnh đạo cách mạng Việt Nam. Nhận thức rõ hơn những truyền thống vẻ vang của Đảng.

Hiểu rõ những vấn đề xây dựng Đảng trong lịch sử để vận dung những kinh nghiệm để làm tốt công tác xây dựng, chỉnh đốn Đảng hiện nay. Chống nguy cơ sai lầm về đường lối, nguy cơ quan liêu, tham nhũng, xa rời quần chúng và những biểu hiện tiêu cực khác. Thực hiện tốt hơn Nghị quyết Trung ương 4 khóa XII (30/10/2016) \textit{Về tăng cường xây dựng, chỉnh đốn Đảng; ngăn chặn, đẩy lùi sự suy thoái về tư tưởng chính trị, đạo đức, lối sống, những biểu hiện "tự diễn biến", "tự chuyển hóa" trong nội bộ}.

Lịch sử Đảng Cộng sản Việt Nam, như Chủ tịch Hồ Chí Minh khẳng định, \textit{là cả một pho lịch sử bằng vàng}. Đó chính là tính khoa học, giá trị thực tiễn sâu sắc trong Cương lĩnh, đường lối của Đảng; là sự lãnh đạo đúng đắn, đáp ứng kịp thời những yêu cầu, nhiệm vụ do lịch sử đặt ra; những kinh nghiệm, bài học quý báu có tính quy luật, lý luận của cách mạng Việt Nam và những truyền thống vẻ vang của Đảng. Nghiên cứu, học tập lịch sử Đảng không chỉ nắm vững những sự kiện, cột mốc lịch sử mà cần thấu hiểu những vấn đề phong phú đó trong quá trình lãnh đạo và đấu tranh, để vận dụng, phát triển trong thời kỳ đổi mới toàn diện, đẩy mạnh công nghiệp hóa, hiện đại hóa đất nước và hội nhập quốc tế.

Mục tiêu của nghiên cứu, học tập môn học lịch sử Đảng Cộng sản Việt Nam là nâng cao nhận thức, hiểu biết về Đảng Cộng sản Việt Nam $-$ đội tiền phong lãnh đạo cách mạng Việt Nam đưa đến những thắng lợi, thành tựu có ý nghĩa lịch sử to lớn trong sự phát triển của lịch sử dân tộc. Qua học tập, nghiên cứu lịch sử Đảng để giáo dục lý tưởng, truyền thống đấu tranh cách mạng của Đảng và dân tộc, củng cố, bồi đắp niềm tin đối với sự lãnh đạo của Đảng, tự hào về Đảng và thế hệ trẻ gia nhập Đảng, tham gia xây dựng Đảng ngày càng vững mạnh, tiếp tục thực hiện sứ mệnh vẻ vang của Đảng lãnh đạo bảo vệ vững chắc Tổ quốc và xây dựng thành công chủ nghĩa xã hội ở Việt Nam.