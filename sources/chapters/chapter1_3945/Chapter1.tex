\chapter{Đảng Cộng sản Việt Nam ra đời và lãnh đạo đấu tranh giành chính quyền (1939 $-$ 1945)}
\section*{Mục tiêu}
\subsection*{Về kiến thức}
Cung cấp cho sinh viên những tri thức có tính hệ thống quá trình ra đời của Đảng Cộng sản Việt Nam (1920 $-$ 1930), nội dung cơ bản, giá trị lịch sử của Cương lĩnh chính trị đầu tiên của Đảng và quá trình Đảng lãnh đạo cuộc đấu tranh giải phóng dân tộc, giành chính quyền (1930 $-$ 1945).

\subsection*{Về tư tưởng}
Cung cấp cơ sở lịch sử, góp phần củng cố niềm tin của thế hệ trẻ vào con đường cách mạng giải phóng dân tộc và phát triển đất nước $-$ sự lựa chọn đúng đắn, tất yếu, khách quan của lãnh tụ Nguyễn Ái Quốc và Đảng Cộng sản Việt Nam thời kỳ đầu dựng Đảng.

\subsection*{Về kỹ năng}
Từ việc nhận thức lịch sử thời kỳ đầu dựng Đảng, góp phần trang bị cho sinh viên phương pháp nhận thức biện chứng, khách quan về quá trình Đảng ra đời và vai trò lãnh đạo của Đảng trong cuộc đấu tranh giải phóng dân tộc, xác lập chính quyền cách mạng.
\section{Đảng Cộng sản Việt Nam ra đời và Cương lĩnh chính trị đầu tiên của Đảng}
\subsection{Bối cảnh lịch sử}
\subsubsection{Tình hình thế giới cuối thế kỷ XIX, đầu thế kỷ XX}
Từ nửa sau thế kỷ XIX, các nước tư bản Âu $-$ Mỹ có những chuyển biến mạnh mẽ trong đời sống kinh tế $-$ xã hội. Chủ nghĩa tư bản phương Tây chuyển nhanh từ giai đoạn tự do cạnh tranh sang giai đoạn độc quyền (giai đoạn đế quốc chủ nghĩa), đẩy mạnh quá trình xâm chiếm và nô dịch các nước nhỏ, yếu ở châu Á, châu Phi và khu vực Mỹ $-$ Latinh, biến các quốc gia này thành thuộc đia của các nước đế quốc. Trước bối cảnh đó, nhân dân các dân tộc bị áp bức đã đứng lên đấu tranh tự giải phóng khỏi ách thực dân, đế quốc, tạo thành phong trào giải phóng dân tộc mạnh mẽ, rộng khắp, nhất là ở châu Á. Cùng với phong trào đấu tranh của giai cấp vô sản chống lại giai cấp tư sản ở các nước tư bản chủ nghĩa, phong trào giải phóng dân tộc ở các nước thuộc địa trở thành một bộ phận quan trọng trong cuộc đấu tranh chung chống tư bản, thực dân. Phong trào giải phóng dân tộc ở các nước châu Á đầu thế kỷ XX phát triển rộng khắp, tác động mạnh mẽ đến phong trào yêu nước Việt Nam.

Trong bối cảnh đó, thắng lợi của Cách mạng tháng Mười Nga năm 1917 đã làm biến đổi sâu sắc tình hình thế giới. Thắng lợi của Cách mạng tháng Mười Nga không chỉ có ý nghĩa to lớn đối với cuộc đấu tranh của giai cấp vô sản đối với các nước tư bản, mà còn có tác động sâu sắc đến phong trào giải phóng dân tộc ở các nước thuộc địa. Tháng 3/1919, Quốc tế Cộng sản, do V. I. Lênin đứng đầu, được thành lập, trở thành bộ tham mưu chiến đấu, tổ chức lãnh đạo phong trào cách mạng vô sản thế giới. Quốc tế Cộng sản không những vạch đường hướng chiến lược cho cách mạng vô sẩn mà cả đối với các vấn đề dân tộc và thuộc địa, giúp đỡ, chỉ đạo phong trào giải phóng dân tộc. Cùng với việc nghiên cứu và hoàn thiện chiến lược và sách lược về vấn đề dân tộc và thuộc địa, Quốc tế Cộng sản đã tiến hành hoạt động truyền bá tư tưởng cách mạng vô sản và thúc đẩy phong trào đấu tranh ở khu vực này đi theo khuynh hướng vô sản. Đại hội II của Quốc tế Cộng sản (1920) đã thông qua luận cương về dân tộc và thuộc địa do V. I. Lênin khởi xướng. Cách mạng tháng Mười và những hoạt động cách mạng của Quốc tế Cộng sản đã ảnh hưởng mạnh mẽ và thức tỉnh phong trào giải phóng dân tộc ở các nước thuộc địa, trong đó có Việt Nam và Đông Dương.

\subsubsection{Tình hình Việt Nam và các phong trào yêu nước trước khi có Đảng}
Là quốc gia Đông Nam Á nằm ở vị trí địa chính trị quan trọng của châu Á, Việt Nam trở thành đối tượng nằm trong mưu đồ xâm lược của thực dân Pháp trong cuộc chạy đua với nhiều đế quốc khác. Sau một quá trình điều tra thám sát lâu dài, thâm nhập kiên trì của các giáo sĩ và thương nhân Pháp, ngày 01/9/1858, thực dân Pháp nổ súng xâm lược Việt Nam tại Đà Nẵng và từ đó từng bước thôn tính Việt Nam. Đó là thời điểm chế độ phong kiến Việt Nam (dưới triều đại phong kiến nhà Nguyễn) đã lâm vào giai đoạn khủng hoảng trầm trọng. Trước hành động xâm lược của Pháp, triều đình nhà Nguyễn từng bước thỏa hiệp (HIệp ước 1862, 1874, 1883) và đến ngày 06/6/1884 với Hiệp ước Patenotre đã đầu hàng hoàn toàn thực dân Pháp, Việt Nam trở thành "một xứ thuộc địa, dân ta là vong quốc nô, Tổ quốc ta bị giày xéo dưới gót sắt của kẻ thù hung ác" \footfullcite[tr. 401]{HCMtt12}.

Tuy triều đình nhà Nguyễn đã đầu hàng thực dân Pháp, nhưng nhân dân Việt Nam vẫn không chịu khuất phục, thực dân Pháp dùng vũ lực để bình định, đàn áp sự nổi dậy của nhân dân. Đồng thời với việc dùng vũ lực đàn áp đẫm máu đối với các phong trào yêu nước của nhân dân Việt Nam, thực dân Pháp tiến hành xây dựng hệ thống chính quyền thuộc địa, bên cạnh đó vẫn duy trì chính quyền phong kiến bản xứ làm tay sai. Pháp thực hiện chính sách "chia để trị" nhằm phá vỡ khối đoàn kết cộng đồng quốc gia dân tộc: chia ba kỳ (Bắc Kỳ, Trung Kỳ, Nam Kỳ) với các chế độ chính trị khác nhau nằm trong \textit{Liên bang Đông Dương thuộc Pháp} (Union Indochinoise) \footnote{Bao gồm: Bắc Kỳ, Trung Kỳ, Nam Kỳ, Cao Miên, Ai Lao} được thành lập ngày 17/10/1887 theo sắc lệnh của Tổng thống Pháp.

Từ năm 1897, thực dân Pháp bắt đầu tiến hành các cuộc khai thác thuộc địa lớn: Cuộc khai thác thuộc địa lần thứ nhất (1897 $-$ 1914) do Toàn quyền Đông Dương Paul Doumer thực hiện và Cuộc khai thác thuộc địa lần thứ hai (1919 $-$ 1929). Mưu đồ của thực dân Pháp nhằm biến Việt Nam nói riêng và Đông Dương nói chung thành thị trường tiêu thụ hàng hóa của "chính quốc", đồng thời ra sức vơ vét tài nguyên, bóc lột sức lao động rẻ mạt của người bản xứ, cùng nhiều hình thức thuế khóa nặng nề.

Chế độ cai trị, bóc lột hà khắc của thực dân Pháp đối với nhân dân Việt Nam là "chế độ độc tài chuyên chế nhất, nó vô cùng khả ố và khủng khiếp hơn cả chế độ chuyên chế của nhà nước quân chủ châu Á thời xưa" \footnote{Bài đăng của Phan Văn Trường trên tờ \textit{La Cloché Félée} (Tiếng chuông rè), số 36, ngày 21/1/1926}. Năm 1862, Pháp đã lập nhà tù ở Côn Đảo để giam cầm những người Việt Nam yêu nước chống Pháp.

Về văn hóa $-$ xã hội, thực dân Pháp thực hiện chính sách "ngu dân" để dễ cai trị, lập nhà tù nhiều hơn trường học, đồng thời du nhập những giá trị phản văn hóa, duy trì tệ nạn xã hội vốn có của chế độ phong kiến và tạo nên nhiều tệ nạn xã hội mới, dùng rượu cồn và thuốc phiện để đầu độc các thế hệ người Việt Nam, ra sức tuyên truyền tư tưởng "khai hóa văn minh" của nước Đại Pháp ...

Chế độ áp bức về chính trị, bóc lột về kinh tế, nô dịch về văn hóa của thực dân Pháp đã làm biến đổi tình hình chính trị, kinh tế, xã hội Việt Nam. Các giai cấp cũ phân hóa, giai cấp, tầng lớp mới xuất hiện với địa vị kinh tế khác nhau và do đó cũng có thái độ chính trị khác nhau đối với vận mệnh dân tộc.

Dưới chế độ phong kiến, giai cấp địa chủ và nông dân là hai giai cấp cơ bản trong xã hội, khi Việt Nam trở thành thuộc địa của Pháp, giai cấp địa chủ bị phân hóa.

Một bộ phận địa chủ câu kết với thực dân Pháp và làm tay sai đắc lực cho Pháp trong việc ra sức đàn áp phong trào yêu nước và bóc lột nông dân; một bộ phận khác nêu cao tinh thần dân tộc khởi xướng và lãnh đạo các phong trào chống Pháp và bảo vệ chế độ phong kiến, tiêu biểu là phong trào Cần vương; một số trở thành lãnh đạo phong trào nông dân chống thực dân Pháp và phong kiến phản động; một bộ phận nhỏ chuyển sang kinh doanh theo lối tư bản.

Giai cấp nông dân chiếm số lượng đông đảo nhất (khoảng hơn $90\%$ dân số), đồng thời là một giai cấp bị phong kiến, thực dân bóc lột nặng nề nhất. Do vậy, ngoài mâu thuẫn giai cấp vốn có với giai cấp địa chủ, từ khi thực dân Pháp xâm lược, giai cấp nông dân còn có mâu thuẫn sâu sắc với thực dân xâm lược. "Tinh thần cách mạng của nông dân không chỉ gắn liền với ruộng đất, với đời sống hằng ngày của họ, mà còn gắn bó một cách sâu sắc với tình cảm quê hương, đất nước, với nền văn hóa hàng nghìn năm của dân tộc" \footfullcite[tr. 119]{LDGCCN}. Đây là lực lượng hùng hậu, có tinh thần đấu tranh kiên cường bất khuất cho nền độc lập tự do của dân tộc và khao khát giành lại ruộng đất cho dân cày, khi có lực lượng tiên phong lãnh đạo, giai cấp nông dân sẵn sàng vùng dậy làm cách mạng lật đổ thực dân phong kiến.

Giai cấp công nhân Việt Nam được hình thành gắn với các cuộc khai thác thuộc địa, với việc thực dân Pháp thiết lập các nhà máy, xí nghiệp, công xưởng, khu đồn điền, ... Ngoài những đặc điểm của giai cấp công nhân quốc tế, giai cấp công nhân Việt Nam có những đặc điểm riêng vì ra đời trong hoàn cảnh một nước thuộc địa nửa phong kiến, chủ yếu xuất thân từ nông dân, cơ cấu chủ yếu là công nhân khai thác mỏ, đồn điền, lực lượng còn nhỏ bé \footnote{Số lượng công nhân đến trước chiến tranh thế giới thứ nhất (1913) có khoảng 10 vạn người; đến cuối năm 1929, số công nhân Việt Nam là hơn 22 vạn người, chiếm trên $1.2\%$ dân số.}, nhưng sớm vươn lên tiếp nhận tư tưởng tiên tiến của thời đại, nhanh chóng phát triển từ "tự phát" đến "tự giác", thể hiện là giai cấp có năng lực lãnh đạo cách mạng.

Giai cấp tư sản Việt Nam xuất hiện muộn hơn giai cấp công nhân. Một bộ phận gắn liền lợi ích với tư bản Pháp, tham gia vào đời sống chính trị, kinh tế của chính quyền thực dân Pháp, trở thành tầng lớp tư sản mại bản. Một bộ phận là giai cấp tư sản dân tộc, họ bị thực dân Pháp chèn ép, kìm hãm, bị lệ thuộc, yếu ớt về kinh tế. Vì vậy, phần lớn tư sản dân tộc Việt Nam có tinh thần dân tộc, yêu nước nhưng không có khả năng tập hợp các giai tầng để tiến hành cách mạng.

Tầng lớp tiểu tư sản (tiểu thương, tiểu chủ, sinh viên, ...) bị đế quốc, tư bản chèn ép, khinh miệt, do đó có tinh thần dân tộc, yêu nước và rất nhạy cảm về chính trị và thời cuộc. Tuy nhiên, do địa vị kinh tế bấp bênh, thái độ hay dao động, thiếu kiên định, do đó tầng lớp tiểu tư sản không thể lãnh đạo cách mạng.

Các sĩ phu phong kiến cũng có sự phân hóa. Một bộ phận hướng sang tư tưởng dân chủ tư sản hoặc tư tưởng vô sản. Một số người khởi xướng các phong trào yêu nước có ảnh hưởng lớn.

Cuối thế kỷ XĨ đầu thế kỷ XX, Việt Nam đã có sự biến đổi rất quan trọng cả về chính trị, kinh tế, xã hội. Chính sách cai trị và khai thác bóc lột của thực dân Pháp đã làm phân hóa những giai cấp vốn là của chế độ phong kiến (địa chủ, nông dân), đồng thời tạo nên những giai cấp, tầng lớp mới (công nhân, tư sản dân tộc, tiểu tư sản) với thái độ chính trị khác nhau. Những mâu thuẫn mới trong xã hội Việt Nam xuất hiện. Trong đó, mâu thuẫn giữa toàn thể dân tộc Việt Nam với thực dân Pháp và phong kiến phản động trở thành mâu thuẫn chủ yếu nhất và ngày càng gay găt.

Trong bối cảnh đó, những luồng tư tưởng ở bên ngoài: tư tưởng Cách mạng tư sản Pháp 1789, phong trào Duy tân Nhật Bản năm 1868, cuộc vận động Duy tân tại Trung Quốc năm 1898, Cách mạng Tân Hợi của Trung Quốc năm 1911, ... đặc biệt là Cách mạng tháng Mười Nga năm 1917 đã tác động mạnh mẽ, làm chuyển biến phong trào yêu nước những năm cuối thế kỷ XIX, đầu thế kỷ XX. Năm 1919, trên chiến hạm của Pháp ở Hắc Hải (Biển Đen), Tôn Đức Thắng tham gia đấu tranh chống việc can thiệp vào nước Nga Xô viết. Năm 1923, luật sư Phan Văn Trường từ Pháp về nước và ông công bố tác phẩm của C. Mác và Ph. Ăngghen: \textit{Tuyên ngôn của Đảng Cộng sản} (The Manifesto of the Communist Party) trên báo \textit{La Cloché Félée}, từ số ra ngày 29/3 đến 20/4/1926, tại Sài Gòn, góp phần tuyên truyền tư tưởng vô sản ở Việt Nam.

\subsubsection{Các phong trào yêu nước của nhân dân Việt Nam trước khi có Đảng}
Ngay từ khi Pháp xâm lược, các phong trào yêu nước chống thực dân Pháp với tinh thần quật cường, bảo vệ nền độc lập dân tộc của nhân dân Việt Nam đã diễn ra liên tục, rộng khắp.

Đến năm 1884, mặc dù triều đình phong kiến nhà Nguyễn đã đầu hàng, nhưng một bộ phận phong kiến yêu nước đã cùng với nhân dân vẫn tiếp tục đấu tranh vũ trang chống Pháp.

Đó là phong trào Cần vương do vua Hàm Nghi và Tôn Thất Thuyết khởi xướng (1885 $-$ 1896). Hưởng ứng lời kêu gọi Cần vương cứu nước, các cuộc khởi nghĩa Ba ĐÌnh (Thanh Hóa), Bãi Sậy (Hưng Yên), Hương Khê (Hà Tĩnh), ... diễn ra sôi nổi và thể hiện tinh thần quật cường chống ngoại xâm của các tầng lớp nhân dân. NHưng ngọn cờ phong kiến lúc đó không còn là ngọn cờ tiêu biểu để tập hợp một cách rộng rãi, toàn thể các tầng lớp nhân dân, không có khả năng liên kết các trung tâm kháng Pháp trên toàn quốc nữa. Cuộc khởi nghĩa của Phan Đình Phùng thất bại (1896) cũng là mốc chấm dứt vai trò lãnh đạo của giai cấp phong kiến đối với phong trào yêu nước chống thực dân Pháp ở Việt Nam. Đầu thế kỷ XX, vua Thành Thái và vua Duy Tân tiếp tục đấu tranh chống Pháp, trong đó có khởi nghĩa của vua Duy Tân (5/1916).

Vào những năm cuối thế kỷ XIX đầu thế kỷ XX, ở vùng miền núi và trung du phía Bắc, \textit{phong trào nông dân} Yên Thế (Bắc Giang) dưới sự lãnh đạo của vị thủ lĩnh nông dân Hoàng Hoa Thám, nghĩa quân đã xây dựng lực lượng chiến đấu, lập căn cứ và đấu tranh kiên cường chống thực dân Pháp. Nhưng phong trào của Hoàng Hoa Thám vẫn mang nặng "cốt cách phong kiến", không có khả năng mở rộng hợp tác và thống nhất tạo thành một cuộc cách mạng giải phóng dân tộc, cuối cùng cũng bị thực dân Pháp đàn áp.

Từ những năm đầu thế kỷ XX, phong trào yêu nước Việt Nam chịu ảnh hưởng, tác động của \textit{trào lưu dân chủ tư sản}, tiêu biểu là xu hướng bạo động của Phan Bội Châu, xu hướng cải cách của Phan Châu Trinh và sau đó là phong trào tiểu tư sản trí thức của tổ chức Việt Nam Quốc dân Đảng (12/1927 $-$ 2/1930) đã tiếp tục diễn ra rộng khắp các tỉnh Bắc Kỳ, nhưng tất cả đều không thành công.

Xu hướng bạo động do Phan Bội Châu tổ chức, lãnh đạo: với chủ trương tập hợp lực lượng với phương pháp bạo động chống Pháp, xây dựng chế độ chính trị như ở Nhật Bản, phong trào theo xu hướng này tổ chức đưa thanh niên yêu nước Việt Nam sang Nhật Bản học tập (gọi là phong trào \textit{Đông du}). Đến năm 1908, Chính phủ Nhật Bản câu kết với thực dân Pháp trục xuất lưu học sinh Việt Nam và những người đứng đầu. Sau khi phong trào Đông du thất bại, năm 1912, Phan Bội Châu thành lập tổ chức \textit{Việt Nam Quang phục Hội} với tôn chỉ là vũ trang đánh đuổi thực dân Pháp, khôi phục VIệt Nam, thành lập nước cộng hòa dân quốc Việt Nam. Nhưng chương trình, kế hoạch hoạt động của Hội lại thiếu rõ ràng. Cuối năm 1913, Phan Bội Châu bị thực dân Pháp bắt giam tại Trung Quốc cho tới đầu năm 1917 và sau này bị quản chế tại Huế cho đến khi ông mất (1940). Ảnh hưởng xu hướng bạo động của tổ chức \textit{Việt Nam Quang phục Hội} đối với phong trào yêu nước Việt Nam đến đây chấm dứt.

Xu hướng cải cách của Phan Châu Trinh: Phan Châu Trinh và những người cùng chí hướng muốn giành độc lập cho dân tộc nhưng không đi theo con đường bạo động như Phan Bội Châu, mà chủ trưởng cải cách đất nước. Phan Châu Trinh cho rằng "bất bạo động, bạo động tắc tử"; phải "khai dân trí, chấn dân khí, hậu dân sinh", phải bãi bỏ chế độ quân chủ, thực hiện dân quyền, khai thông dân trí, mở mang thực nghiệp. Để thực hiện được chủ trương ấy, Phan Châu Trinh đã đề nghị Nhà nước "bảo hộ" Pháp tiến hành cải cách. Đó chính là sự hạn chế trong xu hướng cải cách để cứu nước, vì Phan Châu Trinh đã "đặt vào lòng độ lượng của Pháp cái hi vọng cải tử hoàn sinh cho nước Nam, ... Cụ không rõ bản chất của đế quốc thực dân" \footfullcite[tr. 442]{TVG}. Do vậy, khi phong trào Duy tân lan rộng khắp cả Trung Kỳ và Nam Kỳ, đỉnh cao là vụ chống thuế ở Trung Kỳ (1908), thực dân Pháp đã đàn áp dã man, giết hại nhiều sĩ phu và nhân dân tham gia biểu tình. Nhiều sĩ phu bị bắt, bị đày đi Côn Đảo, trong đó có Phan Châu Trinh, Huỳnh Thúc Kháng, Đặng Nguyên Cẩn, ... Phong trào chống thuế ở Trung Kỳ bị thực dân Pháp dập tắt, cùng với sự kiện tháng 12/1907, thực dân Pháp ra lệnh đóng cửa Trường Đông Kinh Nghĩa Thục \footnote{Trường Đông Kinh Nghĩa Thục do Lương Văn Can, Nguyễn Quyền, Hoàng Tăng Bí, ... thành lập ở Hà Nội, nhằm truyền bá tư tưởng dân chủ, tự do tư sản, nâng cao lòng tự tôn dân tộc cho thanh niên Việt Nam.} phản ánh sự kết thúc xu hướng cải cách trong phong trào cứu nước của Việt Nam.

Phong trào của tổ chức Việt Nam Quốc dân Đảng: khi thực dân Pháp đẩy mạnh khai thác thuộc địa lần thứ hai, mâu thuẫn giữa toàn thể dân tộc Việt Nam với thực dân Pháp càng trở nên gay gắt, các giai cấp, tầng lớp mới trong xã hội Việt Nam đều bước lên vũ đài chính trị. Trong đó, hoạt động có ảnh hưởng rộng và thu hút nhiều học sinh, sinh viên yêu nước ở Bắc Kỳ là tổ chức \textit{Việt Nam Quốc dân Đảng} do Nguyễn Thái Học lãnh đạo. Trên cơ sở các tổ chức yêu nước của tiểu tư sản trí thức, Việt Nam Quốc dân Đảng được chính thức thành lập tháng 12/1927 ở Bắc Kỳ.

Mục đích của Việt Nam Quốc dân Đảng là đánh đuổi thực dân Pháp xâm lược, giành độc lập dân tộc, xây dựng chế độ cộng hòa tư sản, với phương pháp đấu tranh vũ trang nhưng theo lối manh động, ám sát cá nhân và lực lượng chủ yếu là binh lính, sinh viên. Cuộc khởi nghĩa nổ ra ở một số tỉnh, chủ yếu và mạnh nhất là ở Yên Bái (2/1930), tuy oanh liệt nhưng nhanh chóng bị thất bại. Sự thất bại của khởi nghĩa Yên Bái của tổ chức Việt Nam Quốc dân Đảng đã thể hiện là "... một cuộc bạo động bất đắc dĩ, một cuộc bạo động non, để rồi chết luôn không bao giờ ngóc đầu lên nooiro. Khẩu hiệu "không thành công thì thành nhân" biểu lộ tính chất hấp tấp tiểu tư sản, tính chất hăng hái nhất thời và đồng thời cũng biểu lộ tính chất không vững chắc, non yếu của phong trào tư sản" \footfullcite[tr. 41]{LDDD}.

Vào những năm cuối thế kỷ XIX đầu thế kỷ XX, tiếp tục truyền thống yêu nước, bất khuất kiên cường chống ngoại xâm, các phong trào yêu nước theo ngọn cờ phong kiến, ngọn cờ dân chủ tư sản của nhân dân Việt Nam đã diễn ra quyết liệt, liên tục và rộng khắp. Dù với nhiều cách thức tiến hành khác nhau, song đều hướng tới mục tiêu giành độc lập cho dân tộc. Tuy nhiên, "các phong trào cứu nước từ lập trường Cần vương đến lập trường tư sản, tiểu tư sản qua khảo nghiệm lịch sử đều lần lượt thất bại" \footfullcite[tr. 14]{VKDtt51}. Nguyên nhân thất bại của các phong trào đó là do thiếu đường lối chính trị đúng đắn để giải quyết triệt để những mâu thuẫn cơ bản, chủ yếu của xã hội, chưa có một tổ chức vững mạnh để tập hợp, giác ngộ và lãnh đạo toàn dân tộc, chưa xác định được phương pháp đấu tranh thích hợp để đánh đổ kẻ thù.

Các phong trào yêu nước ở Việt Nam cho đến những năm 20 của thế kỷ XX đều thất bại, nhưng đã góp phần cổ vũ mạnh mẽ tinh thần yêu nước của nhân dân, bồi đắp thêm cho chủ nghĩa yêu nước Việt Nam, đặc biệt góp phần thúc đẩy những nhà yêu nước, nhất là lớp thanh niên tri thức tiên tiến chọn lựa một con đường mới, một giải pháp cứu nước, giải phóng dân tộc theo xu thế của thời đại. Nhiệm vụ lịch sử cấp thiết đặt ra cho thế hệ yêu nước đương thời là cần phải có một tổ chức cách mạng tiên phong, có đường lối cứu nước đúng đắn để giải phóng dân tộc.
\subsection{Nguyễn Ái Quốc chuẩn bị các điều kiện để thành lập Đảng}
\subsubsection{Khái quát quá trình tìm đường cứu nước}
Trước yêu cầu cấp thiết giải phóng dân tộc của nhân dân Việt Nam, với nhiệt huyết cứu nước, với nhãn quan chính trị sắc bén, vượt lên trên hạn chế của các bậc yêu nước đương thời, năm 1911, Nguyễn Tất Thành quyết định ra đi tìm đường cứu nước, giải phóng dân tộc. Qua trải nghiệm thực tế ở nhiều nước, Người đã nhận thức được một cách rạch ròi rằng: "dù màu da có khác nhau, trên đời này chỉ có \textit{hai giống người: giống người bóc lột và giống người bị bóc lột}", từ đó xác định rõ kẻ thù và lực lượng đồng minh của nhân dân các dân tộc bị áp bức.

Năm 1917, thắng lợi của Cách mạng tháng Mười Nga đã tác động mạnh mẽ tới nhận thức của Nguyễn Tất Thành $-$ đây là cuộc "cách mạng đến nơi". Người từ nước Anh trở lại nước Pháp và tham gia các hoạt động chính trị hướng về tìm hiểu con đường Cách mạng tháng Mười Nga, về V. I. Lênin.

Đầu năm 1919, Nguyễn Tất Thành tham gia Đảng Xã hội Pháp, một chính đảng tiến bộ nhất luc đó ở Pháp. Tháng 6/1919, tại Hội nghị của các nước thắng trận trong Chiến tranh thế giới thứ nhất họp ở Versailles (Pháp), Tổng thống Mỹ Wooderow Wilson tuyên bố bảo đảm về quyền dân tộc tự quyết cho các nước thuộc địa. Nguyễn Tất Thành lấy tên là Nguyễn Ái Quốc thay mặt \textit{Hội những người An Nam yêu nước} ở Pháp gửi tới Hội nghị bản Yêu sách của nhân dân An Nam (gồm tám điểm đòi quyền tự do cho nhân dân Việt Nam) ngày 18/6/1919. Nhóm người Việt Nam tiêu biểu cho tinh thần yêu nước ở Pháp, gồm: Phan Châu Trinh, Nguyễn An Ninh, Phan Văn Trường, Nguyễn Thế Truyền và Nguyễn Ái Quốc. Những yêu sách đó dù không được Hội nghị đáp ứng, nhưng sự kiện này đã tạo nên tiếng vang lớn trong dư luận quốc tế và Nguyễn Ái Quốc càng hiểu rõ hơn bản chất của đế quốc, thực dân.

Tháng 7/1920, Người đọc bản \textit{Sơ thảo lần thứ nhất những luận cương về vấn đề dân tộc và vấn đề thuộc địa} của V. I. Lênin đăng trên báo \textit{L'Humanite} (Nhân đạo), số ra ngày 16 và 17/7/1920. Những luận điểm của V. I. Lênin về vấn đề dân tộc và thuộc địa đã giải đáp những vấn đề cơ bản và chỉ dẫn hướng phát triển của sự nghiệp cứu nước, giải phóng dân tộc. Lý luận của V. I. Lênin và lập trường đúng đắn của Quốc tế Cộng sản về cách mạng giải phóng các dân tộc thuộc địa là cơ sở để Nguyễn Ái Quốc xác định thái độ ủng hộ việc gia nhập Quốc tế Cộng sản tại Đại hội lần thứ XVIII của Đảng Xã hội Pháp (12/1920) tại thành phố Tour. Tại Đại hội này, Nguyễn Ái Quốc đã bỏ phiếu tán thành Quốc tế III (Quốc tế Cộng sản do V. I. Lênin thành lập).

Ngay sau đó, Nguyễn Ái Quốc cùng với những người vừa bỏ phiếu tán thành Quóc tế Cộng sản đã tuyên bố thành lập \textit{Phân bộ Pháp của Quốc tế Cộng sản} $-$ tức là Đảng Cộng sản Pháp. Với sự kiện này, Nguyễn Ái Quốc trở thành một trong những sáng lập viên của Đảng Cộng sản Pháp và là người cộng sản đầu tiên của Việt Nam, đánh dấu bước chuyển biến quyết định trong tư tưởng và lập trường chính trị của Nguyễn Ái Quốc. Trong những năm 1919 $-$ 1921, Bộ trưởng Bộ Thuộc địa Pháp Albert Sarraut nhiều lần gặp Nguyễn Ái Quốc mua chuộc và đe dọa. Ngày 30/6/1923, Nguyễn Ái Quốc tới Liên Xô và làm việc tại Quốc tế Cộng sản ở Moscow, tham gia nhiều hoạt động, đặc biệt là dự và đọc tham luận tại Đại hội V Quốc tế Cộng sản (17/6 $-$ 07/8/1924), làm việc trực tiếp ở Ban Phương Đông của Quốc tế Cộng sản.

Sau khi xác định được con đường cách mạng đúng đắn, Nguyễn Ái Quốc tiếp tục khảo sát, tìm hiểu để hoàn thiện nhận thức về đường lối cách mạng vô sản, đồng thời tích cực truyền bá chủ nghĩa Mác $-$ Lênin về Việt Nam.

\subsubsection{Chuẩn bị về tư tưởng, chính trị và tổ chức cho sự ra đời của Đảng}
\paragraph{Về tư tưởng}
Từ giữa năm 1921 tại Pháp, cùng một số nhà cách mạng của các nước thuộc địa khác, Nguyễn Ái Quốc tham gia thành lập Hội liên hiệp thuộc địa, sau đó sáng lập ra tờ báo \textit{Le Paria} (Người cùng khổ). Người viết nhiều bài trên các báo \textit{Nhân đạo, Đời sống công nhân, Tạp chí Cộng sản, Tập san Thư tín quốc tế}, ...

Năm 1922, \textit{Ban Nghiên cứu thuộc địa} của Đảng Cộng sản Pháp được thành lập, Nguyễn Ái Quốc được cư làm Trưởng Tiểu ban Nghiên cứu về Đông Dương. Vừa nghiên cứu lý luận, vừa tham gia hoạt động thực tiễn trong phong trào cộng sản và công nhân quốc tế, dưới nhiều phương thức phong phú, Nguyễn Ái Quốc tích cực tố cáo, lên án bản chất áp bức, bóc lột, nô dịch của chủ nghĩa thực dân đối với nhân dân các nước thuộc địa và kêu gọi, thức tỉnh nhân dân bị áp bức đấu tranh giải phóng. Người chỉ rõ bản chất của chủ nghĩa thực dân, xác định chủ nghĩa thực dân là kẻ thù chung của các dân tộc thuộc địa, của giai cấp công nhân và nhân dân lao động trên thế giới. Đồng thời, Người tiến hành tuyên truyền tư tưởng về con đường cách mạng vô sản, con đường cách mạng theo \textit{lý luận Mác $-$ Lênin}, xây dựng mối quan hệ gắn bó giữa những người cộng sản và nhân dân lao động Pháp với các nước thuộc địa và phụ thuộc.

Năm 1927, Nguyễn Ái Quốc khẳng định: "Đảng muốn vững phải có chủ nghĩa làm cốt, trong đảng ai cũng phải hiểu, ai cũng phải theo chủ nghĩa ấy" \footfullcite[tr. 289]{HCMtt2}. Đảng mà không có chủ nghĩa cũng giống như người không có trí khôn, tàu không có bàn chỉ nam. Phải truyền bá tư tưởng vô sản, lý luận Mác $-$ Lênin vào phong trào công nhân và phong trào yêu nước Việt Nam.
\subsection{Thành lập Đảng Cộng sản Việt Nam và Cương lĩnh chính trị đầu tiên của Đảng}
\subsubsection{Các tổ chức cộng sản ra đời}
Với sự nỗ lực cố gắng truyền bá chủ nghĩa Mác $-$ Lênin vào phong trào công nhân và phong trào yêu nước Việt Nam của Nguyễn Ái Quốc và những hoạt động tích cực của các cấp bộ trong tổ chức Hội Việt Nam Cách mạng Thanh niên trên cả nước đã có tác dụng thúc đẩy phong trào yêu nước Việt Nam theo khuynh hướng cách mạng vô sản, nâng cao ý thức giác ngộ và lập trường cách mạng của giai cấp công nhân. Những cuộc đấu tranh của thợ thuyền khắp ba kỳ với nhịp độ, quy mô ngày càng lớn, nội dung chính trị ngày càng sâu sắc. Số lượng các cuộc đấu tranh của công nhân những năm 1928 $-$ 1929 tăng gấp 2.5 lần so với 2 năm 1926 $-$ 1927.

Đến năm 1929, trước sự phát triển mạnh mẽ của phong trào cách mạng Việt Nam, tổ chức Hội Việt Nam Cách mạng Thanh niên không còn thích hợp và đủ sức lãnh đạo phong trào. Trước tình hình đó, tháng 3/1929, những người lãnh đạo Kỳ bộ Bắc Kỳ (Trần Văn Cung, Ngô Gia Tự, Nguyễn Đức Cảnh, Trịnh Đình Cửu, ...) họp tại số nhà 5D, phố Hàm Long, Hà Nội, quyết định thành lập Chi bộ Cộng sản đầu tiên ở Việt Nam. Ngày 17/6/1929, đại biểu của các tổ chức cộng sản ở Bắc Kỳ họp tại số nhà 312 phố Khâm Thiên (Hà Nội), quyết định thành lập \textit{Đông Dương Cộng sản Đảng}, thông qua Tuyên ngôn, Điều lệ; lấy cờ đỏ búa liềm là Đảng kỳ và quyết định xuất bản báo \textit{Búa liềm} làm cơ quan ngôn luận.

Trước ảnh hưởng của \textit{Đông Dương Cộng sản Đảng}, những thanh niên yêu nước ở Nam Kỳ theo xu hướng cộng sản, lần lượt tổ chức những chi bộ cộng sản. Tháng 11/1929, trên cơ sở các chi bộ cộng sản ở Nam Kỳ, \textit{An Nam Cộng sản Đảng} được thành lập tại Khánh Hội, Sài Gòn, công bố Điều lệ, quyết định xuất bản \textit{Tạp chí Bônsơvích}.

Tại Trung Kỳ, Tân Việt Cách mạng Đảng (là một tổ chức thanh niên yêu nước có cả Trẩn Phú, Nguyễn Thị Minh Khai, ...) chịu tác động mạnh mẽ của Hội Việt Nam Cách mạng Thanh niên $-$ đã đi theo khuynh hướng cách mạng vô sản. Tháng 9/1929, những người tiên tiến trong Tân Việt Cách mạng Đảng họp bàn việc thành lập \textit{Đông Dương Cộng sản Liên đoàn} và ra Tuyên đạt, khẳng định: "... những người giác ngộ cộng sản chân chính trong Tân Việt Cách mạng Đảng trịnh trọng tuyên ngôn cùng toàn thể đảng viên Tân Việt Cách mạng Đảng, toàn thể thợ thuyền dân cày và lao khổ biết rằng chúng tôi đã chánh thức lập ra \textit{Đông Dương Cộng sản Liên đoàn}... Muốn làm tròn nhiệm vụ thì trước mắt Đông Dương Cộng sản Liên đoàn là một mặt phải xây dựng cơ sở chi bộ của Liên đoàn tức là thực hành cải tổ Tân Việt Cách mạng Đảng thành đoàn thể cách mạng chân chính..." \footfullcite[tr. 404]{VKDtt1}. Đến cuối tháng 12/1929, tại Đại hội các đại biểu liên tỉnh tại nhà đồng chí Nguyễn Xuân Thanh $-$ Ủy viên Ban Chấp hành liên tỉnh (ga Chợ Thượng, huyện Đức Thọ, tỉnh Hà Tĩnh), nhất trí quyết định "Bỏ tên gọi Tân Việt. Đặt tên mới là Đông Dương Cộng sản Liên đoàn". Khi đang Đại hội, sợ bị lộ, các đại biểu di chuyển đến địa điểm mới thì bị địch bắt vào sáng ngày 01/1/1930. "Có thể coi những ngày cuối tháng 12/1929 là thời điểm hoàn tất quá trình thành lập Đông Dương Cộng sản Liên đoàn được khởi đầu từ sự kiện công bố Tuyên đạt tháng 9/1929" \footfullcite[tr. 404]{LSBN1}.
\subsection{Ý nghĩa lịch sử của việc thành lập Đảng Cộng sản Việt Nam}
Đảng Cộng sản Việt Nam ra đời đã chấm dứt sự khủng hoảng, bế tắc về đường lối cứu nước, đưa cách mạng Việt Nam sang một bước ngoặt lịch sử vĩ đại: cách mạng Việt Nam trở thành một bộ phận khăng khít của cách mạng vô sản thế giới. Đó là kết quả của sự vận động phát triển và thống nhất của phong trào cách mạng trong cả nước, sự chuẩn bị tích cực, sáng tạo, bản lĩnh của lãnh tụ Nguyễn Ái Quốc, sự đoàn kết, nhất trí của những chiến sĩ cách mạng tiên phong vì lợi ích của giai cấp và dân tộc. 

Sự ra đời của Đảng Cộng sản Việt Nam là sản phẩm của sự kết hợp chủ nghĩa Mác $-$ Lênin, tư tưởng Hồ Chí Minh với phong trào công nhân và phong trào yêu nước Việt Nam. Đó cũng là kết quả của sự phát triển cao và thống nhất của phong trào công nhân và phong trào yêu nước Việt Nam được soi sáng bởi chủ nghĩa Mác $-$ Lênin. Chủ tịch Hồ Chí Minh đã khẳng định: \textit{chủ nghĩa Mác $-$ Lênin kết hợp với phong trào công nhân và phong trào yêu nước đã dẫn tới việc thành lập Đảng}, "Việc thành lập Đảng là một bước ngoặt vô cùng quan trọng trong lịch sử cách mạng Việt Nam ta. Nó chứng tỏ rằng giai cấp vô sản ta đã trưởng thành và đủ sức lãnh đạo cách mạng" \footfullcite[tr. 406]{HCMtt12}.

Đảng Cộng sản Việt Nam ra đời với Cương lĩnh chính trị đầu tiên được thông qua tại Hội nghị thành lập Đảng đã khẳng định lần đầu tiên cách mạng Việt Nam có một bản cương lĩnh chính trị phản ánh được quy luật khách quan của xã hội Việt Nam, đáp ứng những nhu cầu cơ bản và cấp bách của xã hội Việt Nam, phù hợp với xu thế của thời đại, định hướng chiến lược đúng đắn cho tiến trình phát triển của cách mạng Việt Nam. Đường lối đó là kết quả của sự vận dụng chủ nghĩa Mác $-$ Lênin vào thực tiễn cách mạng Việt Nam một cách đúng đắn, sáng tạo và có phát triển trong điều kiện lịch sử mới.

Sự ra đời của Đảng Cộng sản Việt Nam với Cương lĩnh chính trị đầu tiên đã khẳng định sự lựa chọn con đường cách mạng cho dân tộc Việt Nam $-$ con đường cách mạng vô sản. Con đường duy nhất đúng giải phóng dân tộc, giải phóng giai cấp và giải phóng con người. Sự lựa chọn con đường cách mạng vô sản phù hợp với nội dung và xu thế của thời đại được mở ra từ Cách mạng tháng Mười Nga vĩ đại: "Đối với nước ta, không còn con đường nào khác để có độc lập dân tộc thật sự và tự do, hạnh phúc cho nhân dân. Cần nhấn mạnh rằng đây là sự lựa chọn của chính lịch sử, sự lựa chọn đã dứt khoát từ năm 1930 với sự ra đời của Đảng ta" \footfullcite[tr. 13-14]{VKDtt51}.

Đảng Cộng sản Việt Nam ra đời là bước ngoặt vĩ đại trong lịch sử phát triển của dân tộc Việt Nam, trở thành nhân tố hàng đầu quyết định đưa cách mạng Việt Nam đi từ thắng lợi này đến thắng lợi khác.
\section{Lãnh đạo quá trình đấu tranh giành chính quyền (1930 $-$ 1945)}

\subsection{Phong trào cách mạng 1930 $-$ 1931 và khôi phục phong trào 1932 $-$ 1935}
\subsubsection{Phong trào cách mạng 1930 $-$ 1931 và Luận cương chính trị (10/1930)}
\subsubsection{Luận cương chính trị của Đảng Cộng sản Đông Dương, tháng 10/1930}
\subsubsection{Cuộc đấu tranh khôi phục tổ chức và phong trào cách mạng, Đại hội Đảng lần thứ nhất (3/1935)}

\subsection{Phong trào dân chủ 1936 $-$ 1939}
\subsubsection{Điều kiện lịch sử và chủ trương của Đảng}
\subsubsection{Phong trào đấu tranhh đòi tự do, dân chủ, cơm áo, hòa bình}

\subsection{Phong trào giải phóng dân tộc 1939 $-$ 1945}
\subsubsection{Bối cảnh lịch sử và chủ trưng chiến lược mới của Đảng}
\subsubsection{Phong trào chống Pháp $-$ Nhật, đẩy mạnh chuẩn bị lực lượng cho cuộc khởi nghĩa vũ trang}
\subsubsection{Tổng khởi nghĩa giành chính quyền}

\subsection{Tính chất, ý nghĩa và kinh nghiệm của Cách mạng tháng Tám năm 1945}
\subsubsection{Tính chất}
\subsubsection{Ý nghĩa}
\subsubsection{Kinh nghiệm}