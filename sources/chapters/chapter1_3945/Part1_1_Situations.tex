\section{Đảng Cộng sản Việt Nam ra đời và Cương lĩnh chính trị đầu tiên của Đảng}
\subsection{Bối cảnh lịch sử}
\subsubsection{Tình hình thế giới cuối thế kỷ XIX, đầu thế kỷ XX}
Từ nửa sau thế kỷ XIX, các nước tư bản Âu $-$ Mỹ có những chuyển biến mạnh mẽ trong đời sống kinh tế $-$ xã hội. Chủ nghĩa tư bản phương Tây chuyển nhanh từ giai đoạn tự do cạnh tranh sang giai đoạn độc quyền (giai đoạn đế quốc chủ nghĩa), đẩy mạnh quá trình xâm chiếm và nô dịch các nước nhỏ, yếu ở châu Á, châu Phi và khu vực Mỹ $-$ Latinh, biến các quốc gia này thành thuộc đia của các nước đế quốc. Trước bối cảnh đó, nhân dân các dân tộc bị áp bức đã đứng lên đấu tranh tự giải phóng khỏi ách thực dân, đế quốc, tạo thành phong trào giải phóng dân tộc mạnh mẽ, rộng khắp, nhất là ở châu Á. Cùng với phong trào đấu tranh của giai cấp vô sản chống lại giai cấp tư sản ở các nước tư bản chủ nghĩa, phong trào giải phóng dân tộc ở các nước thuộc địa trở thành một bộ phận quan trọng trong cuộc đấu tranh chung chống tư bản, thực dân. Phong trào giải phóng dân tộc ở các nước châu Á đầu thế kỷ XX phát triển rộng khắp, tác động mạnh mẽ đến phong trào yêu nước Việt Nam.

Trong bối cảnh đó, thắng lợi của Cách mạng tháng Mười Nga năm 1917 đã làm biến đổi sâu sắc tình hình thế giới. Thắng lợi của Cách mạng tháng Mười Nga không chỉ có ý nghĩa to lớn đối với cuộc đấu tranh của giai cấp vô sản đối với các nước tư bản, mà còn có tác động sâu sắc đến phong trào giải phóng dân tộc ở các nước thuộc địa. Tháng 3/1919, Quốc tế Cộng sản, do V. I. Lênin đứng đầu, được thành lập, trở thành bộ tham mưu chiến đấu, tổ chức lãnh đạo phong trào cách mạng vô sản thế giới. Quốc tế Cộng sản không những vạch đường hướng chiến lược cho cách mạng vô sẩn mà cả đối với các vấn đề dân tộc và thuộc địa, giúp đỡ, chỉ đạo phong trào giải phóng dân tộc. Cùng với việc nghiên cứu và hoàn thiện chiến lược và sách lược về vấn đề dân tộc và thuộc địa, Quốc tế Cộng sản đã tiến hành hoạt động truyền bá tư tưởng cách mạng vô sản và thúc đẩy phong trào đấu tranh ở khu vực này đi theo khuynh hướng vô sản. Đại hội II của Quốc tế Cộng sản (1920) đã thông qua luận cương về dân tộc và thuộc địa do V. I. Lênin khởi xướng. Cách mạng tháng Mười và những hoạt động cách mạng của Quốc tế Cộng sản đã ảnh hưởng mạnh mẽ và thức tỉnh phong trào giải phóng dân tộc ở các nước thuộc địa, trong đó có Việt Nam và Đông Dương.

\subsubsection{Tình hình Việt Nam và các phong trào yêu nước trước khi có Đảng}
Là quốc gia Đông Nam Á nằm ở vị trí địa chính trị quan trọng của châu Á, Việt Nam trở thành đối tượng nằm trong mưu đồ xâm lược của thực dân Pháp trong cuộc chạy đua với nhiều đế quốc khác. Sau một quá trình điều tra thám sát lâu dài, thâm nhập kiên trì của các giáo sĩ và thương nhân Pháp, ngày 01/9/1858, thực dân Pháp nổ súng xâm lược Việt Nam tại Đà Nẵng và từ đó từng bước thôn tính Việt Nam. Đó là thời điểm chế độ phong kiến Việt Nam (dưới triều đại phong kiến nhà Nguyễn) đã lâm vào giai đoạn khủng hoảng trầm trọng. Trước hành động xâm lược của Pháp, triều đình nhà Nguyễn từng bước thỏa hiệp (HIệp ước 1862, 1874, 1883) và đến ngày 06/6/1884 với Hiệp ước Patenotre đã đầu hàng hoàn toàn thực dân Pháp, Việt Nam trở thành "một xứ thuộc địa, dân ta là vong quốc nô, Tổ quốc ta bị giày xéo dưới gót sắt của kẻ thù hung ác" \footfullcite[tr. 401]{HCMtt12}.

Tuy triều đình nhà Nguyễn đã đầu hàng thực dân Pháp, nhưng nhân dân Việt Nam vẫn không chịu khuất phục, thực dân Pháp dùng vũ lực để bình định, đàn áp sự nổi dậy của nhân dân. Đồng thời với việc dùng vũ lực đàn áp đẫm máu đối với các phong trào yêu nước của nhân dân Việt Nam, thực dân Pháp tiến hành xây dựng hệ thống chính quyền thuộc địa, bên cạnh đó vẫn duy trì chính quyền phong kiến bản xứ làm tay sai. Pháp thực hiện chính sách "chia để trị" nhằm phá vỡ khối đoàn kết cộng đồng quốc gia dân tộc: chia ba kỳ (Bắc Kỳ, Trung Kỳ, Nam Kỳ) với các chế độ chính trị khác nhau nằm trong \textit{Liên bang Đông Dương thuộc Pháp} (Union Indochinoise) \footnote{Bao gồm: Bắc Kỳ, Trung Kỳ, Nam Kỳ, Cao Miên, Ai Lao} được thành lập ngày 17/10/1887 theo sắc lệnh của Tổng thống Pháp.

Từ năm 1897, thực dân Pháp bắt đầu tiến hành các cuộc khai thác thuộc địa lớn: Cuộc khai thác thuộc địa lần thứ nhất (1897 $-$ 1914) do Toàn quyền Đông Dương Paul Doumer thực hiện và Cuộc khai thác thuộc địa lần thứ hai (1919 $-$ 1929). Mưu đồ của thực dân Pháp nhằm biến Việt Nam nói riêng và Đông Dương nói chung thành thị trường tiêu thụ hàng hóa của "chính quốc", đồng thời ra sức vơ vét tài nguyên, bóc lột sức lao động rẻ mạt của người bản xứ, cùng nhiều hình thức thuế khóa nặng nề.

Chế độ cai trị, bóc lột hà khắc của thực dân Pháp đối với nhân dân Việt Nam là "chế độ độc tài chuyên chế nhất, nó vô cùng khả ố và khủng khiếp hơn cả chế độ chuyên chế của nhà nước quân chủ châu Á thời xưa" \footnote{Bài đăng của Phan Văn Trường trên tờ \textit{La Cloché Félée} (Tiếng chuông rè), số 36, ngày 21/1/1926}. Năm 1862, Pháp đã lập nhà tù ở Côn Đảo để giam cầm những người Việt Nam yêu nước chống Pháp.

Về văn hóa $-$ xã hội, thực dân Pháp thực hiện chính sách "ngu dân" để dễ cai trị, lập nhà tù nhiều hơn trường học, đồng thời du nhập những giá trị phản văn hóa, duy trì tệ nạn xã hội vốn có của chế độ phong kiến và tạo nên nhiều tệ nạn xã hội mới, dùng rượu cồn và thuốc phiện để đầu độc các thế hệ người Việt Nam, ra sức tuyên truyền tư tưởng "khai hóa văn minh" của nước Đại Pháp ...

Chế độ áp bức về chính trị, bóc lột về kinh tế, nô dịch về văn hóa của thực dân Pháp đã làm biến đổi tình hình chính trị, kinh tế, xã hội Việt Nam. Các giai cấp cũ phân hóa, giai cấp, tầng lớp mới xuất hiện với địa vị kinh tế khác nhau và do đó cũng có thái độ chính trị khác nhau đối với vận mệnh dân tộc.

Dưới chế độ phong kiến, giai cấp địa chủ và nông dân là hai giai cấp cơ bản trong xã hội, khi Việt Nam trở thành thuộc địa của Pháp, giai cấp địa chủ bị phân hóa.

Một bộ phận địa chủ câu kết với thực dân Pháp và làm tay sai đắc lực cho Pháp trong việc ra sức đàn áp phong trào yêu nước và bóc lột nông dân; một bộ phận khác nêu cao tinh thần dân tộc khởi xướng và lãnh đạo các phong trào chống Pháp và bảo vệ chế độ phong kiến, tiêu biểu là phong trào Cần vương; một số trở thành lãnh đạo phong trào nông dân chống thực dân Pháp và phong kiến phản động; một bộ phận nhỏ chuyển sang kinh doanh theo lối tư bản.

Giai cấp nông dân chiếm số lượng đông đảo nhất (khoảng hơn $90\%$ dân số), đồng thời là một giai cấp bị phong kiến, thực dân bóc lột nặng nề nhất. Do vậy, ngoài mâu thuẫn giai cấp vốn có với giai cấp địa chủ, từ khi thực dân Pháp xâm lược, giai cấp nông dân còn có mâu thuẫn sâu sắc với thực dân xâm lược. "Tinh thần cách mạng của nông dân không chỉ gắn liền với ruộng đất, với đời sống hằng ngày của họ, mà còn gắn bó một cách sâu sắc với tình cảm quê hương, đất nước, với nền văn hóa hàng nghìn năm của dân tộc" \footfullcite[tr. 119]{LDGCCN}. Đây là lực lượng hùng hậu, có tinh thần đấu tranh kiên cường bất khuất cho nền độc lập tự do của dân tộc và khao khát giành lại ruộng đất cho dân cày, khi có lực lượng tiên phong lãnh đạo, giai cấp nông dân sẵn sàng vùng dậy làm cách mạng lật đổ thực dân phong kiến.

Giai cấp công nhân Việt Nam được hình thành gắn với các cuộc khai thác thuộc địa, với việc thực dân Pháp thiết lập các nhà máy, xí nghiệp, công xưởng, khu đồn điền, ... Ngoài những đặc điểm của giai cấp công nhân quốc tế, giai cấp công nhân Việt Nam có những đặc điểm riêng vì ra đời trong hoàn cảnh một nước thuộc địa nửa phong kiến, chủ yếu xuất thân từ nông dân, cơ cấu chủ yếu là công nhân khai thác mỏ, đồn điền, lực lượng còn nhỏ bé \footnote{Số lượng công nhân đến trước chiến tranh thế giới thứ nhất (1913) có khoảng 10 vạn người; đến cuối năm 1929, số công nhân Việt Nam là hơn 22 vạn người, chiếm trên $1.2\%$ dân số.}, nhưng sớm vươn lên tiếp nhận tư tưởng tiên tiến của thời đại, nhanh chóng phát triển từ "tự phát" đến "tự giác", thể hiện là giai cấp có năng lực lãnh đạo cách mạng.

Giai cấp tư sản Việt Nam xuất hiện muộn hơn giai cấp công nhân. Một bộ phận gắn liền lợi ích với tư bản Pháp, tham gia vào đời sống chính trị, kinh tế của chính quyền thực dân Pháp, trở thành tầng lớp tư sản mại bản. Một bộ phận là giai cấp tư sản dân tộc, họ bị thực dân Pháp chèn ép, kìm hãm, bị lệ thuộc, yếu ớt về kinh tế. Vì vậy, phần lớn tư sản dân tộc Việt Nam có tinh thần dân tộc, yêu nước nhưng không có khả năng tập hợp các giai tầng để tiến hành cách mạng.

Tầng lớp tiểu tư sản (tiểu thương, tiểu chủ, sinh viên, ...) bị đế quốc, tư bản chèn ép, khinh miệt, do đó có tinh thần dân tộc, yêu nước và rất nhạy cảm về chính trị và thời cuộc. Tuy nhiên, do địa vị kinh tế bấp bênh, thái độ hay dao động, thiếu kiên định, do đó tầng lớp tiểu tư sản không thể lãnh đạo cách mạng.

Các sĩ phu phong kiến cũng có sự phân hóa. Một bộ phận hướng sang tư tưởng dân chủ tư sản hoặc tư tưởng vô sản. Một số người khởi xướng các phong trào yêu nước có ảnh hưởng lớn.

Cuối thế kỷ XĨ đầu thế kỷ XX, Việt Nam đã có sự biến đổi rất quan trọng cả về chính trị, kinh tế, xã hội. Chính sách cai trị và khai thác bóc lột của thực dân Pháp đã làm phân hóa những giai cấp vốn là của chế độ phong kiến (địa chủ, nông dân), đồng thời tạo nên những giai cấp, tầng lớp mới (công nhân, tư sản dân tộc, tiểu tư sản) với thái độ chính trị khác nhau. Những mâu thuẫn mới trong xã hội Việt Nam xuất hiện. Trong đó, mâu thuẫn giữa toàn thể dân tộc Việt Nam với thực dân Pháp và phong kiến phản động trở thành mâu thuẫn chủ yếu nhất và ngày càng gay găt.

Trong bối cảnh đó, những luồng tư tưởng ở bên ngoài: tư tưởng Cách mạng tư sản Pháp 1789, phong trào Duy tân Nhật Bản năm 1868, cuộc vận động Duy tân tại Trung Quốc năm 1898, Cách mạng Tân Hợi của Trung Quốc năm 1911, ... đặc biệt là Cách mạng tháng Mười Nga năm 1917 đã tác động mạnh mẽ, làm chuyển biến phong trào yêu nước những năm cuối thế kỷ XIX, đầu thế kỷ XX. Năm 1919, trên chiến hạm của Pháp ở Hắc Hải (Biển Đen), Tôn Đức Thắng tham gia đấu tranh chống việc can thiệp vào nước Nga Xô viết. Năm 1923, luật sư Phan Văn Trường từ Pháp về nước và ông công bố tác phẩm của C. Mác và Ph. Ăngghen: \textit{Tuyên ngôn của Đảng Cộng sản} (The Manifesto of the Communist Party) trên báo \textit{La Cloché Félée}, từ số ra ngày 29/3 đến 20/4/1926, tại Sài Gòn, góp phần tuyên truyền tư tưởng vô sản ở Việt Nam.

\subsubsection{Các phong trào yêu nước của nhân dân Việt Nam trước khi có Đảng}
Ngay từ khi Pháp xâm lược, các phong trào yêu nước chống thực dân Pháp với tinh thần quật cường, bảo vệ nền độc lập dân tộc của nhân dân Việt Nam đã diễn ra liên tục, rộng khắp.

Đến năm 1884, mặc dù triều đình phong kiến nhà Nguyễn đã đầu hàng, nhưng một bộ phận phong kiến yêu nước đã cùng với nhân dân vẫn tiếp tục đấu tranh vũ trang chống Pháp.

Đó là phong trào Cần vương do vua Hàm Nghi và Tôn Thất Thuyết khởi xướng (1885 $-$ 1896). Hưởng ứng lời kêu gọi Cần vương cứu nước, các cuộc khởi nghĩa Ba ĐÌnh (Thanh Hóa), Bãi Sậy (Hưng Yên), Hương Khê (Hà Tĩnh), ... diễn ra sôi nổi và thể hiện tinh thần quật cường chống ngoại xâm của các tầng lớp nhân dân. NHưng ngọn cờ phong kiến lúc đó không còn là ngọn cờ tiêu biểu để tập hợp một cách rộng rãi, toàn thể các tầng lớp nhân dân, không có khả năng liên kết các trung tâm kháng Pháp trên toàn quốc nữa. Cuộc khởi nghĩa của Phan Đình Phùng thất bại (1896) cũng là mốc chấm dứt vai trò lãnh đạo của giai cấp phong kiến đối với phong trào yêu nước chống thực dân Pháp ở Việt Nam. Đầu thế kỷ XX, vua Thành Thái và vua Duy Tân tiếp tục đấu tranh chống Pháp, trong đó có khởi nghĩa của vua Duy Tân (5/1916).

Vào những năm cuối thế kỷ XIX đầu thế kỷ XX, ở vùng miền núi và trung du phía Bắc, \textit{phong trào nông dân} Yên Thế (Bắc Giang) dưới sự lãnh đạo của vị thủ lĩnh nông dân Hoàng Hoa Thám, nghĩa quân đã xây dựng lực lượng chiến đấu, lập căn cứ và đấu tranh kiên cường chống thực dân Pháp. Nhưng phong trào của Hoàng Hoa Thám vẫn mang nặng "cốt cách phong kiến", không có khả năng mở rộng hợp tác và thống nhất tạo thành một cuộc cách mạng giải phóng dân tộc, cuối cùng cũng bị thực dân Pháp đàn áp.

Từ những năm đầu thế kỷ XX, phong trào yêu nước Việt Nam chịu ảnh hưởng, tác động của \textit{trào lưu dân chủ tư sản}, tiêu biểu là xu hướng bạo động của Phan Bội Châu, xu hướng cải cách của Phan Châu Trinh và sau đó là phong trào tiểu tư sản trí thức của tổ chức Việt Nam Quốc dân Đảng (12/1927 $-$ 2/1930) đã tiếp tục diễn ra rộng khắp các tỉnh Bắc Kỳ, nhưng tất cả đều không thành công.

Xu hướng bạo động do Phan Bội Châu tổ chức, lãnh đạo: với chủ trương tập hợp lực lượng với phương pháp bạo động chống Pháp, xây dựng chế độ chính trị như ở Nhật Bản, phong trào theo xu hướng này tổ chức đưa thanh niên yêu nước Việt Nam sang Nhật Bản học tập (gọi là phong trào \textit{Đông du}). Đến năm 1908, Chính phủ Nhật Bản câu kết với thực dân Pháp trục xuất lưu học sinh Việt Nam và những người đứng đầu. Sau khi phong trào Đông du thất bại, năm 1912, Phan Bội Châu thành lập tổ chức \textit{Việt Nam Quang phục Hội} với tôn chỉ là vũ trang đánh đuổi thực dân Pháp, khôi phục VIệt Nam, thành lập nước cộng hòa dân quốc Việt Nam. Nhưng chương trình, kế hoạch hoạt động của Hội lại thiếu rõ ràng. Cuối năm 1913, Phan Bội Châu bị thực dân Pháp bắt giam tại Trung Quốc cho tới đầu năm 1917 và sau này bị quản chế tại Huế cho đến khi ông mất (1940). Ảnh hưởng xu hướng bạo động của tổ chức \textit{Việt Nam Quang phục Hội} đối với phong trào yêu nước Việt Nam đến đây chấm dứt.

Xu hướng cải cách của Phan Châu Trinh: Phan Châu Trinh và những người cùng chí hướng muốn giành độc lập cho dân tộc nhưng không đi theo con đường bạo động như Phan Bội Châu, mà chủ trưởng cải cách đất nước. Phan Châu Trinh cho rằng "bất bạo động, bạo động tắc tử"; phải "khai dân trí, chấn dân khí, hậu dân sinh", phải bãi bỏ chế độ quân chủ, thực hiện dân quyền, khai thông dân trí, mở mang thực nghiệp. Để thực hiện được chủ trương ấy, Phan Châu Trinh đã đề nghị Nhà nước "bảo hộ" Pháp tiến hành cải cách. Đó chính là sự hạn chế trong xu hướng cải cách để cứu nước, vì Phan Châu Trinh đã "đặt vào lòng độ lượng của Pháp cái hi vọng cải tử hoàn sinh cho nước Nam, ... Cụ không rõ bản chất của đế quốc thực dân" \footfullcite[tr. 442]{TVG}. Do vậy, khi phong trào Duy tân lan rộng khắp cả Trung Kỳ và Nam Kỳ, đỉnh cao là vụ chống thuế ở Trung Kỳ (1908), thực dân Pháp đã đàn áp dã man, giết hại nhiều sĩ phu và nhân dân tham gia biểu tình. Nhiều sĩ phu bị bắt, bị đày đi Côn Đảo, trong đó có Phan Châu Trinh, Huỳnh Thúc Kháng, Đặng Nguyên Cẩn, ... Phong trào chống thuế ở Trung Kỳ bị thực dân Pháp dập tắt, cùng với sự kiện tháng 12/1907, thực dân Pháp ra lệnh đóng cửa Trường Đông Kinh Nghĩa Thục \footnote{Trường Đông Kinh Nghĩa Thục do Lương Văn Can, Nguyễn Quyền, Hoàng Tăng Bí, ... thành lập ở Hà Nội, nhằm truyền bá tư tưởng dân chủ, tự do tư sản, nâng cao lòng tự tôn dân tộc cho thanh niên Việt Nam.} phản ánh sự kết thúc xu hướng cải cách trong phong trào cứu nước của Việt Nam.

Phong trào của tổ chức Việt Nam Quốc dân Đảng: khi thực dân Pháp đẩy mạnh khai thác thuộc địa lần thứ hai, mâu thuẫn giữa toàn thể dân tộc Việt Nam với thực dân Pháp càng trở nên gay gắt, các giai cấp, tầng lớp mới trong xã hội Việt Nam đều bước lên vũ đài chính trị. Trong đó, hoạt động có ảnh hưởng rộng và thu hút nhiều học sinh, sinh viên yêu nước ở Bắc Kỳ là tổ chức \textit{Việt Nam Quốc dân Đảng} do Nguyễn Thái Học lãnh đạo. Trên cơ sở các tổ chức yêu nước của tiểu tư sản trí thức, Việt Nam Quốc dân Đảng được chính thức thành lập tháng 12/1927 ở Bắc Kỳ.

Mục đích của Việt Nam Quốc dân Đảng là đánh đuổi thực dân Pháp xâm lược, giành độc lập dân tộc, xây dựng chế độ cộng hòa tư sản, với phương pháp đấu tranh vũ trang nhưng theo lối manh động, ám sát cá nhân và lực lượng chủ yếu là binh lính, sinh viên. Cuộc khởi nghĩa nổ ra ở một số tỉnh, chủ yếu và mạnh nhất là ở Yên Bái (2/1930), tuy oanh liệt nhưng nhanh chóng bị thất bại. Sự thất bại của khởi nghĩa Yên Bái của tổ chức Việt Nam Quốc dân Đảng đã thể hiện là "... một cuộc bạo động bất đắc dĩ, một cuộc bạo động non, để rồi chết luôn không bao giờ ngóc đầu lên nooiro. Khẩu hiệu "không thành công thì thành nhân" biểu lộ tính chất hấp tấp tiểu tư sản, tính chất hăng hái nhất thời và đồng thời cũng biểu lộ tính chất không vững chắc, non yếu của phong trào tư sản" \footfullcite[tr. 41]{LDDD}.

Vào những năm cuối thế kỷ XIX đầu thế kỷ XX, tiếp tục truyền thống yêu nước, bất khuất kiên cường chống ngoại xâm, các phong trào yêu nước theo ngọn cờ phong kiến, ngọn cờ dân chủ tư sản của nhân dân Việt Nam đã diễn ra quyết liệt, liên tục và rộng khắp. Dù với nhiều cách thức tiến hành khác nhau, song đều hướng tới mục tiêu giành độc lập cho dân tộc. Tuy nhiên, "các phong trào cứu nước từ lập trường Cần vương đến lập trường tư sản, tiểu tư sản qua khảo nghiệm lịch sử đều lần lượt thất bại" \footfullcite[tr. 14]{VKDtt51}. Nguyên nhân thất bại của các phong trào đó là do thiếu đường lối chính trị đúng đắn để giải quyết triệt để những mâu thuẫn cơ bản, chủ yếu của xã hội, chưa có một tổ chức vững mạnh để tập hợp, giác ngộ và lãnh đạo toàn dân tộc, chưa xác định được phương pháp đấu tranh thích hợp để đánh đổ kẻ thù.

Các phong trào yêu nước ở Việt Nam cho đến những năm 20 của thế kỷ XX đều thất bại, nhưng đã góp phần cổ vũ mạnh mẽ tinh thần yêu nước của nhân dân, bồi đắp thêm cho chủ nghĩa yêu nước Việt Nam, đặc biệt góp phần thúc đẩy những nhà yêu nước, nhất là lớp thanh niên tri thức tiên tiến chọn lựa một con đường mới, một giải pháp cứu nước, giải phóng dân tộc theo xu thế của thời đại. Nhiệm vụ lịch sử cấp thiết đặt ra cho thế hệ yêu nước đương thời là cần phải có một tổ chức cách mạng tiên phong, có đường lối cứu nước đúng đắn để giải phóng dân tộc.