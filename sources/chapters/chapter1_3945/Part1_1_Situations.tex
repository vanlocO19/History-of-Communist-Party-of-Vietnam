\section{Đảng Cộng sản Việt Nam ra đời và Cương lĩnh chính trị đầu tiên của Đảng}
\subsection{Bối cảnh lịch sử}
\subsubsection{Tình hình thế giới cuối thế kỷ XIX, đầu thế kỷ XX}
Từ nửa sau thế kỷ XIX, các nước tư bản Âu $-$ Mỹ có những chuyển biến mạnh mẽ trong đời sống kinh tế $-$ xã hội. Chủ nghĩa tư bản phương Tây chuyển nhanh từ giai đoạn tự do cạnh tranh sang giai đoạn độc quyền (giai đoạn đế quốc chủ nghĩa), đẩy mạnh quá trình xâm chiếm và nô dịch các nước nhỏ, yếu ở châu Á, châu Phi và khu vực Mỹ $-$ Latinh, biến các quốc gia này thành thuộc đia của các nước đế quốc. Trước bối cảnh đó, nhân dân các dân tộc bị áp bức đã đứng lên đấu tranh tự giải phóng khỏi ách thực dân, đế quốc, tạo thành phong trào giải phóng dân tộc mạnh mẽ, rộng khắp, nhất là ở châu Á. Cùng với phong trào đấu tranh của giai cấp vô sản chống lại giai cấp tư sản ở các nước tư bản chủ nghĩa, phong trào giải phóng dân tộc ở các nước thuộc địa trở thành một bộ phận quan trọng trong cuộc đấu tranh chung chống tư bản, thực dân. Phong trào giải phóng dân tộc ở các nước châu Á đầu thế kỷ XX phát triển rộng khắp, tác động mạnh mẽ đến phong trào yêu nước Việt Nam.

Trong bối cảnh đó, thắng lợi của Cách mạng tháng Mười Nga năm 1917 đã làm biến đổi sâu sắc tình hình thế giới. Thắng lợi của Cách mạng tháng Mười Nga không chỉ có ý nghĩa to lớn đối với cuộc đấu tranh của giai cấp vô sản đối với các nước tư bản, mà còn có tác động sâu sắc đến phong trào giải phóng dân tộc ở các nước thuộc địa. Tháng 3/1919, Quốc tế Cộng sản, do V. I. Lênin đứng đầu, được thành lập, trở thành bộ tham mưu chiến đấu, tổ chức lãnh đạo phong trào cách mạng vô sản thế giới. Quốc tế Cộng sản không những vạch đường hướng chiến lược cho cách mạng vô sẩn mà cả đối với các vấn đề dân tộc và thuộc địa, giúp đỡ, chỉ đạo phong trào giải phóng dân tộc. Cùng với việc nghiên cứu và hoàn thiện chiến lược và sách lược về vấn đề dân tộc và thuộc địa, Quốc tế Cộng sản đã tiến hành hoạt động truyền bá tư tưởng cách mạng vô sản và thúc đẩy phong trào đấu tranh ở khu vực này đi theo khuynh hướng vô sản. Đại hội II của Quốc tế Cộng sản (1920) đã thông qua luận cương về dân tộc và thuộc địa do V. I. Lênin khởi xướng. Cách mạng tháng Mười và những hoạt động cách mạng của Quốc tế Cộng sản đã ảnh hưởng mạnh mẽ và thức tỉnh phong trào giải phóng dân tộc ở các nước thuộc địa, trong đó có Việt Nam và Đông Dương.

\subsubsection{Tình hình Việt Nam và các phong trào yêu nước trước khi có Đảng}
Là quốc gia Đông Nam Á nằm ở vị trí địa chính trị quan trọng của châu Á, Việt Nam trở thành đối tượng nằm trong mưu đồ xâm lược của thực dân Pháp trong cuộc chạy đua với nhiều đế quốc khác. Sau một quá trình điều tra thám sát lâu dài, thâm nhập kiên trì của các giáo sĩ và thương nhân Pháp, ngày 01/9/1858, thực dân Pháp nổ súng xâm lược Việt Nam tại Đà Nẵng và từ đó từng bước thôn tính Việt Nam. Đó là thời điểm chế độ phong kiến Việt Nam (dưới triều đại phong kiến nhà Nguyễn) đã lâm vào giai đoạn khủng hoảng trầm trọng. Trước hành động xâm lược của Pháp, triều đình nhà Nguyễn từng bước thỏa hiệp (HIệp ước 1862, 1874, 1883) và đến ngày 06/6/1884 với Hiệp ước Patenotre đã đầu hàng hoàn toàn thực dân Pháp, Việt Nam trở thành "một xứ thuộc địa, dân ta là vong quốc nô, Tổ quốc ta bị giày xéo dưới gót sắt của kẻ thù hung ác" \footfullcite[tr. 401]{HCMtt12}.

Tuy triều đình nhà Nguyễn đã đầu hàng thực dân Pháp, nhưng nhân dân Việt Nam vẫn không chịu khuất phục, thực dân Pháp dùng vũ lực để bình định, đàn áp sự nổi dậy của nhân dân. Đồng thời với việc dùng vũ lực đàn áp đẫm máu đối với các phong trào yêu nước của nhân dân Việt Nam, thực dân Pháp tiến hành xây dựng hệ thống chính quyền thuộc địa, bên cạnh đó vẫn duy trì chính quyền phong kiến bản xứ làm tay sai. Pháp thực hiện chính sách "chia để trị" nhằm phá vỡ khối đoàn kết cộng đồng quốc gia dân tộc: chia ba kỳ (Bắc Kỳ, Trung Kỳ, Nam Kỳ) với các chế độ chính trị khác nhau nằm trong \textit{Liên bang Đông Dương thuộc Pháp} (Union Indochinoise) \footnote{Bao gồm: Bắc Kỳ, Trung Kỳ, Nam Kỳ, Cao Miên, Ai Lao} được thành lập ngày 17/10/1887 theo sắc lệnh của Tổng thống Pháp.

Từ năm 1897, thực dân Pháp bắt đầu tiến hành các cuộc khai thác thuộc địa lớn: Cuộc khai thác thuộc địa lần thứ nhất (1897 $-$ 1914) do Toàn quyền Đông Dương Paul Doumer thực hiện và Cuộc khai thác thuộc địa lần thứ hai (1919 $-$ 1929). Mưu đồ của thực dân Pháp nhằm biến Việt Nam nói riêng và Đông Dương nói chung thành thị trường tiêu thụ hàng hóa của "chính quốc", đồng thời ra sức vơ vét tài nguyên, bóc lột sức lao động rẻ mạt của người bản xứ, cùng nhiều hình thức thuế khóa nặng nề.

Chế độ cai trị, bóc lột hà khắc của thực dân Pháp đối với nhân dân Việt Nam là "chế độ độc tài chuyên chế nhất, nó vô cùng khả ố và khủng khiếp hơn cả chế độ chuyên chế của nhà nước quân chủ châu Á thời xưa" \footnote{Bài đăng của Phan Văn Trường trên tờ \textit{La Cloché Félée} (Tiếng chuông rè), số 36, ngày 21/1/1926}. Năm 1862, Pháp đã lập nhà tù ở Côn Đảo để giam cầm những người Việt Nam yêu nước chống Pháp.

Về văn hóa $-$ xã hội, thực dân Pháp thực hiện chính sách "ngu dân" để dễ cai trị, lập nhà tù nhiều hơn trường học, đồng thời du nhập những giá trị phản văn hóa, duy trì tệ nạn xã hội vốn có của chế độ phong kiến và tạo nên nhiều tệ nạn xã hội mới, dùng rượu cồn và thuốc phiện để đầu độc các thế hệ người Việt Nam, ra sức tuyên truyền tư tưởng "khai hóa văn minh" của nước Đại Pháp ...

Chế độ áp bức về chính trị, bóc lột về kinh tế, nô dịch về văn hóa của thực dân Pháp đã làm biến đổi tình hình chính trị, kinh tế, xã hội Việt Nam. Các giai cấp cũ phân hóa, giai cấp, tầng lớp mới xuất hiện với địa vị kinh tế khác nhau và do đó cũng có thái độ chính trị khác nhau đối với vận mệnh dân tộc.

Dưới chế độ phong kiến, giai cấp địa chủ và nông dân là hai giai cấp cơ bản trong xã hội, khi Việt Nam trở thành thuộc địa của Pháp, giai cấp địa chủ bị phân hóa.

Một bộ phận địa chủ câu kết với thực dân Pháp và làm tay sai đắc lực cho Pháp trong việc ra sức đàn áp phong trào yêu nước và bóc lột nông dân; một bộ phận khác nêu cao tinh thần dân tộc khởi xướng và lãnh đạo các phong trào chống Pháp và bảo vệ chế độ phong kiến, tiêu biểu là phong trào Cần vương; một số trở thành lãnh đạo phong trào nông dân chống thực dân Pháp và phong kiến phản động; một bộ phận nhỏ chuyển sang kinh doanh theo lối tư bản.

Giai cấp nông dân chiếm số lượng đông đảo nhất (khoảng hơn $90\%$ dân số), đồng thời là một giai cấp bị phong kiến, thực dân bóc lột nặng nề nhất. Do vậy, ngoài mâu thuẫn giai cấp vốn có với giai cấp địa chủ, từ khi thực dân Pháp xâm lược, giai cấp nông dân còn có mâu thuẫn sâu sắc với thực dân xâm lược. "Tinh thần cách mạng của nông dân không chỉ gắn liền với ruộng đất, với đời sống hằng ngày của họ, mà còn gắn bó một cách sâu sắc với tình cảm quê hương, đất nước, với nền văn hóa hàng nghìn năm của dân tộc" \footfullcite[tr. 119]{LDGCCN}. Đây là lực lượng hùng hậu, có tinh thần đấu tranh kiên cường bất khuất cho nền độc lập tự do của dân tộc và khao khát giành lại ruộng đất cho dân cày, khi có lực lượng tiên phong lãnh đạo, giai cấp nông dân sẵn sàng vùng dậy làm cách mạng lật đổ thực dân phong kiến.