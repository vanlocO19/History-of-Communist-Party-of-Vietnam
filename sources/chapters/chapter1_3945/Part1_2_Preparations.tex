\subsection{Nguyễn Ái Quốc chuẩn bị các điều kiện để thành lập Đảng}
\subsubsection{Khái quát quá trình tìm đường cứu nước}
Trước yêu cầu cấp thiết giải phóng dân tộc của nhân dân Việt Nam, với nhiệt huyết cứu nước, với nhãn quan chính trị sắc bén, vượt lên trên hạn chế của các bậc yêu nước đương thời, năm 1911, Nguyễn Tất Thành quyết định ra đi tìm đường cứu nước, giải phóng dân tộc. Qua trải nghiệm thực tế ở nhiều nước, Người đã nhận thức được một cách rạch ròi rằng: "dù màu da có khác nhau, trên đời này chỉ có \textit{hai giống người: giống người bóc lột và giống người bị bóc lột}", từ đó xác định rõ kẻ thù và lực lượng đồng minh của nhân dân các dân tộc bị áp bức.

Năm 1917, thắng lợi của Cách mạng tháng Mười Nga đã tác động mạnh mẽ tới nhận thức của Nguyễn Tất Thành $-$ đây là cuộc "cách mạng đến nơi". Người từ nước Anh trở lại nước Pháp và tham gia các hoạt động chính trị hướng về tìm hiểu con đường Cách mạng tháng Mười Nga, về V. I. Lênin.

Đầu năm 1919, Nguyễn Tất Thành tham gia Đảng Xã hội Pháp, một chính đảng tiến bộ nhất luc đó ở Pháp. Tháng 6/1919, tại Hội nghị của các nước thắng trận trong Chiến tranh thế giới thứ nhất họp ở Versailles (Pháp), Tổng thống Mỹ Wooderow Wilson tuyên bố bảo đảm về quyền dân tộc tự quyết cho các nước thuộc địa. Nguyễn Tất Thành lấy tên là Nguyễn Ái Quốc thay mặt \textit{Hội những người An Nam yêu nước} ở Pháp gửi tới Hội nghị bản Yêu sách của nhân dân An Nam (gồm tám điểm đòi quyền tự do cho nhân dân Việt Nam) ngày 18/6/1919. Nhóm người Việt Nam tiêu biểu cho tinh thần yêu nước ở Pháp, gồm: Phan Châu Trinh, Nguyễn An Ninh, Phan Văn Trường, Nguyễn Thế Truyền và Nguyễn Ái Quốc. Những yêu sách đó dù không được Hội nghị đáp ứng, nhưng sự kiện này đã tạo nên tiếng vang lớn trong dư luận quốc tế và Nguyễn Ái Quốc càng hiểu rõ hơn bản chất của đế quốc, thực dân.

Tháng 7/1920, Người đọc bản \textit{Sơ thảo lần thứ nhất những luận cương về vấn đề dân tộc và vấn đề thuộc địa} của V. I. Lênin đăng trên báo \textit{L'Humanite} (Nhân đạo), số ra ngày 16 và 17/7/1920. Những luận điểm của V. I. Lênin về vấn đề dân tộc và thuộc địa đã giải đáp những vấn đề cơ bản và chỉ dẫn hướng phát triển của sự nghiệp cứu nước, giải phóng dân tộc. Lý luận của V. I. Lênin và lập trường đúng đắn của Quốc tế Cộng sản về cách mạng giải phóng các dân tộc thuộc địa là cơ sở để Nguyễn Ái Quốc xác định thái độ ủng hộ việc gia nhập Quốc tế Cộng sản tại Đại hội lần thứ XVIII của Đảng Xã hội Pháp (12/1920) tại thành phố Tour. Tại Đại hội này, Nguyễn Ái Quốc đã bỏ phiếu tán thành Quốc tế III (Quốc tế Cộng sản do V. I. Lênin thành lập).

Ngay sau đó, Nguyễn Ái Quốc cùng với những người vừa bỏ phiếu tán thành Quóc tế Cộng sản đã tuyên bố thành lập \textit{Phân bộ Pháp của Quốc tế Cộng sản} $-$ tức là Đảng Cộng sản Pháp. Với sự kiện này, Nguyễn Ái Quốc trở thành một trong những sáng lập viên của Đảng Cộng sản Pháp và là người cộng sản đầu tiên của Việt Nam, đánh dấu bước chuyển biến quyết định trong tư tưởng và lập trường chính trị của Nguyễn Ái Quốc. Trong những năm 1919 $-$ 1921, Bộ trưởng Bộ Thuộc địa Pháp Albert Sarraut nhiều lần gặp Nguyễn Ái Quốc mua chuộc và đe dọa. Ngày 30/6/1923, Nguyễn Ái Quốc tới Liên Xô và làm việc tại Quốc tế Cộng sản ở Moscow, tham gia nhiều hoạt động, đặc biệt là dự và đọc tham luận tại Đại hội V Quốc tế Cộng sản (17/6 $-$ 07/8/1924), làm việc trực tiếp ở Ban Phương Đông của Quốc tế Cộng sản.

Sau khi xác định được con đường cách mạng đúng đắn, Nguyễn Ái Quốc tiếp tục khảo sát, tìm hiểu để hoàn thiện nhận thức về đường lối cách mạng vô sản, đồng thời tích cực truyền bá chủ nghĩa Mác $-$ Lênin về Việt Nam.

\subsubsection{Chuẩn bị về tư tưởng, chính trị và tổ chức cho sự ra đời của Đảng}
\paragraph{Về tư tưởng}
Từ giữa năm 1921 tại Pháp, cùng một số nhà cách mạng của các nước thuộc địa khác, Nguyễn Ái Quốc tham gia thành lập Hội liên hiệp thuộc địa, sau đó sáng lập ra tờ báo \textit{Le Paria} (Người cùng khổ). Người viết nhiều bài trên các báo \textit{Nhân đạo, Đời sống công nhân, Tạp chí Cộng sản, Tập san Thư tín quốc tế}, ...

Năm 1922, \textit{Ban Nghiên cứu thuộc địa} của Đảng Cộng sản Pháp được thành lập, Nguyễn Ái Quốc được cư làm Trưởng Tiểu ban Nghiên cứu về Đông Dương. Vừa nghiên cứu lý luận, vừa tham gia hoạt động thực tiễn trong phong trào cộng sản và công nhân quốc tế, dưới nhiều phương thức phong phú, Nguyễn Ái Quốc tích cực tố cáo, lên án bản chất áp bức, bóc lột, nô dịch của chủ nghĩa thực dân đối với nhân dân các nước thuộc địa và kêu gọi, thức tỉnh nhân dân bị áp bức đấu tranh giải phóng. Người chỉ rõ bản chất của chủ nghĩa thực dân, xác định chủ nghĩa thực dân là kẻ thù chung của các dân tộc thuộc địa, của giai cấp công nhân và nhân dân lao động trên thế giới. Đồng thời, Người tiến hành tuyên truyền tư tưởng về con đường cách mạng vô sản, con đường cách mạng theo \textit{lý luận Mác $-$ Lênin}, xây dựng mối quan hệ gắn bó giữa những người cộng sản và nhân dân lao động Pháp với các nước thuộc địa và phụ thuộc.

Năm 1927, Nguyễn Ái Quốc khẳng định: "Đảng muốn vững phải có chủ nghĩa làm cốt, trong đảng ai cũng phải hiểu, ai cũng phải theo chủ nghĩa ấy" \footfullcite[tr. 289]{HCMtt2}. Đảng mà không có chủ nghĩa cũng giống như người không có trí khôn, tàu không có bàn chỉ nam. Phải truyền bá tư tưởng vô sản, lý luận Mác $-$ Lênin vào phong trào công nhân và phong trào yêu nước Việt Nam.