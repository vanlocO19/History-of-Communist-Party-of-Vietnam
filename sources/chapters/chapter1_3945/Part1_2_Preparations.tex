\subsection{Nguyễn Ái Quốc chuẩn bị các điều kiện để thành lập Đảng}
\subsubsection{Khái quát quá trình tìm đường cứu nước}
Trước yêu cầu cấp thiết giải phóng dân tộc của nhân dân Việt Nam, với nhiệt huyết cứu nước, với nhãn quan chính trị sắc bén, vượt lên trên hạn chế của các bậc yêu nước đương thời, năm 1911, Nguyễn Tất Thành quyết định ra đi tìm đường cứu nước, giải phóng dân tộc. Qua trải nghiệm thực tế ở nhiều nước, Người đã nhận thức được một cách rạch ròi rằng: "dù màu da có khác nhau, trên đời này chỉ có \textit{hai giống người: giống người bóc lột và giống người bị bóc lột}", từ đó xác định rõ kẻ thù và lực lượng đồng minh của nhân dân các dân tộc bị áp bức.

Năm 1917, thắng lợi của Cách mạng tháng Mười Nga đã tác động mạnh mẽ tới nhận thức của Nguyễn Tất Thành $-$ đây là cuộc "cách mạng đến nơi". Người từ nước Anh trở lại nước Pháp và tham gia các hoạt động chính trị hướng về tìm hiểu con đường Cách mạng tháng Mười Nga, về V. I. Lênin.

Đầu năm 1919, Nguyễn Tất Thành tham gia Đảng Xã hội Pháp, một chính đảng tiến bộ nhất luc đó ở Pháp. Tháng 6/1919, tại Hội nghị của các nước thắng trận trong Chiến tranh thế giới thứ nhất họp ở Versailles (Pháp), Tổng thống Mỹ Wooderow Wilson tuyên bố bảo đảm về quyền dân tộc tự quyết cho các nước thuộc địa. Nguyễn Tất Thành lấy tên là Nguyễn Ái Quốc thay mặt \textit{Hội những người An Nam yêu nước} ở Pháp gửi tới Hội nghị bản Yêu sách của nhân dân An Nam (gồm tám điểm đòi quyền tự do cho nhân dân Việt Nam) ngày 18/6/1919. Nhóm người Việt Nam tiêu biểu cho tinh thần yêu nước ở Pháp, gồm: Phan Châu Trinh, Nguyễn An Ninh, Phan Văn Trường, Nguyễn Thế Truyền và Nguyễn Ái Quốc. Những yêu sách đó dù không được Hội nghị đáp ứng, nhưng sự kiện này đã tạo nên tiếng vang lớn trong dư luận quốc tế và Nguyễn Ái Quốc càng hiểu rõ hơn bản chất của đế quốc, thực dân.

Tháng 7/1920, Người đọc bản \textit{Sơ thảo lần thứ nhất những luận cương về vấn đề dân tộc và vấn đề thuộc địa} của V. I. Lênin đăng trên báo \textit{L'Humanite} (Nhân đạo), số ra ngày 16 và 17/7/1920. Những luận điểm của V. I. Lênin về vấn đề dân tộc và thuộc địa đã giải đáp những vấn đề cơ bản và chỉ dẫn hướng phát triển của sự nghiệp cứu nước, giải phóng dân tộc. Lý luận của V. I. Lênin và lập trường đúng đắn của Quốc tế Cộng sản về cách mạng giải phóng các dân tộc thuộc địa là cơ sở để Nguyễn Ái Quốc xác định thái độ ủng hộ việc gia nhập Quốc tế Cộng sản tại Đại hội lần thứ XVIII của Đảng Xã hội Pháp (12/1920) tại thành phố Tour. Tại Đại hội này, Nguyễn Ái Quốc đã bỏ phiếu tán thành Quốc tế III (Quốc tế Cộng sản do V. I. Lênin thành lập).

Ngay sau đó, Nguyễn Ái Quốc cùng với những người vừa bỏ phiếu tán thành Quóc tế Cộng sản đã tuyên bố thành lập \textit{Phân bộ Pháp của Quốc tế Cộng sản} $-$ tức là Đảng Cộng sản Pháp. Với sự kiện này, Nguyễn Ái Quốc trở thành một trong những sáng lập viên của Đảng Cộng sản Pháp và là người cộng sản đầu tiên của Việt Nam, đánh dấu bước chuyển biến quyết định trong tư tưởng và lập trường chính trị của Nguyễn Ái Quốc. Trong những năm 1919 $-$ 1921, Bộ trưởng Bộ Thuộc địa Pháp Albert Sarraut nhiều lần gặp Nguyễn Ái Quốc mua chuộc và đe dọa. Ngày 30/6/1923, Nguyễn Ái Quốc tới Liên Xô và làm việc tại Quốc tế Cộng sản ở Moscow, tham gia nhiều hoạt động, đặc biệt là dự và đọc tham luận tại Đại hội V Quốc tế Cộng sản (17/6 $-$ 07/8/1924), làm việc trực tiếp ở Ban Phương Đông của Quốc tế Cộng sản.

Sau khi xác định được con đường cách mạng đúng đắn, Nguyễn Ái Quốc tiếp tục khảo sát, tìm hiểu để hoàn thiện nhận thức về đường lối cách mạng vô sản, đồng thời tích cực truyền bá chủ nghĩa Mác $-$ Lênin về Việt Nam.

\subsubsection{Chuẩn bị về tư tưởng, chính trị và tổ chức cho sự ra đời của Đảng}
\paragraph{Về tư tưởng}
Từ giữa năm 1921 tại Pháp, cùng một số nhà cách mạng của các nước thuộc địa khác, Nguyễn Ái Quốc tham gia thành lập Hội liên hiệp thuộc địa, sau đó sáng lập ra tờ báo \textit{Le Paria} (Người cùng khổ). Người viết nhiều bài trên các báo \textit{Nhân đạo, Đời sống công nhân, Tạp chí Cộng sản, Tập san Thư tín quốc tế}, ...

Năm 1922, \textit{Ban Nghiên cứu thuộc địa} của Đảng Cộng sản Pháp được thành lập, Nguyễn Ái Quốc được cư làm Trưởng Tiểu ban Nghiên cứu về Đông Dương. Vừa nghiên cứu lý luận, vừa tham gia hoạt động thực tiễn trong phong trào cộng sản và công nhân quốc tế, dưới nhiều phương thức phong phú, Nguyễn Ái Quốc tích cực tố cáo, lên án bản chất áp bức, bóc lột, nô dịch của chủ nghĩa thực dân đối với nhân dân các nước thuộc địa và kêu gọi, thức tỉnh nhân dân bị áp bức đấu tranh giải phóng. Người chỉ rõ bản chất của chủ nghĩa thực dân, xác định chủ nghĩa thực dân là kẻ thù chung của các dân tộc thuộc địa, của giai cấp công nhân và nhân dân lao động trên thế giới. Đồng thời, Người tiến hành tuyên truyền tư tưởng về con đường cách mạng vô sản, con đường cách mạng theo \textit{lý luận Mác $-$ Lênin}, xây dựng mối quan hệ gắn bó giữa những người cộng sản và nhân dân lao động Pháp với các nước thuộc địa và phụ thuộc.

Năm 1927, Nguyễn Ái Quốc khẳng định: "Đảng muốn vững phải có chủ nghĩa làm cốt, trong đảng ai cũng phải hiểu, ai cũng phải theo chủ nghĩa ấy" \footfullcite[tr. 289]{HCMtt2}. Đảng mà không có chủ nghĩa cũng giống như người không có trí khôn, tàu không có bàn chỉ nam. Phải truyền bá tư tưởng vô sản, lý luận Mác $-$ Lênin vào phong trào công nhân và phong trào yêu nước Việt Nam.

\paragraph{Về chính trị}
Xuất phát từ thực tiễn cách mạng thế giới và đặc điểm của phong trào giải phóng dân tộc ở các nước thuộc địa, kế thừa và phát triển quan điểm của V. I. Lênin về cách mạng giải phóng dân tộc, Nguyễn Ái Quốc đưa ra những luận điểm quan trọng về  cách mạng giải phóng dân tộc. Người khẳng định rằng, con đường cách mạng của các dân tộc bị áp bức là \textit{giải phóng giai cấp, giải phóng dân tộc}; cả hai cuộc giải phóng này chỉ có thể là sự nghiệp của chủ nghĩa cộng sản. Đường lối chính trị của Đảng cách mạng phải hướng tới giành độc lập cho dân tộc, tự do, hạnh phúc cho đồng bào, hướng tới xây dựng nhà nước mang lại quyền và lợi ích cho nhân dân.

Nguyễn Ái Quốc xác định cách mạng giải phóng dân tộc ở các nước thuộc địa là một bộ phận của cách mạng vô sản thế giới; giữa cách mạng giải phóng dân tộc ở các nước thuộc địa với cách mạng vô sản ở "chính quốc" có mối quan hệ chặt chẽ với nhau, hỗ trợ cho nhau, nhưng cách mạng giải phóng dân tộc ở các nước thuộc địa không phụ thuộc vào cách mạng vô sản ở "chính quốc" mà có thể thành công trước cách mạng vô sản ở "chính quốc", góp phần tích cực thúc đẩy cách mạng vô sản ở "chính quốc".

Đối với các dân tộc thuộc địa, Nguyễn Ái Quốc chỉ rõ: trong nước nông nghiệp lạc hậu, nông dân là lực lượng đông đảo nhất, bị đế quốc, phong kiến áp bức, bóc lột nặng nề, vì vậy phải thu phục và lôi cuốn được nông dân, phải xây dựng khối liên minh công nông làm động lực cách mạng: "công nông là gốc của cách mệnh; còn học trò, nhà buôn nhỏ, điền chủ nhỏ, ... là bầu bạn cách mệnh của công nông" \footfullcite[tr. 288]{HCMtt2}. Do vậy, Người xác định rằng, cách mạng "là việc chung của cả dân chúng chứ không phải là việc của một hai người" \footfullcite[tr. 283]{HCMtt2}.

Về vấn đề Đảng Cộng sản, Nguyễn Ái Quốc khẳng định: "Cách mạng trước hết phải có đảng cách mệnh, để trong thì vận động và tổ chức dân chúng, ngoài thì liên lạc với dân tộc bị áp bức và vô sản giai cấp mọi nơi. Đảng có vững cách mệnh mới thành công, cugx như người cầm lái có vững thuyền mới chạy" \footfullcite[tr. 289]{HCMtt2}. 

Phong trào "vô sản hóa" do Kỳ bộ Bắc Kỳ Hội Việt Nam Cách mạng Thanh niên phát động từ ngày 29/9/1928 đã góp phần truyền bá tư tưởng vô sản, rèn luyện cán bộ và xây dựng phát triển tổ chức của công nhân.

\paragraph{Về tổ chức}
Sau khi lựa chọn con đường cứu nước $-$ con đường cách mạng vô sản $-$ cho dân tộc Việt Nam, Nguyễn Ái Quốc thực hiện "lộ trình" "đi vào quần chúng, thức tỉnh họ, tổ chức họ, đoàn kết họ, đưa họ ra đấu tranh giành tự do độc lập" \footfullcite[tr. 209]{HCMtt1}. Vì vậy, sau một thời gian hoạt động ở Liên Xô để tìm hiểu, khảo sát thực tế về cách mạng vô sản, tháng 11/1924, Người đến Quảng Châu (Trung Quốc) $-$ nơi có đông người Việt Nam yêu nước hoạt động $-$ để xúc tiến các công việc về tổ chức thành lập đảng cộng sản. Tháng 2/1925, Người lựa chọn một số thanh niên tích cực trong \textit{Tâm tâm xã}, lập ra nhóm \textit{Cộng sản đoàn}.

Tháng 6/1925, Nguyễn Ái Quốc thành lập \textit{Hội Việt Nam Cách mạng Thanh niên} tại Quảng Châu (Trung Quốc), nòng cốt là Cộng sản đoàn. Hội đã công bố chương trình, điều lệ của Hội, mục đích: để làm cách mệnh dân tộc (đập tan bọn Pháp và giành độc lập cho xứ sở) rồi sau đó làm cách mạng thế giới (lật đổ chủ nghĩa đế quốc và thực hiện chủ nghĩa cộng sản). Hệ thống tổ chức của Hội gồm 5 cấp: trung ương bộ, kỳ bộ, tỉnh bộ hay thành bộ, huyện bộ và chi bộ. Tổng bộ là cơ quan lãnh đạo cao nhất giữa hai kỳ đại hội. Trụ sở đặt tại Quảng Châu.

Hội đã xuất bản tờ báo \textit{Thanh niên} (do Nguyễn Ái Quốc sáng lập và trực tiếp chỉ đạo), tuyên truyền tôn chỉ, mục đích của Hội, tuyên truyền chủ nghĩa Mác $-$ Lênin và phương hướng phát triển của cuộc vận động giải phóng dân tộc Việt Nam. Báo in bằng tiếng Việt và ra hằng tuần, mỗi số in khoảng 100 bản. Ngày 21/6/1925 ra số đầu tiên, đến tháng 4/1927, báo do Nguyễn Ái Quốc phụ trách và ra được 88 số. Sau khi Nguyễn Ái Quốc rời Quảng Châu (4/1927) đi Liên Xô, những đồng chí khác trong Tổng bộ vẫn tiếp tục việc xuất bản và hoạt động cho đến tháng 2/1930 với 202 số (từ số 89 trở đi, trụ sở báo chuyển về Thượng Hải). Một số lượng lớn báo Thanh niên được bí mật đưa về nước và tới các trung tâm phong trào yêu nước của người Việt Nam ở nước ngoài. Báo \textit{Thanh niên} đánh dấu sự ra đời của báo chí cách mạng Việt Nam.

Sau khi thành lập, Hội tổ chức các lớp huấn luyện chính trị do Nguyễn Ái Quốc trực tiếp phụ trách, phái người về nước vận động, lựa chọn và đưa một số thanh niên tích cực sang Quảng Châu để đào tạo, bồi dưỡng về lý luận chính trị. Từ giữa năm 1925 đến tháng 4/1927, Hội đã tổ chức được trên 10 lớp huấn luyện tại nhà số 13 và 13B đường Văn Minh, Quảng Châu (nay là nhà số 248 và 250). Sau khi được đào tạo, các hội viên được cử về nước xây dựng và phát triển phong trào cách mạng theo khuynh hướng vô sản. Trong số học viên được đào tạo ở Quảng Châu, có nhiều đồng chí được cử đi học trường Đại học Cộng sản Phương Đông (Liên Xô) và trường Quân chính Hoàng Phố (Trung Quốc).

Sau sự biến chính trị ở Quảng Châu (4/1927), Nguyễn Ái Quốc trở lại Moscow và sau đó được Quốc tế Cộng sản cử đi công tác ở nhiều nước châu Âu. Năm 1928, Người trở về châu Á và hoạt động ở Xiêm (tức Thái Lan).

Các bài giảng của Nguyễn Ái Quốc trong các lớp đào tạo, bồi dưỡng cho những người Việt Nam yêu nước tại Quảng Châu, được \textit{Hội Liên hiệp các dân tộc bị áp bức ở Á Đông} xuất bản thành cuốn \textit{Đường cách mệnh}. Đây là cuốn sách chính trị đầu tiên của cách mạng Việt Nam, trong đó tầm quan trọng của lý luận cách mạng được đặt ở vị trí hàng đầu đối với cuộc vận động cách mạng và đối với đảng cách mạng tiên phong. \textit{Đường cách mệnh} xác định rõ con đường, mục tiêu, lực lượng và phương pháp đấu tranh của cách mạng. Tác phẩm thể hiện tư tưởng nổi bật của lãnh tụ Nguyễn Ái Quốc dựa trên cơ sở vận dụng sáng tạo chủ nghĩa Mác $-$ Lênin vào đặc điểm của Việt Nam. Những điều kiện về tư tưởng, lý luận, chính trị và tổ chức để thành lập Đảng đã được thể hiện rõ trong tác phẩm.

Ở trong nước, từ đầu năm 1926, Hội Việt Nam Cách mạng Thanh niên đã bắt đầu phát triển cơ sở ở trong nước, đến đầu năm 1927, các kỳ bộ được thành lập. Hội còn chú trọng xây dựng cơ sở trong Việt kiều ở Xiêm (Thái Lan). Hội Việt Nam Cách mạng Thanh niên chưa phải là chính đảng cộng sản, nhưng chương trình hành động đã thể hiện quan điểm, lập trường của giai cấp công nhân, là tổ chức tiền thân dẫn tới ra đời các tổ chức cộng sản ở Việt Nam. Hội là tổ chức trực tiếp truyền bá chủ nghĩa Mác $-$ Lênin vào Việt Nam và cũng là sự chuẩn bị quan trọng về tổ chức để tiến tới thành lập chính đảng của giai cấp công nhân ở Việt Nam. Những hoạt động của Hội có ảnh hưởng và thúc đẩy mạnh mẽ sự chuyển biến của phong trào công nhân, phong trào yêu nước Việt Nam những năm 1928 $-$ 1929 theo xu hướng cách mạng vô sản. Đó là tổ chức tiền thân của Đảng Cộng sản Việt Nam.