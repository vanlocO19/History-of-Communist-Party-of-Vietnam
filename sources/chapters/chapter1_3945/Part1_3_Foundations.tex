\subsection{Thành lập Đảng Cộng sản Việt Nam và Cương lĩnh chính trị đầu tiên của Đảng}
\subsubsection{Các tổ chức cộng sản ra đời}
Với sự nỗ lực cố gắng truyền bá chủ nghĩa Mác $-$ Lênin vào phong trào công nhân và phong trào yêu nước Việt Nam của Nguyễn Ái Quốc và những hoạt động tích cực của các cấp bộ trong tổ chức Hội Việt Nam Cách mạng Thanh niên trên cả nước đã có tác dụng thúc đẩy phong trào yêu nước Việt Nam theo khuynh hướng cách mạng vô sản, nâng cao ý thức giác ngộ và lập trường cách mạng của giai cấp công nhân. Những cuộc đấu tranh của thợ thuyền khắp ba kỳ với nhịp độ, quy mô ngày càng lớn, nội dung chính trị ngày càng sâu sắc. Số lượng các cuộc đấu tranh của công nhân những năm 1928 $-$ 1929 tăng gấp 2.5 lần so với 2 năm 1926 $-$ 1927.

Đến năm 1929, trước sự phát triển mạnh mẽ của phong trào cách mạng Việt Nam, tổ chức Hội Việt Nam Cách mạng Thanh niên không còn thích hợp và đủ sức lãnh đạo phong trào. Trước tình hình đó, tháng 3/1929, những người lãnh đạo Kỳ bộ Bắc Kỳ (Trần Văn Cung, Ngô Gia Tự, Nguyễn Đức Cảnh, Trịnh Đình Cửu, ...) họp tại số nhà 5D, phố Hàm Long, Hà Nội, quyết định thành lập Chi bộ Cộng sản đầu tiên ở Việt Nam. Ngày 17/6/1929, đại biểu của các tổ chức cộng sản ở Bắc Kỳ họp tại số nhà 312 phố Khâm Thiên (Hà Nội), quyết định thành lập \textit{Đông Dương Cộng sản Đảng}, thông qua Tuyên ngôn, Điều lệ; lấy cờ đỏ búa liềm là Đảng kỳ và quyết định xuất bản báo \textit{Búa liềm} làm cơ quan ngôn luận.

Trước ảnh hưởng của \textit{Đông Dương Cộng sản Đảng}, những thanh niên yêu nước ở Nam Kỳ theo xu hướng cộng sản, lần lượt tổ chức những chi bộ cộng sản. Tháng 11/1929, trên cơ sở các chi bộ cộng sản ở Nam Kỳ, \textit{An Nam Cộng sản Đảng} được thành lập tại Khánh Hội, Sài Gòn, công bố Điều lệ, quyết định xuất bản \textit{Tạp chí Bônsơvích}.

Tại Trung Kỳ, Tân Việt Cách mạng Đảng (là một tổ chức thanh niên yêu nước có cả Trẩn Phú, Nguyễn Thị Minh Khai, ...) chịu tác động mạnh mẽ của Hội Việt Nam Cách mạng Thanh niên $-$ đã đi theo khuynh hướng cách mạng vô sản. Tháng 9/1929, những người tiên tiến trong Tân Việt Cách mạng Đảng họp bàn việc thành lập \textit{Đông Dương Cộng sản Liên đoàn} và ra Tuyên đạt, khẳng định: "... những người giác ngộ cộng sản chân chính trong Tân Việt Cách mạng Đảng trịnh trọng tuyên ngôn cùng toàn thể đảng viên Tân Việt Cách mạng Đảng, toàn thể thợ thuyền dân cày và lao khổ biết rằng chúng tôi đã chánh thức lập ra \textit{Đông Dương Cộng sản Liên đoàn}... Muốn làm tròn nhiệm vụ thì trước mắt Đông Dương Cộng sản Liên đoàn là một mặt phải xây dựng cơ sở chi bộ của Liên đoàn tức là thực hành cải tổ Tân Việt Cách mạng Đảng thành đoàn thể cách mạng chân chính..." \footfullcite[tr. 404]{VKDtt1}. Đến cuối tháng 12/1929, tại Đại hội các đại biểu liên tỉnh tại nhà đồng chí Nguyễn Xuân Thanh $-$ Ủy viên Ban Chấp hành liên tỉnh (ga Chợ Thượng, huyện Đức Thọ, tỉnh Hà Tĩnh), nhất trí quyết định "Bỏ tên gọi Tân Việt. Đặt tên mới là Đông Dương Cộng sản Liên đoàn". Khi đang Đại hội, sợ bị lộ, các đại biểu di chuyển đến địa điểm mới thì bị địch bắt vào sáng ngày 01/1/1930. "Có thể coi những ngày cuối tháng 12/1929 là thời điểm hoàn tất quá trình thành lập Đông Dương Cộng sản Liên đoàn được khởi đầu từ sự kiện công bố Tuyên đạt tháng 9/1929" \footfullcite[tr. 404]{LSBN1}.

Sự ra đời ba tổ chức cộng sản trên cả nước diễn ra trong vòng nửa cuối năm 1929 đã khẳng định bước phát triển về chất của phong trào yêu nước Việt Nam theo khuynh hướng cách mạng vô sản, phù hợp với xu thế và nhu cầu bức thiết của lịch sử Việt Nam. Tuy nhiên, sự ra đời của ba tổ chức cộng sản ở ba miền đều tuyên bố ủng hộ Quốc tế Cộng sản, kêu gọi Quốc tế Cộng sản thừa nhận tổ chức của mình và đều tự nhận là đảng cách mạng chân chính, không tránh khỏi phân tán về lực lượng và thiếu thống nhất về tổ chức trên cả nước.

Sự chuyển biến mạnh mẽ các phong trào đấu tranh của các tầng lớp nhân dân ngày càng lên cao, nhu cầu thành lập một chính đảng cách mạng có đủ khả năng tập hợp lực lượng toàn dân tộc và đảm nhiệm vai trò lãnh đạo sự nghiệp giải phóng dân tộc ngày càng trở nên bức thiết đối với cách mạng Việt Nam lúc bấy giờ.

\subsubsection{Hội nghị thành lập Đảng Cộng sản Việt Nam}
Trước nhu cầu cấp bách của phong trào cách mạng trong nước, với tư cách là phái viên của Quốc tế Cộng sản, ngày 23/12/1929, Nguyễn Ái Quốc đến Hong Kong (Trung Quốc) triệu tập đại biểu của Đông Dương Cộng sản Đảng và An Nam Cộng sản Đảng đến họp tại Cửu Long (Hong Kong), tiến hành Hội nghị hợp nhất các tổ chức cộng sản thành một chính đảng duy nhất của Việt Nam.

Thời gain Hội nghị từ ngày 06/1 đến ngày 07/2/1930. Sau này Đảng quyết nghị lấy ngày 03 tháng 2 dương lịch làm ngày kỷ niệm thành lập Đảng \footnote{Sau này, đến Đại hội đại biểu toàn quốc lần thứ III của Đảng (9/1960) quyết nghị: "... từ nay sẽ lấy ngày 3 tháng 2 dương lịch mỗi năm làm ngày kỷ niệm thành lập Đảng".}. Trong \textit{Báo cáo gửi Quốc tế Cộng sản}, ngày 18/2/1930, Nguyễn Ái Quốc viết: "Chúng tôi họp vào ngày mồng 6-1. Với tư cách là phái viên của Quốc tế Cộng sản có đầy đủ quyền quyết định mọi vấn đề liên quan đến phong trào cách mạng ở Đông Dương, tôi nói cho họ biết những sai lầm và họ phải làm gì. Họ đồng ý thống nhất vào một đảng. Chúng tôi cùng nhau xác định cương lĩnh và chiến lược theo đường lối của Quốc tế Cộng sản... Các đại biểu trở về An Nam ngày 8-2" \footfullcite[tr. 19-20]{VKDtt2}.

Thành phần dự Hội nghị: gồm 2 đại biểu của Đông Dương Cộng sản Đảng (Trịnh Đình Cửu và Nguyễn Đức Cảnh), 2 đại biểu của An Nam Cộng sản Đảng (Châu Văn Liêm và Nguyễn Thiệu), dưới sự chủ trì của lãnh tụ Nguyễn Ái Quốc $-$ đại biểu của Quốc tế Cộng sản.

Chương trình nghị sự của Hội nghị:
\begin{enumerate}
\item Đại biểu của Quốc tế Cộng sản nói lý do cuộc hội nghị;
\item Thảo luận ý kiến của đại biểu Quốc tế Cộng sản về:
\begin{enumerate}
\item Việc hợp nhất tất cả các nhóm cộng sản thành một tổ chức chung, tổ chức này sẽ là một Đảng Cộng sản chân chính;
\item Kế hoạch thành lập tổ chức đó.
\end{enumerate}
\end{enumerate}

Lãnh tụ Nguyễn Ái Quốc nêu ra năm điểm lớn cần thảo luận và thống nhất:
"\begin{enumerate}
\item Bỏ mọi thành kiến xung đột cũ, thành thật hợp tác để thống nhất các nhóm cộng sản Đông Dương;
\item Định tên Đảng là Đảng Cộng sản Việt Nam;
\item Thảo Chính cương và Điều lệ sơ lược;
\item Định kế hoạch thực hiện việc thống nhất;
\item Cử một Ban Trung ương lâm thời..." \footfullcite[tr.2]{VKDtt2}.
\end{enumerate}

Hội nghị thảo luận, tán thành ý kiến chỉ đạo của Nguyễn Ái Quốc, thông qua các văn kiện quan trọng do lãnh tụ Nguyễn Ái Quốc soạn thảo: \textit{Chánh cương vắn tắt của Đảng, Sách lược vắn tắt của Đảng, Chương trình tóm tắt của Đảng, Điều lệ vắn tắt của Đảng Cộng sản Việt Nam}.

Hội nghị xác định rõ tôn chỉ mục đích của Đảng: "Đảng Cộng sản Việt Nam tổ chức ra để lãnh đạo quần chúng lao khổ làm giai cấp tranh đấu để tiêu trừ tư bản đế quốc chủ nghĩa, làm cho thực hiện xã hội cộng sản". Quy trình điều kiện vào Đảng: là những người "tin theo chủ nghĩa cộng sản, chương trình đảng và Quốc tế Cộng sản, hăng hái tranh đấu và dám hi sinh phục tùng mệnh lệnh Đảng và đóng kinh phí, chịu phấn đấu trong một bộ phận đảng" \footfullcite[tr. 7-8]{VKDtt2}.

Hội nghị chủ trương các đại biểu về nước phải cử một Trung ương lâm thời để lãnh đạo cách mạng Việt Nam. Hệ thống tổ chức Đảng từ chi bộ, huyện bộ, thị bộ hay khu bộ, tỉnh bộ, thành bộ hay đặc biệt bộ và Trung ương.

Ngoài ra, Hội nghị còn quyết địn chủ trương xây dựng các tổ chức công hội, nông hội, cứu tế, tổ chức phản đế và xuất bản một tạp chí lý luận và ba tờ báo tuyên truyền của Đảng.

Đến ngày 24/2/1930, việc thống nhất các tổ chức cộng sản thành một chính đảng duy nhất được hoàn thành với Quyết nghị của Lâm thời chấp ủy Đảng Cộng sản Việt Nam, chấp thuận Đông Dương Cộng sản Liên đoàn gia nhập Đảng Cộng sản Việt Nam.

Hội nghị thành lập Đảng Cộng sản Việt Nam dưới sự chủ trì của lãnh tụ Nguyễn Ái Quốc có giá trị như một Đại hội Đảng. Sau Hội nghị, Nguyễn Ái Quốc ra \textit{Lời kêu gọi} nhân dịp thành lập Đảng. Mở đầu \textit{Lời kêu gọi}, Người viết: "Nhận chỉ thị của Quốc tế Cộng sản giải quyết vấn đề cách mạng nước ta, tôi đã hoàn thành nhiệm vụ".

\subsubsection{Nội dung cơ bản của Cương lĩnh chính trị đầu tiên của Đảng}
Trong các văn kiện do lãnh tụ Nguyễn Ái Quốc soạn thảo, được thông qua tại Hội nghị thành lập Đảng, có hai văn kiện, đó là: \textit{Chánh cương vắn tắt của Đảng} và \textit{Sách lược vắn tắt của Đảng} \footfullcite[tr. 2-5]{VKDtt2} đã phản ánh về đường hướng phát triển và những vấn đề cơ bản về chiến lược và sách lược của cách mạng Việt Nam. Vì vậy, hai văn kiện trên là Cương lĩnh chính trị đầu tiên của Đảng Cộng sản Việt Nam \footnote{Theo Thông báo Kết luận của Bộ Chính trị số 31-TB/TW, ngày 01/6/2011, về một số vấn đề trong bản thảo \textit{Lịch sử Đảng Cộng sản Việt Nam tập I (1930 $-$ 1954)}}.

Cương lĩnh chính trị đầu tiên xác định mục tiêu chiến lược của cách mạng Việt Nam: Từ việc phân tích thực trạng và mâu thuẫn trong xã hội Việt Nam $-$ một xã hội thuộc địa nửa phong kiến, mâu thuẫn giữa dân tộc Việt Nam trong đó có công nhân, nông dân với đế quốc ngày càng gay gắt cần phải giải quyết, đi đến xác định đường lối chiến lược của cách mạng Việt Nam "chủ trương làm tư sản dân quyền cách mạng và thổ địa cách mạng để đi tới xã hội cộng sản". Như vậy, mục tiêu chiến lược được nêu ra trong Cương lĩnh đầu tiên của Đảng đã làm rõ nội dung của cách mạng thuộc địa nằm trong phạm trù của cách mạng vô sản.

Xác định nhiệm vụ chủ yếu trước mắt của cách mạng Việt Nam: "Đánh đổ đế quốc chủ nghĩa Pháp và bọn phong kiến", "Làm cho nước Nam được hoàn toàn độc lập". Cương lĩnh đã xác định: Chống đế quốc và chống phong kiến là nhiệm vụ cơ bản để giành độc lập cho dân tộc và ruộng đất cho dân cày, trong đó chống đế quốc, giành độc lập cho dân tộc được đặt ở vị trí hàng đầu.

Về phương diện xã hội, Cương lĩnh xác định rõ: "a) Dân chúng được tự do tổ chức. b) Nam nữ bình quyền, v.v... c) Phổ thông giáo dục theo công nông hóa". Về phương diện kinh tế, Cương lĩnh xác định: Thủ tiêu hết các thứ quốc trái; thâu hết sản nghiệp lớn (như công nghiệp, vận tải, ngân hàng, v.v.) của tư bản đế quốc chủ nghĩa Pháp để giao cho Chính phủ công nông binh quản lý; thâu hết ruộng đất của đế quốc chủ nghĩa làm của công chia cho dân cày nghèo; bỏ sưu thuế cho dân cày nghèo; mở mang công nghiệp và nông nghiệp; thi hành luật ngày làm tám giờ ... Những nhiệm vụ của cách mạng Việt Nam về phương diện xã hội và phương diện kinh tế nêu trên vừa phản ánh đúng tình hình kinh tế, xã hội, cần được giải quyết ở Việt Nam, vừa thể hiện tính cách mạng, toàn diện, triệt để là xóa bỏ tận gốc ách thống trị, bóc lột hà khắc của ngoại bang, nhằm giải phóng dân tộc, giải phóng giai cấp, giải phóng xã hội, đặc biệt là giải phóng cho hai giai cấp công nhân và nông dân.

Xác định lực lượng cách mạng: phải đoàn kết công nhân, nông dân $-$ đây là lực lượng cơ bản, trong đó có giai cấp công nhân lãnh đạo; đồng thời chủ trương đoàn kết tất cả các giai cấp, các lực lượng tiến bộ, yêu nước để tập trung chống đế quốc và tay sai. Do vậy, Đảng "phải thu phục được đại bộ phận giai cáp mình", "phải thu phục cho được đại bộ phận dân cày, hết sức liên lạc với tiểu tư sản, trí thức, trung nông, ... để kéo họ đi vòa phe vô sản giai cấp. Còn đối với bọn phú nông, trung, tiểu địa chủ và tư bản An Nam mà chưa rõ mặt phản cách mạng thì phải lợi dụng, ít lâu mới làm cho họ đứng trung lập". Đây là cơ sở của tư tưởng chiến lược đại đoàn kết toàn dân tộc, xây dựng khối đại đoàn kết rộng rãi các giai cấp, các tầng lớp nhân dân yêu nước và các tổ chức yêu nước, cách mạng, trên cơ sở đánh giá đúng đắn thái độ các giai cấp phù hợp với đặc điểm xã hội Việt Nam.

Xác định phương pháp tiến hành cách mạng giải phóng dân tộc, Cương lĩnh khẳng định phải bằng con đường bạo lực cách mạng của quần chúng, trong bất cứ hoàn cảnh nào cũng không được thỏa hiệp "không khi nào nhượng bộ một chút lợi ích gì của công nông mà đi vào đường thỏa hiệp". Có sách lược đấu tranh cách mạng thích hợp để lôi kéo tiểu tư sản, trí thức, trung nông về phía giai cấp vô sản, nhưng kiên quyết: "bộ phận nào đã ra mặt phản cách mạng (Đảng Lập hiến, v.v.) thì phải đánh đổ".

Xác định tinh thần đoàn kết quốc tế, Cương lĩnh chỉ rõ trong khi thực hiện nhiệm vụ giải phóng dân tộc, đồng thời tranh thủ sự đoàn kết, ủng hộ của các dân tộc bị áp bức và giai cấp vô sản thế giới, nhất là giai cấp vô sản Pháp. Cương lĩnh nêu rõ cách mạng Việt Nam liên lạc mật thiết và là một bộ phận của cách mạng vô sản thế giới: "trong khi tuyên truyền cái khẩu hiệu nước An Nam độc lập, phải đồng thời tuyên truyền và thực hành liên lạc với bị áp bức dân tộc và vô sản giai cấp thế giới". Như vậy, ngay từ khi thành lập, Đảng Cộng sản Việt Nam đã nêu cao chủ nghĩa quốc tế và mang bản chất quốc tế của giai cấp công nhân.

Xác định vai trò lãnh đạo của Đảng: "Đảng là đội tiên phong của vô sản giai cấp, phải thu phục cho được đại bộ phận giai cấp mình, phải làm cho giai cấp mình lãnh đạo được dân chúng" \footfullcite[tr. 4]{VKDtt2}. "Đảng là đội tiên phong của đạo quân vô sản gồm một số lớn của giai cấp công nhân và làm cho họ có đủ năng lực lãnh đạo quần chúng" \footfullcite[tr. 6]{VKDtt2}.

Cương lĩnh chính trị đầu tiên của Đảng đã phản ánh một cách súc tích các luận điểm cơ bản của cách mạng Việt Nam. Trong đó, thể hiện bản lĩnh chính trị độc lập, tự chủ, sáng tạo trong việc đánh giá đặc điểm, tính chất xã hội thuộc địa nửa phong kiến Việt Nam trong những năm 20 của thế kỷ XX, chỉ rõ những mâu thuẫn cơ bản và chủ yếu của dân tộc Việt Nam lúc đó, đặc biệt là việc đánh giá đúng đắn, sát thực thái độ các giai tầng xã hội đối với nhiệm vụ giải phóng dân tộc. Từ đó, các văn kiện đã xác định đường lối chiến lược và sách lược của cách mạng Việt Nam, đồng thời xác định phương pháp cách mạng, nhiệm vụ cách mạng và lực lượng của cách mạng để thực hiện đường lối chiến lược và sách lược đã đề ra.

Như vậy, trước yêu cầu của lịch sử cách mạng Việt Nam cần phải thống nhất các tổ chức cộng sản trong nước, chấm dứt sự chia rẽ bất lợi cho cách mạng, với uy tín chính trị và phương thức hợp nhất phù hợp, Nguyễn Ái Quốc đã kịp thời triệu tập và chủ trì hợp nhất các tổ chức cộng sản. Những văn kiện được thông qua trong Hội nghị hợp nhất dù "vắn tắt", nhưng đã phản ánh những vấn đề cơ bản trước mắt và lâu dài cho cách mạng Việt Nam, đưa cách mạng Việt Nam sang một trang sử mới.