\subsection{Ý nghĩa lịch sử của việc thành lập Đảng Cộng sản Việt Nam}
Đảng Cộng sản Việt Nam ra đời đã chấm dứt sự khủng hoảng, bế tắc về đường lối cứu nước, đưa cách mạng Việt Nam sang một bước ngoặt lịch sử vĩ đại: cách mạng Việt Nam trở thành một bộ phận khăng khít của cách mạng vô sản thế giới. Đó là kết quả của sự vận động phát triển và thống nhất của phong trào cách mạng trong cả nước, sự chuẩn bị tích cực, sáng tạo, bản lĩnh của lãnh tụ Nguyễn Ái Quốc, sự đoàn kết, nhất trí của những chiến sĩ cách mạng tiên phong vì lợi ích của giai cấp và dân tộc. 

Sự ra đời của Đảng Cộng sản Việt Nam là sản phẩm của sự kết hợp chủ nghĩa Mác $-$ Lênin, tư tưởng Hồ Chí Minh với phong trào công nhân và phong trào yêu nước Việt Nam. Đó cũng là kết quả của sự phát triển cao và thống nhất của phong trào công nhân và phong trào yêu nước Việt Nam được soi sáng bởi chủ nghĩa Mác $-$ Lênin. Chủ tịch Hồ Chí Minh đã khẳng định: \textit{chủ nghĩa Mác $-$ Lênin kết hợp với phong trào công nhân và phong trào yêu nước đã dẫn tới việc thành lập Đảng}, "Việc thành lập Đảng là một bước ngoặt vô cùng quan trọng trong lịch sử cách mạng Việt Nam ta. Nó chứng tỏ rằng giai cấp vô sản ta đã trưởng thành và đủ sức lãnh đạo cách mạng" \footfullcite[tr. 406]{HCMtt12}.

Đảng Cộng sản Việt Nam ra đời với Cương lĩnh chính trị đầu tiên được thông qua tại Hội nghị thành lập Đảng đã khẳng định lần đầu tiên cách mạng Việt Nam có một bản cương lĩnh chính trị phản ánh được quy luật khách quan của xã hội Việt Nam, đáp ứng những nhu cầu cơ bản và cấp bách của xã hội Việt Nam, phù hợp với xu thế của thời đại, định hướng chiến lược đúng đắn cho tiến trình phát triển của cách mạng Việt Nam. Đường lối đó là kết quả của sự vận dụng chủ nghĩa Mác $-$ Lênin vào thực tiễn cách mạng Việt Nam một cách đúng đắn, sáng tạo và có phát triển trong điều kiện lịch sử mới.

Sự ra đời của Đảng Cộng sản Việt Nam với Cương lĩnh chính trị đầu tiên đã khẳng định sự lựa chọn con đường cách mạng cho dân tộc Việt Nam $-$ con đường cách mạng vô sản. Con đường duy nhất đúng giải phóng dân tộc, giải phóng giai cấp và giải phóng con người. Sự lựa chọn con đường cách mạng vô sản phù hợp với nội dung và xu thế của thời đại được mở ra từ Cách mạng tháng Mười Nga vĩ đại: "Đối với nước ta, không còn con đường nào khác để có độc lập dân tộc thật sự và tự do, hạnh phúc cho nhân dân. Cần nhấn mạnh rằng đây là sự lựa chọn của chính lịch sử, sự lựa chọn đã dứt khoát từ năm 1930 với sự ra đời của Đảng ta" \footfullcite[tr. 13-14]{VKDtt51}.

Đảng Cộng sản Việt Nam ra đời là bước ngoặt vĩ đại trong lịch sử phát triển của dân tộc Việt Nam, trở thành nhân tố hàng đầu quyết định đưa cách mạng Việt Nam đi từ thắng lợi này đến thắng lợi khác.