\chapter{Đảng lãnh đạo hai cuộc kháng chiến, hoàn thành giải phóng dân tộc, thống nhất đất nước (1945 $-$ 1975)}
\section*{Mục tiêu}
%\subsection*{Về kiến thức}
%\subsection*{Về tư tưởng}
%\subsection*{Về kỹ năng}

\section{Lãnh đạo xây dựng, bảo vệ chính quyền cách mạng, kháng chiến chống thực dân Pháp xâm lược 1945 $-$ 1954}

%\subsection{Xây dựng và bảo vệ chính quyền cách mạng 1945 $-$ 1946}
%\subsubsection{Tình hình Việt Nam sau Cách mạng tháng Tám}
%\subsubsection{Xây dựng chế độ mới và chính quyền cách mạng}
%\subsubsection{Tổ chức cuộc kháng chiến chống thực dân Pháp xâm lược ở Nam Bộ, đấu tranh bảo vệ chính quyền cách mạng non trẻ}

%\subsection{Đường lối kháng chiến toàn quốc và quá trình tổ chức thực hiện từ năm 1946 đến năm 1950}
%\subsubsection{Cuộc kháng chiến toàn quốc bùng nổ và đường lối kháng chiến của Đảng}
%\subsubsection{Tổ chức, chỉ đạo cuộc kháng chiến từ năm 1947 đến năm 1950}

%\subsection{Đẩy mạnh cuộc kháng chiến đến thắng lợi 1951 $-$ 1954}
%\subsubsection{Đại hội đại biểu toàn quốc lần thứ II và Chính cương của Đảng (2/1951)}
%\subsubsection{Đẩy mạnh phát triển cuộc kháng chiến về mọi mặt}
%\subsubsection{Kết hợp đấu tranh quân sự và ngoại giao, kết thúc thắng lợi cuộc kháng chiến}

%\subsection{Ý nghĩa lịch sử và kinh nghiệm của Đảng trong lãnh đạo kháng chiến chống Pháp và can thiệp Mỹ}
%\subsubsection{Ý nghĩa thắng lợi của cuộc kháng chiến}
%\subsubsection{Kinh nghiệm của Đảng về lãnh đạo kháng chiến}

\section{Lãnh đạo xây dựng chủ nghĩa xã hội ở miền Bắc và kháng chiến chống đế quốc Mỹ xâm lược, giải phóng miền Nam, thống nhất đất nước (1954 $-$ 1975)}

%\subsection{Sự lãnh đạo của Đảng đối với cách mạng hai miền Nam Bắc 1954 $-$ 1965}
%\subsubsection{Khôi phục kinh tế, cải tạo xã hội chủ nghĩa ở miền Bắc, chuyển cách mạng miền Nam từ thế giữ gìn lực lượng sang thế tiến công 1954 $-$ 1960}
%\subsubsection{Xây dựng chủ nghĩa xã hội ở miền Bắc, phát triển thế tiến công của cách mạng miền Nam 1961 $-$ 1965}

%\subsection{Lãnh đạo cách mạng cả nước 1965 $-$ 1975}
%\subsubsection{Đường lối kháng chiến chống Mỹ cứu nước của Đảng}
%\subsubsection{Xây dựng hậu phương, chống chiến tranh phá hoại của đế quốc Mỹ ở miền Bắc; giữ vững thế chiến lược tiến công, đánh bại chiến lược Chiến tranh cục bộ của đế quốc Mỹ 1965 $-$ 1968}
%\subsubsection{Khôi phục kinh tế, bảo vệ miền Bắc, đẩy mạnh cuộc chiến đấu giải phóng miền Nam, thống nhất Tổ quốc (1969 $-$ 1975)}

%\subsection{Ý nghĩa lịch sử và kinh nghiệm lãnh đạo của Đảng thời kỳ 1954 $-$ 1975}
%\subsubsection{Ý nghĩa}
%\subsubsection{Kinh nghiệm}