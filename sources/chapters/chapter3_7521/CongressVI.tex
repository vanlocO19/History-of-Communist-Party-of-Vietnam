\subsection{Đổi mới toàn diện, đưa đất nước ra khỏi khủng hoảng kinh tế $-$ xã hội 1986 $-$ 1996}
\subsubsection{Đại hội đại biểu toàn quốc lần thứ VI và thực hiện đường lối đổi mới toàn diện}
Đại hội VI của Đảng diễn ra tại Hà Nội, từ ngày 15 đến ngày 18/12/1986, trong bối cảnh cuộc cách mạng khoa học $-$ kỹ thuật đang phát triển mạnh, xu thế đối thoại trên thế giới đang dần thay thế xu thế đối đầu. Đổi mới đã trở thành xu thế của thời đại. Liên Xô và các nước xã hội chủ nghĩa đều tiến hành cải tổ sự nghiệp xây dựng chủ nghĩa xã hội. Việt Nam đang bị các đế quốc và thế lực thù địch bao vây, cấm vận và ở tình trạng khủng hoảng kinh tế $-$ xã hội. Lương thực, thực phẩm, hàng tiêu dùng đều khan hiếm; lạm phát tăng $300\%$ năm 1985 lên $774\%$ năm 1986. Các hiện tượng tiêu cực, vi phạm pháp luật, vượt biên trái phép diễn ra khá phổ biến. Đổi mới đã trở thành đòi hỏi bức thiết của tình hình đất nước.

Dự Đại hội có 1129 đại biểu thay mặt cho gần 2 triệu đảng viên cả nước và có 32 đoàn đại biểu quốc tế đến dự. Đại hội đã thông qua các văn kiện chính trị quan trọng, khởi xướng đường lối đổi mới toàn diện, bầu Ban Chấp hành Trung ương gồm 124 ủy viên chính thức, bầu Bộ Chính trị gồm 13 ủy viên chính thức, bầu đồng chí Nguyễn Văn Linh làm Tổng Bí thư của Đảng.

Đường lối đổi mới toàn diện do Đại hội VI đề ra thể hiện trên các lĩnh vực nổi bật:

Đại hội đã nhìn thẳng vào sự thật, đánh giá đúng sự thật, nói rõ sự thật, đánh giá thành tựu, nghiêm túc kiểm điểm, chỉ rõ những sai lầm, khuyết điểm của Đảng trong thời kỳ 1975 $-$ 1986. Đó là những sai lầm nghiêm trọng và kéo dài về chủ trương, chính sách lớn, sai lầm về chỉ đạo chiến lược và tổ chức thực hiện. Khuynh hướng tư tưởng chủ yếu của những sai lầm, khuyết điểm đó, đặc biệt là trên lĩnh vực kinh tế là bệnh chủ quan duy ý chí, lối suy nghĩ và hành động giản đơn, nóng vội, chạy theo nguyện vọng chủ quan. Đó là tư tưởng tiểu tư sản, vừa "tả" khuynh vừa hữu khuynh. Nguyên nhân của mọi nguyên nhân bắt nguồn từ những khuyết điểm trong hoạt động tư tưởng, tổ chức và công tác cán bộ của Đảng. Đại hội rút ra bốn bài học quý báu: \textit{Một là}, trong toàn bộ hoạt động của mình, Đảng phải quán triệt tư tưởng "lấy dân làm gốc". \textit{Hai là}, Đảng phải luôn luôn xuất phát từ thực tế, tôn trọng và hành động theo quy luật khách quan. \textit{Ba là}, phải biết kết hợp sức mạnh dân tộc với sức mạnh thời đại trong điều kiện mới. \textit{Bốn là}, chăm lo xây dựng Đảng ngang tầm với một đảng cầm quyền lãnh đạo nhân dân tiến hành cách mạng xã hội chủ nghĩa.

Thực hiện nhất quán chính sách phát triển nhiều thành phần kinh tế. Đổi mới cơ chế quản lý, xóa bỏ cơ chế tập trung quan liêu, hành chính, bao cấp chuyển sang hạch toán, kinh doanh, kết hợp kế hoạch với thị trường. Nhiệm vụ bao trùm, mục tiêu tổng quát trong những năm còn lại của chặng đường đầu tiên là: Sản xuất đủ tiêu dùng và có tích lũy; bước đầu tạo ra một cơ cấu kinh tế hợp lý, trong đó đặc biệt chú trọng ba chương trình kinh tế lớn là lương thực $-$ thực phẩm, hàng tiêu dùng và hàng xuất khẩu, coi đó là sự cụ thể hóa nội dung công nghiệp hóa trong chặng đường đầu của thời kỳ quá độ. Thực hiện cải tạo xã hội chủ nghĩa thường xuyên với hình thức, bước đi thích hợp, làm cho quan hệ sản xuất phù hợp và lực lượng sản xuất phát triển. Đổi mới cơ chế quản lý kinh tế, giải quyết cho được những vấn đề cấp bách về phân phối, lưu thông. Xây dựng và tổ chức thực hiện một cách thiết thực, có hiệu quả các chính sách xã hội. Bảo đảm nhu cầu củng cố quốc phòng và an ninh. Năm phương hướng lớn phát triển kinh tế là: Bố trí lại cơ cấu sản xuất; điều chỉnh cơ cấu đầu tư xây dựng và củng cố quan hệ sản xuất xã hội chủ nghĩa; sử dụng và cải tạo đúng đắn các thành phần kinh tế; mở rộng và nâng cao hiệu quả kinh tế đối ngoại. Đại hội VI nhấn mạnh: "Tư tưởng chỉ đạo của kế hoạch và các chính sách kinh tế là giải phóng mọi năng lực sản xuất hiện có, khai thác mọi khả năng tiềm tàng của đất nước và sử dụng có hiệu quả sự giúp đỡ quốc tế để phát triển mạnh mẽ lực lượng sản xuất đi đôi với xây dựng và củng cố quan hệ sản xuất xã hội chủ nghĩa" \footfullcite[tr. 380]{VKDtt47}.

Đại hội khẳng định, chính sách xã hội bao trùm mọi mặt của cuộc sống con người, cần có chính sách cơ bản, lâu dài, xác định được những nhiệm vụ, phù hợp với yêu cầu, khả năng trong chặng đường đầu tiên. Bốn nhóm chính sách xã hội là: Kế hoạch hóa dân số, giải quyết việc làm cho người lao động. Thực hiện công bằng xã hội, bảo đảm an toàn xã hội, khôi phục trật tự, kỷ cương trong mọi lĩnh vực xã hội. Chăm lo đáp ứng nhu cầu giáo dục, văn hóa, bảo vệ và tăng cường sức khỏe của nhân dân. Xây dựng chính sách bảo trợ xã hội.

Đề cao cảnh giác, tăng cường khả năng quốc phòng và an ninh của đất nước, quyết đánh thắng kiểu chiến tranh phá hoại nhiều mặt của địch, bảo đảm chủ động trong mọi tình huống để bảo vệ Tổ quốc.

Đối ngoại góp phần quan trọng vào cuộc đấu tranh của nhân dân thế giới vì hòa bình, độc lập dân tộc, dân chủ và chủ nghĩa xã hội, tăng cường tình hữu nghị và hợp tác toàn diện với Liên Xô và các nước xã hội chủ nghĩa; bình thường hóa quan hệ với Trung Quốc vì lợi ích của nhân dân hai nước, vì hòa bình ở Đông Nam Á và trên thế giới. Kết hợp sức mạnh của dân tộc với sức mạnh thời đại, phấn đấu giữu vững hòa bình ở Đông Dương, Đông Nam Á và trên thế giới, tăng cường quan hệ đặc biệt giữa ba nước Đông Dương, quan hệ hữu nghị và hợp tác với Liên Xô và các nước trong cộng đồng xã hội chủ nghĩa.

Đổi mới sự lãnh đạo của Đảng cần phải đổi mới tư duy, trước hết là tư duy kinh tế, đổi mới công tác tư tưởng; đổi mới công tác cán bộ và phong cách làm việc, giữ vững các nguyên tắc tổ chức và sinh hoạt Đảng; tăng cường đoàn kết nhất trí trong Đảng. Đảng cần phát huy quyền làm chủ tập thể của nhân dân lao động, thực hiện "dân biết, dân bàn, dân làm, dân kiểm tra"; tăng cường hiệu lực quản lý của Nhà nước là điều kiện tất yếu để huy động lực lượng của quần chúng.

Đại hội VI của Đảng là Đại hội khởi xướng đường lối đổi mới toàn diện, đánh dấu bước ngoặt phát triển mới trong thời kỳ quá độ lên chủ nghĩa xã hội. Các văn kiện của Đại hội mang tính chất khoa học và cách mạng, tạo bước ngoặt cho sự phát triển của cách mạng Việt Nam. Tuy nhiên, hạn chế của Đại hội VI là chưa tìm ra những giải pháp hiệu quả tháo gỡ tình trạng rối ren trong phân phối lưu thông.