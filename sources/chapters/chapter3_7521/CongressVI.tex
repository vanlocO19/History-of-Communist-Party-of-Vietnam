\subsection{Đổi mới toàn diện, đưa đất nước ra khỏi khủng hoảng kinh tế $-$ xã hội 1986 $-$ 1996}
\subsubsection{Đại hội đại biểu toàn quốc lần thứ VI và thực hiện đường lối đổi mới toàn diện}
Đại hội VI của Đảng diễn ra tại Hà Nội, từ ngày 15 đến ngày 18/12/1986, trong bối cảnh cuộc cách mạng khoa học $-$ kỹ thuật đang phát triển mạnh, xu thế đối thoại trên thế giới đang dần thay thế xu thế đối đầu. Đổi mới đã trở thành xu thế của thời đại. Liên Xô và các nước xã hội chủ nghĩa đều tiến hành cải tổ sự nghiệp xây dựng chủ nghĩa xã hội. Việt Nam đang bị các đế quốc và thế lực thù địch bao vây, cấm vận và ở tình trạng khủng hoảng kinh tế $-$ xã hội. Lương thực, thực phẩm, hàng tiêu dùng đều khan hiếm; lạm phát tăng $300\%$ năm 1985 lên $774\%$ năm 1986. Các hiện tượng tiêu cực, vi phạm pháp luật, vượt biên trái phép diễn ra khá phổ biến. Đổi mới đã trở thành đòi hỏi bức thiết của tình hình đất nước.

Dự Đại hội có 1129 đại biểu thay mặt cho gần 2 triệu đảng viên cả nước và có 32 đoàn đại biểu quốc tế đến dự. Đại hội đã thông qua các văn kiện chính trị quan trọng, khởi xướng đường lối đổi mới toàn diện, bầu Ban Chấp hành Trung ương gồm 124 ủy viên chính thức, bầu Bộ Chính trị gồm 13 ủy viên chính thức, bầu đồng chí Nguyễn Văn Linh làm Tổng Bí thư của Đảng.