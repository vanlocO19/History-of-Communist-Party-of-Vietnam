\section*{Lời mở đầu}
Thực hiện kết luận số 94-KL/TW của Ban Bí thư Trung ương Đảng, ngày 28/3/2014, "Về tiếp tục đổi mới học tập lý luận chính trị trong hệ thống giáo dục quốc dân"; thực hiện Quyết định số 5001/QĐ-BGDĐT của Bộ trưởng Bộ Giáo dục và Đào tạo, ngày 29/11/2017, về việc thành lập Hội đồng biên soạn chương trình, giáo trình môn học \textit{Lịch sử Đảng Cộng sản Việt Nam} của các chuyên ngành đào tạo chuyên và không chuyên về lý luận chính trị trình độ đại học, nhiệm vụ biên soạn đã được \textit{Hội đồng biên soạn} triển khai nghiêm túc, đúng tiến độ theo định hướng của lãnh đạo Ban Tuyên giáo Trung ương, lãnh đạo Bộ Giáo dục và Đào tạo, trực tiếp là Ban Chỉ đạo.

Quá trình biên soạn giáo trình, Hội đồng đã kế thừa các giáo trình Lịch sử Đảng Cộng sản Việt Nam của Hội đồng Trung ương chỉ đạo biên soạn giáo trình quốc gia các môn lý luận Mác $-$ Lênin và tư tưởng Hồ Chí Minh, giáo trình của Bộ Giáo dục và Đào tạo và giáo trình của Học viện Chính trị quốc gia Hồ Chí Minh. Giáo trình biên soạn lần này cho cả hai hệ cố gắng thể hiện rõ những kết quả nghiên cứu mới của khoa học Lịch sử Đảng Cộng sản Việt Nam, những tổng kết và kết luận của các Đại hội Đảng toàn quốc và một số Hội nghị Trung ương, bảo đảm tính Đảng và tính khoa học. Với hệ chuyên lý luận chính trị, nội dung sâu hơn, nhất là các Cương lĩnh và những bài học lớn trong sự lãnh đạo của Đảng hướng vào làm rõ một số vấn đề có tính quy luật, lý luận của cách mạng Việt Nam.

Các Cương lĩnh của Đảng (2/1930, 10/1930, 3/1951, 6/1991 và Cương lĩnh xây dựng đất nước trong thời kỳ quá độ lên chủ nghĩa xã hội (bổ sung, phát triển năm 2011)) được trình bày gắn với các chương về các thời kỳ lịch sử. Nhiệm vụ xây dựng chủ nghĩa xã hội và bảo vệ miền Bắc (1954 $-$ 1975) được trình bày trong Chương 2 về hai cuộc kháng chiến giành độc lập hoàn toàn và thống nhất Tổ quốc.

Hội đồng biên soạn đã nhận được sự quan tâm chỉ đạo của đồng chí Võ Văn Thưởng, Ủy viên Bộ Chính trị khóa XII, Bí thư Trung ương Đảng khóa XII, Trưởng Ban Tuyên Giáo Trung ương khóa XII; GS. TS. Phùng Xuân Nhạ, Ủy viên Trung ương Đảng khóa XII, Bộ trưởng Bộ Giáo dục và Đào tạo nhiệm kỳ 2016 $-$ 2021; PGS. TS. Phạm Văn Linh, Phó Chủ tịch Hội đồng Lý luận Trung ương khóa XII, Trưởng Ban Chỉ đạo biên soạn giáo trình các môn Lý luận chính trị. 

Hội đồng biên soạn đã nhận được ý kiến đóng góp cả về nội dung và kết cấu giáo trình của các thầy, cô giáo trực tiếp giảng dạy môn Lịch sử Đảng Cộng sản Việt Nam ở các Học viện, trường đại học trong cả nước qua các cuộc hội thảo, tọa đàm, tập huấn và giảng dạy thí điểm. Hội đồng đã nhận được ý kiến đóng góp trực tiếp của các nhà khoa học: GS. TS. Phùng Hữu Phú, GS. TS. Nguyễn Ngọc Cơ, PSG. TS. Trần Đức Cường, PGS. TS. Đoàn Ngọc Hải, PGS. TS. Trịnh Đình Tùng, PGS. TS. Bùi Kim Đỉnh, PGS. TS. Nguyễn Danh Tiên, PGS. TS. Nguyễn Xuân Tú, PGS. TS. Huỳnh Thị Gấm, TS. Đào Thị Bích Hồng, PGS. TS. Nguyễn Đình Lê, TS. Nguyễn Thị Thanh Huyền, TS. Nguyễn Đình Cả và nhiều nhà khoa học khác.

Mặc dù đã có nhiều cố gắng trong biên soạn, bổ sung, song giáo trình khó tránh khỏi những thiếu sót, hạn chế. Quá trình giảng dạy, học tập rất mong được tiếp tục bổ sung, tu chỉnh để không ngừng nâng cao chất lượng giáo trình môn học Lịch sử Đảng Cộng sản Việt Nam.

\begin{flushright}
\textbf{HỘI ĐỒNG BIÊN SOẠN}
\end{flushright}